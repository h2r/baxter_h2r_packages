\documentclass{book}
\usepackage[a4paper,top=2.5cm,bottom=2.5cm,left=2.5cm,right=2.5cm]{geometry}
\usepackage{makeidx}
\usepackage{natbib}
\usepackage{graphicx}
\usepackage{multicol}
\usepackage{float}
\usepackage{listings}
\usepackage{color}
\usepackage{ifthen}
\usepackage[table]{xcolor}
\usepackage{textcomp}
\usepackage{alltt}
\usepackage{ifpdf}
\ifpdf
\usepackage[pdftex,
            pagebackref=true,
            colorlinks=true,
            linkcolor=blue,
            unicode
           ]{hyperref}
\else
\usepackage[ps2pdf,
            pagebackref=true,
            colorlinks=true,
            linkcolor=blue,
            unicode
           ]{hyperref}
\usepackage{pspicture}
\fi
\usepackage[utf8]{inputenc}
\usepackage{mathptmx}
\usepackage[scaled=.90]{helvet}
\usepackage{courier}
\usepackage{sectsty}
\usepackage{amssymb}
\usepackage[titles]{tocloft}
\usepackage{doxygen}
\lstset{language=C++,inputencoding=utf8,basicstyle=\footnotesize,breaklines=true,breakatwhitespace=true,tabsize=4,numbers=left }
\makeindex
\setcounter{tocdepth}{3}
\renewcommand{\footrulewidth}{0.4pt}
\renewcommand{\familydefault}{\sfdefault}
\hfuzz=15pt
\setlength{\emergencystretch}{15pt}
\hbadness=750
\tolerance=750
\begin{document}
\hypersetup{pageanchor=false,citecolor=blue}
\begin{titlepage}
\vspace*{7cm}
\begin{center}
{\Large My Project }\\
\vspace*{1cm}
{\large Generated by Doxygen 1.8.3.1}\\
\vspace*{0.5cm}
{\small Sat Mar 29 2014 16:12:35}\\
\end{center}
\end{titlepage}
\clearemptydoublepage
\pagenumbering{roman}
\tableofcontents
\clearemptydoublepage
\pagenumbering{arabic}
\hypersetup{pageanchor=true,citecolor=blue}
\chapter{Module Index}
\section{Modules}
Here is a list of all modules\-:\begin{DoxyCompactList}
\item \contentsline{section}{Matlab M\-A\-T File I/\-O Library}{\pageref{group__MAT}}{}
\item \contentsline{section}{M\-A\-T File I/\-O Utitlity Functions}{\pageref{group__mat__util}}{}
\end{DoxyCompactList}

\chapter{Class Index}
\section{Class List}
Here are the classes, structs, unions and interfaces with brief descriptions\-:\begin{DoxyCompactList}
\item\contentsline{section}{\hyperlink{struct__mat__t}{\-\_\-mat\-\_\-t} }{\pageref{struct__mat__t}}{}
\item\contentsline{section}{\hyperlink{structComplexSplit}{Complex\-Split} \\*Complex data type using split storage }{\pageref{structComplexSplit}}{}
\item\contentsline{section}{\hyperlink{structmatvar__internal}{matvar\-\_\-internal} }{\pageref{structmatvar__internal}}{}
\item\contentsline{section}{\hyperlink{structmatvar__t}{matvar\-\_\-t} \\*Matlab variable information }{\pageref{structmatvar__t}}{}
\item\contentsline{section}{\hyperlink{structsparse__t}{sparse\-\_\-t} \\*Sparse data information }{\pageref{structsparse__t}}{}
\end{DoxyCompactList}

\chapter{File Index}
\section{File List}
Here is a list of all documented files with brief descriptions:\begin{DoxyCompactList}
\item\contentsline{section}{/home/xren/work/kerneldescriptor/KernelDescriptors\_\-CPU/dependencies/matio/src/\hyperlink{endian_8c}{endian.c} (Functions to handle endian specifics )}{\pageref{endian_8c}}{}
\item\contentsline{section}{/home/xren/work/kerneldescriptor/KernelDescriptors\_\-CPU/dependencies/matio/src/\hyperlink{inflate_8c}{inflate.c} (Functions to inflate data/tags )}{\pageref{inflate_8c}}{}
\item\contentsline{section}{/home/xren/work/kerneldescriptor/KernelDescriptors\_\-CPU/dependencies/matio/src/\hyperlink{io_8c}{io.c} }{\pageref{io_8c}}{}
\item\contentsline{section}{/home/xren/work/kerneldescriptor/KernelDescriptors\_\-CPU/dependencies/matio/src/\hyperlink{mat_8c}{mat.c} }{\pageref{mat_8c}}{}
\item\contentsline{section}{/home/xren/work/kerneldescriptor/KernelDescriptors\_\-CPU/dependencies/matio/src/\hyperlink{mat4_8c}{mat4.c} }{\pageref{mat4_8c}}{}
\item\contentsline{section}{/home/xren/work/kerneldescriptor/KernelDescriptors\_\-CPU/dependencies/matio/src/{\bfseries mat4.h} }{\pageref{mat4_8h}}{}
\item\contentsline{section}{/home/xren/work/kerneldescriptor/KernelDescriptors\_\-CPU/dependencies/matio/src/\hyperlink{mat5_8c}{mat5.c} }{\pageref{mat5_8c}}{}
\item\contentsline{section}{/home/xren/work/kerneldescriptor/KernelDescriptors\_\-CPU/dependencies/matio/src/{\bfseries mat5.h} }{\pageref{mat5_8h}}{}
\item\contentsline{section}{/home/xren/work/kerneldescriptor/KernelDescriptors\_\-CPU/dependencies/matio/src/{\bfseries mat73.h} }{\pageref{mat73_8h}}{}
\item\contentsline{section}{/home/xren/work/kerneldescriptor/KernelDescriptors\_\-CPU/dependencies/matio/src/\hyperlink{matio_8h}{matio.h} }{\pageref{matio_8h}}{}
\item\contentsline{section}{/home/xren/work/kerneldescriptor/KernelDescriptors\_\-CPU/dependencies/matio/src/{\bfseries matio\_\-private.h} }{\pageref{matio__private_8h}}{}
\item\contentsline{section}{/home/xren/work/kerneldescriptor/KernelDescriptors\_\-CPU/dependencies/matio/src/{\bfseries matio\_\-pubconf.h} }{\pageref{matio__pubconf_8h}}{}
\item\contentsline{section}{/home/xren/work/kerneldescriptor/KernelDescriptors\_\-CPU/dependencies/matio/src/{\bfseries matioConfig.h} }{\pageref{matioConfig_8h}}{}
\item\contentsline{section}{/home/xren/work/kerneldescriptor/KernelDescriptors\_\-CPU/dependencies/matio/src/\hyperlink{read__data_8c}{read\_\-data.c} }{\pageref{read__data_8c}}{}
\end{DoxyCompactList}

\chapter{Module Documentation}
\hypertarget{group__MAT}{
\section{Matlab MAT File I/O Library}
\label{group__MAT}\index{Matlab MAT File I/O Library@{Matlab MAT File I/O Library}}
}
\subsection*{Classes}
\begin{DoxyCompactItemize}
\item 
struct \hyperlink{structComplexSplit}{ComplexSplit}
\begin{DoxyCompactList}\small\item\em Complex data type using split storage. \item\end{DoxyCompactList}\item 
struct \hyperlink{structmatvar__t}{matvar\_\-t}
\begin{DoxyCompactList}\small\item\em Matlab variable information. \item\end{DoxyCompactList}\item 
struct \hyperlink{structsparse__t}{sparse\_\-t}
\begin{DoxyCompactList}\small\item\em sparse data information \item\end{DoxyCompactList}\end{DoxyCompactItemize}
\subsection*{Files}
\begin{DoxyCompactItemize}
\item 
file \hyperlink{inflate_8c}{inflate.c}


\begin{DoxyCompactList}\small\item\em Functions to inflate data/tags. \item\end{DoxyCompactList}

\item 
file \hyperlink{mat_8c}{mat.c}
\item 
file \hyperlink{mat4_8c}{mat4.c}
\item 
file \hyperlink{mat5_8c}{mat5.c}
\item 
file \hyperlink{mat5_8c}{mat5.c}
\item 
file \hyperlink{matio_8h}{matio.h}
\item 
file \hyperlink{read__data_8c}{read\_\-data.c}
\end{DoxyCompactItemize}
\subsection*{Typedefs}
\begin{DoxyCompactItemize}
\item 
\hypertarget{group__MAT_gab0fc888f5a5d79943b16284b1f91c2e8}{
typedef struct \hyperlink{struct__mat__t}{\_\-mat\_\-t} \hyperlink{group__MAT_gab0fc888f5a5d79943b16284b1f91c2e8}{mat\_\-t}}
\label{group__MAT_gab0fc888f5a5d79943b16284b1f91c2e8}

\begin{DoxyCompactList}\small\item\em Matlab MAT File information Contains information about a Matlab MAT file. \item\end{DoxyCompactList}\item 
typedef struct \hyperlink{structmatvar__t}{matvar\_\-t} \hyperlink{group__MAT_ga24775c96a2a6d073581639c780b7896c}{matvar\_\-t}
\begin{DoxyCompactList}\small\item\em Matlab variable information. \item\end{DoxyCompactList}\item 
typedef struct \hyperlink{structsparse__t}{sparse\_\-t} \hyperlink{group__MAT_ga3ce6ed53a1909e27e92f3eaffc2f92ed}{sparse\_\-t}
\begin{DoxyCompactList}\small\item\em sparse data information \item\end{DoxyCompactList}\end{DoxyCompactItemize}
\subsection*{Enumerations}
\begin{DoxyCompactItemize}
\item 
enum \hyperlink{group__MAT_gaa9dcbc70f538af79bd557593ff6b5cdb}{mat\_\-acc} \{ \hyperlink{group__MAT_ggaa9dcbc70f538af79bd557593ff6b5cdba8dd1457651b27ba9bea6cfba158c037c}{MAT\_\-ACC\_\-RDONLY} =  0, 
\hyperlink{group__MAT_ggaa9dcbc70f538af79bd557593ff6b5cdba0f65f27ea42fde32d62b702b82329c1f}{MAT\_\-ACC\_\-RDWR} =  1
 \}
\begin{DoxyCompactList}\small\item\em MAT file access types. \item\end{DoxyCompactList}\item 
enum \hyperlink{group__MAT_gad03442b8378999189d510e3745c702b7}{mat\_\-ft} \{ \hyperlink{group__MAT_ggad03442b8378999189d510e3745c702b7a765c5d1d5038947646260dc82483517e}{MAT\_\-FT\_\-MAT73} =  0x0200, 
\hyperlink{group__MAT_ggad03442b8378999189d510e3745c702b7a31ade1f6989411dc0299007e2c7d33b2}{MAT\_\-FT\_\-MAT5} =  0x0100, 
\hyperlink{group__MAT_ggad03442b8378999189d510e3745c702b7a858b4f5da65548219b1c3ad47aa478d3}{MAT\_\-FT\_\-MAT4} =  0x0010
 \}
\begin{DoxyCompactList}\small\item\em MAT file versions. \item\end{DoxyCompactList}\item 
enum \hyperlink{group__MAT_gacf7b3b879282b7ab3a51190e49bf3453}{matio\_\-types} \{ \par
\hyperlink{group__MAT_ggacf7b3b879282b7ab3a51190e49bf3453a2a7318fe8bf9464935e7ed8902618293}{MAT\_\-T\_\-UNKNOWN} =  0, 
\hyperlink{group__MAT_ggacf7b3b879282b7ab3a51190e49bf3453a9807f5033ed4f9b548953742d9fd1658}{MAT\_\-T\_\-INT8} =  1, 
\hyperlink{group__MAT_ggacf7b3b879282b7ab3a51190e49bf3453a01c1bd7db68f90552862eb5d311be408}{MAT\_\-T\_\-UINT8} =  2, 
\hyperlink{group__MAT_ggacf7b3b879282b7ab3a51190e49bf3453a8c5b2e381946e95ea8d81ac216743302}{MAT\_\-T\_\-INT16} =  3, 
\par
\hyperlink{group__MAT_ggacf7b3b879282b7ab3a51190e49bf3453a05bc7af7680aa68be95126ae0a4c2e31}{MAT\_\-T\_\-UINT16} =  4, 
\hyperlink{group__MAT_ggacf7b3b879282b7ab3a51190e49bf3453a83e06a68320726c6572bfbb9f3addb1d}{MAT\_\-T\_\-INT32} =  5, 
\hyperlink{group__MAT_ggacf7b3b879282b7ab3a51190e49bf3453aa397e285a23fe240368b752897652c6a}{MAT\_\-T\_\-UINT32} =  6, 
\hyperlink{group__MAT_ggacf7b3b879282b7ab3a51190e49bf3453a3a3657d40e9212c923d9b9d03531b64c}{MAT\_\-T\_\-SINGLE} =  7, 
\par
\hyperlink{group__MAT_ggacf7b3b879282b7ab3a51190e49bf3453a31e721ecf7e188196f83c32838288797}{MAT\_\-T\_\-DOUBLE} =  9, 
\hyperlink{group__MAT_ggacf7b3b879282b7ab3a51190e49bf3453a9e825b5d18b8f946eaf2b4b57e51c145}{MAT\_\-T\_\-INT64} =  12, 
\hyperlink{group__MAT_ggacf7b3b879282b7ab3a51190e49bf3453a45547932c46be27118abe08302d7e29f}{MAT\_\-T\_\-UINT64} =  13, 
\hyperlink{group__MAT_ggacf7b3b879282b7ab3a51190e49bf3453a32985fee89a4df8db4b3f5d3a48823d3}{MAT\_\-T\_\-MATRIX} =  14, 
\par
\hyperlink{group__MAT_ggacf7b3b879282b7ab3a51190e49bf3453a30437f2eb3becc2fa6e5d96599d7f724}{MAT\_\-T\_\-COMPRESSED} =  15, 
\hyperlink{group__MAT_ggacf7b3b879282b7ab3a51190e49bf3453ac34ad81f5cbd3b7d0d95e57e5be0149b}{MAT\_\-T\_\-UTF8} =  16, 
\hyperlink{group__MAT_ggacf7b3b879282b7ab3a51190e49bf3453a87ffc0412143c326a1fcc759d5d81bdc}{MAT\_\-T\_\-UTF16} =  17, 
\hyperlink{group__MAT_ggacf7b3b879282b7ab3a51190e49bf3453a11e43c0e0be79b1983090e02ae583109}{MAT\_\-T\_\-UTF32} =  18, 
\par
\hyperlink{group__MAT_ggacf7b3b879282b7ab3a51190e49bf3453a9456a83c0b22022af42461a09d63cdb2}{MAT\_\-T\_\-STRING} =  20, 
\hyperlink{group__MAT_ggacf7b3b879282b7ab3a51190e49bf3453a07599cf2cca6d2b2d059378563318ba5}{MAT\_\-T\_\-CELL} =  21, 
\hyperlink{group__MAT_ggacf7b3b879282b7ab3a51190e49bf3453a4f4d5a6e1d42c6aa81ffb810e5da5c85}{MAT\_\-T\_\-STRUCT} =  22, 
\hyperlink{group__MAT_ggacf7b3b879282b7ab3a51190e49bf3453acf106b0c23021582375f59bc9fce89b1}{MAT\_\-T\_\-ARRAY} =  23, 
\par
\hyperlink{group__MAT_ggacf7b3b879282b7ab3a51190e49bf3453ae76686f267dd1641cd55dce306af6d10}{MAT\_\-T\_\-FUNCTION} =  24
 \}
\begin{DoxyCompactList}\small\item\em Matlab data types. \item\end{DoxyCompactList}\item 
enum \hyperlink{group__MAT_gad4d60ae7b709fc81bfd744fb4c857c40}{matio\_\-classes} \{ \par
\hyperlink{group__MAT_ggad4d60ae7b709fc81bfd744fb4c857c40a5c76eef0ca0373d25abe49053be6fa9a}{MAT\_\-C\_\-EMPTY} =  0, 
\hyperlink{group__MAT_ggad4d60ae7b709fc81bfd744fb4c857c40a2f7abb47a1c51e248bd4e5e03cc81b08}{MAT\_\-C\_\-CELL} =  1, 
\hyperlink{group__MAT_ggad4d60ae7b709fc81bfd744fb4c857c40acb467c7749c80902b798134c729bb521}{MAT\_\-C\_\-STRUCT} =  2, 
\hyperlink{group__MAT_ggad4d60ae7b709fc81bfd744fb4c857c40afe45104b68b044c83a2f99e349fa1ea6}{MAT\_\-C\_\-OBJECT} =  3, 
\par
\hyperlink{group__MAT_ggad4d60ae7b709fc81bfd744fb4c857c40aacdec5834df0861130b393697646119c}{MAT\_\-C\_\-CHAR} =  4, 
\hyperlink{group__MAT_ggad4d60ae7b709fc81bfd744fb4c857c40a0d5655b7e6178a2242cb3bb56ff4c8d2}{MAT\_\-C\_\-SPARSE} =  5, 
\hyperlink{group__MAT_ggad4d60ae7b709fc81bfd744fb4c857c40a5d70e0862e5bdb7bd86bf7ba5948f307}{MAT\_\-C\_\-DOUBLE} =  6, 
\hyperlink{group__MAT_ggad4d60ae7b709fc81bfd744fb4c857c40a2825631e26a961cbe0f79db50a39cea2}{MAT\_\-C\_\-SINGLE} =  7, 
\par
\hyperlink{group__MAT_ggad4d60ae7b709fc81bfd744fb4c857c40a984ff310f9e906100fcff95f704f43c5}{MAT\_\-C\_\-INT8} =  8, 
\hyperlink{group__MAT_ggad4d60ae7b709fc81bfd744fb4c857c40a81270f8093cb4808e992c1d29d84d4e3}{MAT\_\-C\_\-UINT8} =  9, 
\hyperlink{group__MAT_ggad4d60ae7b709fc81bfd744fb4c857c40a40370e9de516c5036a67a5865c071006}{MAT\_\-C\_\-INT16} =  10, 
\hyperlink{group__MAT_ggad4d60ae7b709fc81bfd744fb4c857c40a8bede21dbf6c1edc0bbccc1481bccae7}{MAT\_\-C\_\-UINT16} =  11, 
\par
\hyperlink{group__MAT_ggad4d60ae7b709fc81bfd744fb4c857c40adb44fc39694e3152ae5e69470a2fefe8}{MAT\_\-C\_\-INT32} =  12, 
\hyperlink{group__MAT_ggad4d60ae7b709fc81bfd744fb4c857c40a9a17a7edd45b19ef68197db81b27e816}{MAT\_\-C\_\-UINT32} =  13, 
\hyperlink{group__MAT_ggad4d60ae7b709fc81bfd744fb4c857c40a1ea83bcde49b35477494412973f82409}{MAT\_\-C\_\-INT64} =  14, 
\hyperlink{group__MAT_ggad4d60ae7b709fc81bfd744fb4c857c40a86470e25c3763d9a24623f04326195dd}{MAT\_\-C\_\-UINT64} =  15, 
\par
\hyperlink{group__MAT_ggad4d60ae7b709fc81bfd744fb4c857c40aaa9bf08312779cd1ab8e504a162ddcea}{MAT\_\-C\_\-FUNCTION} =  16
 \}
\begin{DoxyCompactList}\small\item\em Matlab variable classes. \item\end{DoxyCompactList}\item 
enum \hyperlink{group__MAT_gab9d6ef9e3ddca78a317b173f01d53fbb}{matio\_\-flags} \{ \hyperlink{group__MAT_ggab9d6ef9e3ddca78a317b173f01d53fbbacd7b091a11184aad7fc6078c04470780}{MAT\_\-F\_\-COMPLEX} =  0x0800, 
\hyperlink{group__MAT_ggab9d6ef9e3ddca78a317b173f01d53fbba49084e0c796aa7963e53f7539525d40d}{MAT\_\-F\_\-GLOBAL} =  0x0400, 
\hyperlink{group__MAT_ggab9d6ef9e3ddca78a317b173f01d53fbba57eb5c6e200bcbc0f1b7982f29a169c2}{MAT\_\-F\_\-LOGICAL} =  0x0200, 
\hyperlink{group__MAT_ggab9d6ef9e3ddca78a317b173f01d53fbba3a88beaec448e0485ffe21b18a540c1d}{MAT\_\-F\_\-CLASS\_\-T} =  0x00ff
 \}
\begin{DoxyCompactList}\small\item\em Matlab array flags. \item\end{DoxyCompactList}\item 
enum \hyperlink{group__MAT_ga768c318af97bd2567758ecb001ceb7f4}{matio\_\-compression} \{ \hyperlink{group__MAT_gga768c318af97bd2567758ecb001ceb7f4ac549b871996d1ef05d40056bf5bb52e5}{COMPRESSION\_\-NONE} =  0, 
\hyperlink{group__MAT_gga768c318af97bd2567758ecb001ceb7f4a1f453c9a2c01b52294b37a1226837f86}{COMPRESSION\_\-ZLIB} =  1
 \}
\begin{DoxyCompactList}\small\item\em Matlab compression options. \item\end{DoxyCompactList}\item 
enum \{ \hyperlink{group__MAT_gga06fc87d81c62e9abb8790b6e5713c55ba8938378c70879fe916177141cce0417e}{BY\_\-NAME} =  1, 
\hyperlink{group__MAT_gga06fc87d81c62e9abb8790b6e5713c55ba5f4d5606de1ec27f80f4a50186909005}{BY\_\-INDEX} =  2
 \}
\end{DoxyCompactItemize}
\subsection*{Functions}
\begin{DoxyCompactItemize}
\item 
\hyperlink{struct__mat__t}{mat\_\-t} $\ast$ \hyperlink{group__MAT_ga22d404f203af7869c841400e7ad247cf}{Mat\_\-CreateVer} (const char $\ast$matname, const char $\ast$hdr\_\-str, enum \hyperlink{group__MAT_gad03442b8378999189d510e3745c702b7}{mat\_\-ft} mat\_\-file\_\-ver)
\begin{DoxyCompactList}\small\item\em Creates a new Matlab MAT file. \item\end{DoxyCompactList}\item 
\hyperlink{struct__mat__t}{mat\_\-t} $\ast$ \hyperlink{group__MAT_gafbfedb5636a99f0ef867520c47f77d18}{Mat\_\-Open} (const char $\ast$matname, int mode)
\begin{DoxyCompactList}\small\item\em Opens an existing Matlab MAT file. \item\end{DoxyCompactList}\item 
int \hyperlink{group__MAT_ga101c92ff7bde4a2d4615661beba09262}{Mat\_\-Close} (\hyperlink{struct__mat__t}{mat\_\-t} $\ast$mat)
\begin{DoxyCompactList}\small\item\em Closes an open Matlab MAT file. \item\end{DoxyCompactList}\item 
int \hyperlink{group__MAT_ga4d6e3892d2e216c507a744ba0e070d0b}{Mat\_\-Rewind} (\hyperlink{struct__mat__t}{mat\_\-t} $\ast$mat)
\begin{DoxyCompactList}\small\item\em Rewinds a Matlab MAT file to the first variable. \item\end{DoxyCompactList}\item 
size\_\-t \hyperlink{group__MAT_ga2bf682f015b22fa796a8885e997661e7}{Mat\_\-SizeOfClass} (int class\_\-type)
\begin{DoxyCompactList}\small\item\em Returns the size of a Matlab Class. \item\end{DoxyCompactList}\item 
\hyperlink{structmatvar__t}{matvar\_\-t} $\ast$ \hyperlink{group__MAT_gae7c9c3699f6e9c31a9c490300013098c}{Mat\_\-VarCalloc} (void)
\begin{DoxyCompactList}\small\item\em Allocates memory for a new \hyperlink{structmatvar__t}{matvar\_\-t} and initializes all the fields. \item\end{DoxyCompactList}\item 
\hyperlink{structmatvar__t}{matvar\_\-t} $\ast$ \hyperlink{group__MAT_ga1c54a84bb4d810c6fccdb8869489eac4}{Mat\_\-VarCreate} (const char $\ast$name, enum \hyperlink{group__MAT_gad4d60ae7b709fc81bfd744fb4c857c40}{matio\_\-classes} class\_\-type, enum \hyperlink{group__MAT_gacf7b3b879282b7ab3a51190e49bf3453}{matio\_\-types} data\_\-type, int rank, size\_\-t $\ast$dims, void $\ast$data, int opt)
\begin{DoxyCompactList}\small\item\em Creates a MAT Variable with the given name and (optionally) data. \item\end{DoxyCompactList}\item 
int \hyperlink{group__MAT_gabf139e48d48177e5069338fa2919c60a}{Mat\_\-VarDelete} (\hyperlink{struct__mat__t}{mat\_\-t} $\ast$mat, const char $\ast$name)
\begin{DoxyCompactList}\small\item\em Deletes a variable from a file. \item\end{DoxyCompactList}\item 
\hyperlink{structmatvar__t}{matvar\_\-t} $\ast$ \hyperlink{group__MAT_ga7ef80c5d99d7918b2b09db3bea106ecc}{Mat\_\-VarDuplicate} (const \hyperlink{structmatvar__t}{matvar\_\-t} $\ast$in, int opt)
\begin{DoxyCompactList}\small\item\em Duplicates a \hyperlink{structmatvar__t}{matvar\_\-t} structure. \item\end{DoxyCompactList}\item 
void \hyperlink{group__MAT_ga1d14716f7450530fd1c9d02413787f0e}{Mat\_\-VarFree} (\hyperlink{structmatvar__t}{matvar\_\-t} $\ast$matvar)
\begin{DoxyCompactList}\small\item\em Frees all the allocated memory associated with the structure. \item\end{DoxyCompactList}\item 
int \hyperlink{group__MAT_ga9b8d09f631538b14ca29792e0334e349}{Mat\_\-CalcSingleSubscript} (int rank, int $\ast$dims, int $\ast$subs)
\begin{DoxyCompactList}\small\item\em Calculate a single subscript from a set of subscript values. \item\end{DoxyCompactList}\item 
int $\ast$ \hyperlink{group__MAT_gabe2571a4b9b6cff3b31aa6f152deba61}{Mat\_\-CalcSubscripts} (int rank, int $\ast$dims, int index)
\begin{DoxyCompactList}\small\item\em Calculate a set of subscript values from a single(linear) subscript. \item\end{DoxyCompactList}\item 
\hyperlink{structmatvar__t}{matvar\_\-t} $\ast$ \hyperlink{group__MAT_gac1e15063439c0bd3eb0c986514c742dc}{Mat\_\-VarGetCell} (\hyperlink{structmatvar__t}{matvar\_\-t} $\ast$matvar, int index)
\begin{DoxyCompactList}\small\item\em Returns a pointer to the Cell array at a specific index. \item\end{DoxyCompactList}\item 
\hyperlink{structmatvar__t}{matvar\_\-t} $\ast$$\ast$ \hyperlink{group__MAT_ga0732b0a6c40975b036068b9a14422d45}{Mat\_\-VarGetCells} (\hyperlink{structmatvar__t}{matvar\_\-t} $\ast$matvar, int $\ast$start, int $\ast$stride, int $\ast$edge)
\begin{DoxyCompactList}\small\item\em Indexes a cell array. \item\end{DoxyCompactList}\item 
\hyperlink{structmatvar__t}{matvar\_\-t} $\ast$$\ast$ \hyperlink{group__MAT_ga004987d665654409f74eaf8e82bb1380}{Mat\_\-VarGetCellsLinear} (\hyperlink{structmatvar__t}{matvar\_\-t} $\ast$matvar, int start, int stride, int edge)
\begin{DoxyCompactList}\small\item\em Indexes a cell array. \item\end{DoxyCompactList}\item 
size\_\-t \hyperlink{group__MAT_gaeeb798fead2f765bddfb19016c7fdbcc}{Mat\_\-VarGetSize} (\hyperlink{structmatvar__t}{matvar\_\-t} $\ast$matvar)
\begin{DoxyCompactList}\small\item\em Calculates the size of a matlab variable in bytes. \item\end{DoxyCompactList}\item 
int \hyperlink{group__MAT_ga9f8ab8a7e4206c16cb20437acc6960d2}{Mat\_\-VarAddStructField} (\hyperlink{structmatvar__t}{matvar\_\-t} $\ast$matvar, \hyperlink{structmatvar__t}{matvar\_\-t} $\ast$$\ast$fields)
\begin{DoxyCompactList}\small\item\em Adds a field to a structure. \item\end{DoxyCompactList}\item 
int \hyperlink{group__MAT_ga56b9a545990a0f253164018e37111741}{Mat\_\-VarGetNumberOfFields} (\hyperlink{structmatvar__t}{matvar\_\-t} $\ast$matvar)
\begin{DoxyCompactList}\small\item\em Returns the number of fields in a structure variable. \item\end{DoxyCompactList}\item 
\hyperlink{structmatvar__t}{matvar\_\-t} $\ast$ \hyperlink{group__MAT_ga7018bfe6934b96ae32e8f2a1712eefab}{Mat\_\-VarGetStructField} (\hyperlink{structmatvar__t}{matvar\_\-t} $\ast$matvar, void $\ast$name\_\-or\_\-index, int opt, int index)
\begin{DoxyCompactList}\small\item\em Finds a field of a structure. \item\end{DoxyCompactList}\item 
\hyperlink{structmatvar__t}{matvar\_\-t} $\ast$ \hyperlink{group__MAT_ga509178d7dc15faf9f7cd0440df6009b9}{Mat\_\-VarGetStructs} (\hyperlink{structmatvar__t}{matvar\_\-t} $\ast$matvar, int $\ast$start, int $\ast$stride, int $\ast$edge, int copy\_\-fields)
\begin{DoxyCompactList}\small\item\em Indexes a structure. \item\end{DoxyCompactList}\item 
\hyperlink{structmatvar__t}{matvar\_\-t} $\ast$ \hyperlink{group__MAT_gaa56680fb7b2cd3d410f659e945da8141}{Mat\_\-VarGetStructsLinear} (\hyperlink{structmatvar__t}{matvar\_\-t} $\ast$matvar, int start, int stride, int edge, int copy\_\-fields)
\begin{DoxyCompactList}\small\item\em Indexes a structure. \item\end{DoxyCompactList}\item 
void \hyperlink{group__MAT_ga9100c145e338b84b55d5d0795d5d390a}{Mat\_\-VarPrint} (\hyperlink{structmatvar__t}{matvar\_\-t} $\ast$matvar, int printdata)
\begin{DoxyCompactList}\small\item\em Prints the variable information. \item\end{DoxyCompactList}\item 
int \hyperlink{group__MAT_ga1845000f4fc6252ec5ff11c4b9f0759f}{Mat\_\-VarReadData} (\hyperlink{struct__mat__t}{mat\_\-t} $\ast$mat, \hyperlink{structmatvar__t}{matvar\_\-t} $\ast$matvar, void $\ast$data, int $\ast$start, int $\ast$stride, int $\ast$edge)
\begin{DoxyCompactList}\small\item\em Reads MAT variable data from a file. \item\end{DoxyCompactList}\item 
int \hyperlink{group__MAT_gaa8060d7c8e5da0aa9ee5f96e5f1eb30a}{Mat\_\-VarReadDataAll} (\hyperlink{struct__mat__t}{mat\_\-t} $\ast$mat, \hyperlink{structmatvar__t}{matvar\_\-t} $\ast$matvar)
\begin{DoxyCompactList}\small\item\em Reads all the data for a matlab variable. \item\end{DoxyCompactList}\item 
int \hyperlink{group__MAT_gaad61c8449a2106afa697280ff0ee9dd8}{Mat\_\-VarReadDataLinear} (\hyperlink{struct__mat__t}{mat\_\-t} $\ast$mat, \hyperlink{structmatvar__t}{matvar\_\-t} $\ast$matvar, void $\ast$data, int start, int stride, int edge)
\begin{DoxyCompactList}\small\item\em Reads MAT variable data from a file. \item\end{DoxyCompactList}\item 
\hyperlink{structmatvar__t}{matvar\_\-t} $\ast$ \hyperlink{group__MAT_ga72dd99330507b17177e22f9ed3bea5e6}{Mat\_\-VarReadNextInfo} (\hyperlink{struct__mat__t}{mat\_\-t} $\ast$mat)
\begin{DoxyCompactList}\small\item\em Reads the information of the next variable in a MAT file. \item\end{DoxyCompactList}\item 
\hyperlink{structmatvar__t}{matvar\_\-t} $\ast$ \hyperlink{group__MAT_ga46da2e45ed96d3f1a6ec643757f2b086}{Mat\_\-VarReadInfo} (\hyperlink{struct__mat__t}{mat\_\-t} $\ast$mat, const char $\ast$name)
\begin{DoxyCompactList}\small\item\em Reads the information of a variable with the given name from a MAT file. \item\end{DoxyCompactList}\item 
\hyperlink{structmatvar__t}{matvar\_\-t} $\ast$ \hyperlink{group__MAT_ga3505f63029763eaa73d5a19f1115eb42}{Mat\_\-VarRead} (\hyperlink{struct__mat__t}{mat\_\-t} $\ast$mat, const char $\ast$name)
\begin{DoxyCompactList}\small\item\em Reads the variable with the given name from a MAT file. \item\end{DoxyCompactList}\item 
\hyperlink{structmatvar__t}{matvar\_\-t} $\ast$ \hyperlink{group__MAT_ga7c321d6aafd93916ba6c5655ad78e9ca}{Mat\_\-VarReadNext} (\hyperlink{struct__mat__t}{mat\_\-t} $\ast$mat)
\begin{DoxyCompactList}\small\item\em Reads the next variable in a MAT file. \item\end{DoxyCompactList}\item 
int \hyperlink{group__MAT_ga1ae164415dfd98cdf48ad07033b6e0bb}{Mat\_\-VarWriteInfo} (\hyperlink{struct__mat__t}{mat\_\-t} $\ast$mat, \hyperlink{structmatvar__t}{matvar\_\-t} $\ast$matvar)
\begin{DoxyCompactList}\small\item\em Writes the given MAT variable to a MAT file. \item\end{DoxyCompactList}\item 
int \hyperlink{group__MAT_ga43179b930fb30c025a153a55a083a98a}{Mat\_\-VarWriteData} (\hyperlink{struct__mat__t}{mat\_\-t} $\ast$mat, \hyperlink{structmatvar__t}{matvar\_\-t} $\ast$matvar, void $\ast$data, int $\ast$start, int $\ast$stride, int $\ast$edge)
\begin{DoxyCompactList}\small\item\em Writes the given data to the MAT variable. \item\end{DoxyCompactList}\item 
int \hyperlink{group__MAT_ga77c5ad24d45047830046fe3ed25da8ad}{Mat\_\-VarWrite} (\hyperlink{struct__mat__t}{mat\_\-t} $\ast$mat, \hyperlink{structmatvar__t}{matvar\_\-t} $\ast$matvar, int compress)
\begin{DoxyCompactList}\small\item\em Writes the given MAT variable to a MAT file. \item\end{DoxyCompactList}\end{DoxyCompactItemize}


\subsection{Typedef Documentation}
\hypertarget{group__MAT_ga24775c96a2a6d073581639c780b7896c}{
\index{MAT@{MAT}!matvar\_\-t@{matvar\_\-t}}
\index{matvar\_\-t@{matvar\_\-t}!MAT@{MAT}}
\subsubsection[{matvar\_\-t}]{\setlength{\rightskip}{0pt plus 5cm}typedef struct {\bf matvar\_\-t}  {\bf matvar\_\-t}}}
\label{group__MAT_ga24775c96a2a6d073581639c780b7896c}


Matlab variable information. 

Contains information about a Matlab variable \hypertarget{group__MAT_ga3ce6ed53a1909e27e92f3eaffc2f92ed}{
\index{MAT@{MAT}!sparse\_\-t@{sparse\_\-t}}
\index{sparse\_\-t@{sparse\_\-t}!MAT@{MAT}}
\subsubsection[{sparse\_\-t}]{\setlength{\rightskip}{0pt plus 5cm}typedef struct {\bf sparse\_\-t}  {\bf sparse\_\-t}}}
\label{group__MAT_ga3ce6ed53a1909e27e92f3eaffc2f92ed}


sparse data information 

Contains information and data for a sparse matrix 

\subsection{Enumeration Type Documentation}
\hypertarget{group__MAT_ga06fc87d81c62e9abb8790b6e5713c55b}{
\subsubsection[{"@0}]{\setlength{\rightskip}{0pt plus 5cm}anonymous enum}}
\label{group__MAT_ga06fc87d81c62e9abb8790b6e5713c55b}
matio lookup type \begin{Desc}
\item[Enumerator: ]\par
\begin{description}
\index{BY\_\-NAME@{BY\_\-NAME}!MAT@{MAT}}\index{MAT@{MAT}!BY\_\-NAME@{BY\_\-NAME}}\item[{\em 
\hypertarget{group__MAT_gga06fc87d81c62e9abb8790b6e5713c55ba8938378c70879fe916177141cce0417e}{
BY\_\-NAME}
\label{group__MAT_gga06fc87d81c62e9abb8790b6e5713c55ba8938378c70879fe916177141cce0417e}
}]Lookup by name \index{BY\_\-INDEX@{BY\_\-INDEX}!MAT@{MAT}}\index{MAT@{MAT}!BY\_\-INDEX@{BY\_\-INDEX}}\item[{\em 
\hypertarget{group__MAT_gga06fc87d81c62e9abb8790b6e5713c55ba5f4d5606de1ec27f80f4a50186909005}{
BY\_\-INDEX}
\label{group__MAT_gga06fc87d81c62e9abb8790b6e5713c55ba5f4d5606de1ec27f80f4a50186909005}
}]Lookup by index \end{description}
\end{Desc}

\hypertarget{group__MAT_gaa9dcbc70f538af79bd557593ff6b5cdb}{
\index{MAT@{MAT}!mat\_\-acc@{mat\_\-acc}}
\index{mat\_\-acc@{mat\_\-acc}!MAT@{MAT}}
\subsubsection[{mat\_\-acc}]{\setlength{\rightskip}{0pt plus 5cm}enum {\bf mat\_\-acc}}}
\label{group__MAT_gaa9dcbc70f538af79bd557593ff6b5cdb}


MAT file access types. 

MAT file access types \begin{Desc}
\item[Enumerator: ]\par
\begin{description}
\index{MAT\_\-ACC\_\-RDONLY@{MAT\_\-ACC\_\-RDONLY}!MAT@{MAT}}\index{MAT@{MAT}!MAT\_\-ACC\_\-RDONLY@{MAT\_\-ACC\_\-RDONLY}}\item[{\em 
\hypertarget{group__MAT_ggaa9dcbc70f538af79bd557593ff6b5cdba8dd1457651b27ba9bea6cfba158c037c}{
MAT\_\-ACC\_\-RDONLY}
\label{group__MAT_ggaa9dcbc70f538af79bd557593ff6b5cdba8dd1457651b27ba9bea6cfba158c037c}
}]Read only file access. \index{MAT\_\-ACC\_\-RDWR@{MAT\_\-ACC\_\-RDWR}!MAT@{MAT}}\index{MAT@{MAT}!MAT\_\-ACC\_\-RDWR@{MAT\_\-ACC\_\-RDWR}}\item[{\em 
\hypertarget{group__MAT_ggaa9dcbc70f538af79bd557593ff6b5cdba0f65f27ea42fde32d62b702b82329c1f}{
MAT\_\-ACC\_\-RDWR}
\label{group__MAT_ggaa9dcbc70f538af79bd557593ff6b5cdba0f65f27ea42fde32d62b702b82329c1f}
}]Read/Write file access. \end{description}
\end{Desc}

\hypertarget{group__MAT_gad03442b8378999189d510e3745c702b7}{
\index{MAT@{MAT}!mat\_\-ft@{mat\_\-ft}}
\index{mat\_\-ft@{mat\_\-ft}!MAT@{MAT}}
\subsubsection[{mat\_\-ft}]{\setlength{\rightskip}{0pt plus 5cm}enum {\bf mat\_\-ft}}}
\label{group__MAT_gad03442b8378999189d510e3745c702b7}


MAT file versions. 

MAT file versions \begin{Desc}
\item[Enumerator: ]\par
\begin{description}
\index{MAT\_\-FT\_\-MAT73@{MAT\_\-FT\_\-MAT73}!MAT@{MAT}}\index{MAT@{MAT}!MAT\_\-FT\_\-MAT73@{MAT\_\-FT\_\-MAT73}}\item[{\em 
\hypertarget{group__MAT_ggad03442b8378999189d510e3745c702b7a765c5d1d5038947646260dc82483517e}{
MAT\_\-FT\_\-MAT73}
\label{group__MAT_ggad03442b8378999189d510e3745c702b7a765c5d1d5038947646260dc82483517e}
}]Matlab version 7.3 file. \index{MAT\_\-FT\_\-MAT5@{MAT\_\-FT\_\-MAT5}!MAT@{MAT}}\index{MAT@{MAT}!MAT\_\-FT\_\-MAT5@{MAT\_\-FT\_\-MAT5}}\item[{\em 
\hypertarget{group__MAT_ggad03442b8378999189d510e3745c702b7a31ade1f6989411dc0299007e2c7d33b2}{
MAT\_\-FT\_\-MAT5}
\label{group__MAT_ggad03442b8378999189d510e3745c702b7a31ade1f6989411dc0299007e2c7d33b2}
}]Matlab level-\/5 file. \index{MAT\_\-FT\_\-MAT4@{MAT\_\-FT\_\-MAT4}!MAT@{MAT}}\index{MAT@{MAT}!MAT\_\-FT\_\-MAT4@{MAT\_\-FT\_\-MAT4}}\item[{\em 
\hypertarget{group__MAT_ggad03442b8378999189d510e3745c702b7a858b4f5da65548219b1c3ad47aa478d3}{
MAT\_\-FT\_\-MAT4}
\label{group__MAT_ggad03442b8378999189d510e3745c702b7a858b4f5da65548219b1c3ad47aa478d3}
}]Version 4 file. \end{description}
\end{Desc}

\hypertarget{group__MAT_gad4d60ae7b709fc81bfd744fb4c857c40}{
\index{MAT@{MAT}!matio\_\-classes@{matio\_\-classes}}
\index{matio\_\-classes@{matio\_\-classes}!MAT@{MAT}}
\subsubsection[{matio\_\-classes}]{\setlength{\rightskip}{0pt plus 5cm}enum {\bf matio\_\-classes}}}
\label{group__MAT_gad4d60ae7b709fc81bfd744fb4c857c40}


Matlab variable classes. 

Matlab variable classes \begin{Desc}
\item[Enumerator: ]\par
\begin{description}
\index{MAT\_\-C\_\-EMPTY@{MAT\_\-C\_\-EMPTY}!MAT@{MAT}}\index{MAT@{MAT}!MAT\_\-C\_\-EMPTY@{MAT\_\-C\_\-EMPTY}}\item[{\em 
\hypertarget{group__MAT_ggad4d60ae7b709fc81bfd744fb4c857c40a5c76eef0ca0373d25abe49053be6fa9a}{
MAT\_\-C\_\-EMPTY}
\label{group__MAT_ggad4d60ae7b709fc81bfd744fb4c857c40a5c76eef0ca0373d25abe49053be6fa9a}
}]Empty array. \index{MAT\_\-C\_\-CELL@{MAT\_\-C\_\-CELL}!MAT@{MAT}}\index{MAT@{MAT}!MAT\_\-C\_\-CELL@{MAT\_\-C\_\-CELL}}\item[{\em 
\hypertarget{group__MAT_ggad4d60ae7b709fc81bfd744fb4c857c40a2f7abb47a1c51e248bd4e5e03cc81b08}{
MAT\_\-C\_\-CELL}
\label{group__MAT_ggad4d60ae7b709fc81bfd744fb4c857c40a2f7abb47a1c51e248bd4e5e03cc81b08}
}]Matlab cell array class. \index{MAT\_\-C\_\-STRUCT@{MAT\_\-C\_\-STRUCT}!MAT@{MAT}}\index{MAT@{MAT}!MAT\_\-C\_\-STRUCT@{MAT\_\-C\_\-STRUCT}}\item[{\em 
\hypertarget{group__MAT_ggad4d60ae7b709fc81bfd744fb4c857c40acb467c7749c80902b798134c729bb521}{
MAT\_\-C\_\-STRUCT}
\label{group__MAT_ggad4d60ae7b709fc81bfd744fb4c857c40acb467c7749c80902b798134c729bb521}
}]Matlab structure class. \index{MAT\_\-C\_\-OBJECT@{MAT\_\-C\_\-OBJECT}!MAT@{MAT}}\index{MAT@{MAT}!MAT\_\-C\_\-OBJECT@{MAT\_\-C\_\-OBJECT}}\item[{\em 
\hypertarget{group__MAT_ggad4d60ae7b709fc81bfd744fb4c857c40afe45104b68b044c83a2f99e349fa1ea6}{
MAT\_\-C\_\-OBJECT}
\label{group__MAT_ggad4d60ae7b709fc81bfd744fb4c857c40afe45104b68b044c83a2f99e349fa1ea6}
}]Matlab object class. \index{MAT\_\-C\_\-CHAR@{MAT\_\-C\_\-CHAR}!MAT@{MAT}}\index{MAT@{MAT}!MAT\_\-C\_\-CHAR@{MAT\_\-C\_\-CHAR}}\item[{\em 
\hypertarget{group__MAT_ggad4d60ae7b709fc81bfd744fb4c857c40aacdec5834df0861130b393697646119c}{
MAT\_\-C\_\-CHAR}
\label{group__MAT_ggad4d60ae7b709fc81bfd744fb4c857c40aacdec5834df0861130b393697646119c}
}]Matlab character array class. \index{MAT\_\-C\_\-SPARSE@{MAT\_\-C\_\-SPARSE}!MAT@{MAT}}\index{MAT@{MAT}!MAT\_\-C\_\-SPARSE@{MAT\_\-C\_\-SPARSE}}\item[{\em 
\hypertarget{group__MAT_ggad4d60ae7b709fc81bfd744fb4c857c40a0d5655b7e6178a2242cb3bb56ff4c8d2}{
MAT\_\-C\_\-SPARSE}
\label{group__MAT_ggad4d60ae7b709fc81bfd744fb4c857c40a0d5655b7e6178a2242cb3bb56ff4c8d2}
}]Matlab sparse array class. \index{MAT\_\-C\_\-DOUBLE@{MAT\_\-C\_\-DOUBLE}!MAT@{MAT}}\index{MAT@{MAT}!MAT\_\-C\_\-DOUBLE@{MAT\_\-C\_\-DOUBLE}}\item[{\em 
\hypertarget{group__MAT_ggad4d60ae7b709fc81bfd744fb4c857c40a5d70e0862e5bdb7bd86bf7ba5948f307}{
MAT\_\-C\_\-DOUBLE}
\label{group__MAT_ggad4d60ae7b709fc81bfd744fb4c857c40a5d70e0862e5bdb7bd86bf7ba5948f307}
}]Matlab double-\/precision class. \index{MAT\_\-C\_\-SINGLE@{MAT\_\-C\_\-SINGLE}!MAT@{MAT}}\index{MAT@{MAT}!MAT\_\-C\_\-SINGLE@{MAT\_\-C\_\-SINGLE}}\item[{\em 
\hypertarget{group__MAT_ggad4d60ae7b709fc81bfd744fb4c857c40a2825631e26a961cbe0f79db50a39cea2}{
MAT\_\-C\_\-SINGLE}
\label{group__MAT_ggad4d60ae7b709fc81bfd744fb4c857c40a2825631e26a961cbe0f79db50a39cea2}
}]Matlab single-\/precision class. \index{MAT\_\-C\_\-INT8@{MAT\_\-C\_\-INT8}!MAT@{MAT}}\index{MAT@{MAT}!MAT\_\-C\_\-INT8@{MAT\_\-C\_\-INT8}}\item[{\em 
\hypertarget{group__MAT_ggad4d60ae7b709fc81bfd744fb4c857c40a984ff310f9e906100fcff95f704f43c5}{
MAT\_\-C\_\-INT8}
\label{group__MAT_ggad4d60ae7b709fc81bfd744fb4c857c40a984ff310f9e906100fcff95f704f43c5}
}]Matlab signed 8-\/bit integer class. \index{MAT\_\-C\_\-UINT8@{MAT\_\-C\_\-UINT8}!MAT@{MAT}}\index{MAT@{MAT}!MAT\_\-C\_\-UINT8@{MAT\_\-C\_\-UINT8}}\item[{\em 
\hypertarget{group__MAT_ggad4d60ae7b709fc81bfd744fb4c857c40a81270f8093cb4808e992c1d29d84d4e3}{
MAT\_\-C\_\-UINT8}
\label{group__MAT_ggad4d60ae7b709fc81bfd744fb4c857c40a81270f8093cb4808e992c1d29d84d4e3}
}]Matlab unsigned 8-\/bit integer class. \index{MAT\_\-C\_\-INT16@{MAT\_\-C\_\-INT16}!MAT@{MAT}}\index{MAT@{MAT}!MAT\_\-C\_\-INT16@{MAT\_\-C\_\-INT16}}\item[{\em 
\hypertarget{group__MAT_ggad4d60ae7b709fc81bfd744fb4c857c40a40370e9de516c5036a67a5865c071006}{
MAT\_\-C\_\-INT16}
\label{group__MAT_ggad4d60ae7b709fc81bfd744fb4c857c40a40370e9de516c5036a67a5865c071006}
}]Matlab signed 16-\/bit integer class. \index{MAT\_\-C\_\-UINT16@{MAT\_\-C\_\-UINT16}!MAT@{MAT}}\index{MAT@{MAT}!MAT\_\-C\_\-UINT16@{MAT\_\-C\_\-UINT16}}\item[{\em 
\hypertarget{group__MAT_ggad4d60ae7b709fc81bfd744fb4c857c40a8bede21dbf6c1edc0bbccc1481bccae7}{
MAT\_\-C\_\-UINT16}
\label{group__MAT_ggad4d60ae7b709fc81bfd744fb4c857c40a8bede21dbf6c1edc0bbccc1481bccae7}
}]Matlab unsigned 16-\/bit integer class. \index{MAT\_\-C\_\-INT32@{MAT\_\-C\_\-INT32}!MAT@{MAT}}\index{MAT@{MAT}!MAT\_\-C\_\-INT32@{MAT\_\-C\_\-INT32}}\item[{\em 
\hypertarget{group__MAT_ggad4d60ae7b709fc81bfd744fb4c857c40adb44fc39694e3152ae5e69470a2fefe8}{
MAT\_\-C\_\-INT32}
\label{group__MAT_ggad4d60ae7b709fc81bfd744fb4c857c40adb44fc39694e3152ae5e69470a2fefe8}
}]Matlab signed 32-\/bit integer class. \index{MAT\_\-C\_\-UINT32@{MAT\_\-C\_\-UINT32}!MAT@{MAT}}\index{MAT@{MAT}!MAT\_\-C\_\-UINT32@{MAT\_\-C\_\-UINT32}}\item[{\em 
\hypertarget{group__MAT_ggad4d60ae7b709fc81bfd744fb4c857c40a9a17a7edd45b19ef68197db81b27e816}{
MAT\_\-C\_\-UINT32}
\label{group__MAT_ggad4d60ae7b709fc81bfd744fb4c857c40a9a17a7edd45b19ef68197db81b27e816}
}]Matlab unsigned 32-\/bit integer class. \index{MAT\_\-C\_\-INT64@{MAT\_\-C\_\-INT64}!MAT@{MAT}}\index{MAT@{MAT}!MAT\_\-C\_\-INT64@{MAT\_\-C\_\-INT64}}\item[{\em 
\hypertarget{group__MAT_ggad4d60ae7b709fc81bfd744fb4c857c40a1ea83bcde49b35477494412973f82409}{
MAT\_\-C\_\-INT64}
\label{group__MAT_ggad4d60ae7b709fc81bfd744fb4c857c40a1ea83bcde49b35477494412973f82409}
}]Matlab unsigned 32-\/bit integer class. \index{MAT\_\-C\_\-UINT64@{MAT\_\-C\_\-UINT64}!MAT@{MAT}}\index{MAT@{MAT}!MAT\_\-C\_\-UINT64@{MAT\_\-C\_\-UINT64}}\item[{\em 
\hypertarget{group__MAT_ggad4d60ae7b709fc81bfd744fb4c857c40a86470e25c3763d9a24623f04326195dd}{
MAT\_\-C\_\-UINT64}
\label{group__MAT_ggad4d60ae7b709fc81bfd744fb4c857c40a86470e25c3763d9a24623f04326195dd}
}]Matlab unsigned 32-\/bit integer class. \index{MAT\_\-C\_\-FUNCTION@{MAT\_\-C\_\-FUNCTION}!MAT@{MAT}}\index{MAT@{MAT}!MAT\_\-C\_\-FUNCTION@{MAT\_\-C\_\-FUNCTION}}\item[{\em 
\hypertarget{group__MAT_ggad4d60ae7b709fc81bfd744fb4c857c40aaa9bf08312779cd1ab8e504a162ddcea}{
MAT\_\-C\_\-FUNCTION}
\label{group__MAT_ggad4d60ae7b709fc81bfd744fb4c857c40aaa9bf08312779cd1ab8e504a162ddcea}
}]Matlab unsigned 32-\/bit integer class. \end{description}
\end{Desc}

\hypertarget{group__MAT_ga768c318af97bd2567758ecb001ceb7f4}{
\index{MAT@{MAT}!matio\_\-compression@{matio\_\-compression}}
\index{matio\_\-compression@{matio\_\-compression}!MAT@{MAT}}
\subsubsection[{matio\_\-compression}]{\setlength{\rightskip}{0pt plus 5cm}enum {\bf matio\_\-compression}}}
\label{group__MAT_ga768c318af97bd2567758ecb001ceb7f4}


Matlab compression options. 

Matlab compression options \begin{Desc}
\item[Enumerator: ]\par
\begin{description}
\index{COMPRESSION\_\-NONE@{COMPRESSION\_\-NONE}!MAT@{MAT}}\index{MAT@{MAT}!COMPRESSION\_\-NONE@{COMPRESSION\_\-NONE}}\item[{\em 
\hypertarget{group__MAT_gga768c318af97bd2567758ecb001ceb7f4ac549b871996d1ef05d40056bf5bb52e5}{
COMPRESSION\_\-NONE}
\label{group__MAT_gga768c318af97bd2567758ecb001ceb7f4ac549b871996d1ef05d40056bf5bb52e5}
}]No compression. \index{COMPRESSION\_\-ZLIB@{COMPRESSION\_\-ZLIB}!MAT@{MAT}}\index{MAT@{MAT}!COMPRESSION\_\-ZLIB@{COMPRESSION\_\-ZLIB}}\item[{\em 
\hypertarget{group__MAT_gga768c318af97bd2567758ecb001ceb7f4a1f453c9a2c01b52294b37a1226837f86}{
COMPRESSION\_\-ZLIB}
\label{group__MAT_gga768c318af97bd2567758ecb001ceb7f4a1f453c9a2c01b52294b37a1226837f86}
}]zlib compression \end{description}
\end{Desc}

\hypertarget{group__MAT_gab9d6ef9e3ddca78a317b173f01d53fbb}{
\index{MAT@{MAT}!matio\_\-flags@{matio\_\-flags}}
\index{matio\_\-flags@{matio\_\-flags}!MAT@{MAT}}
\subsubsection[{matio\_\-flags}]{\setlength{\rightskip}{0pt plus 5cm}enum {\bf matio\_\-flags}}}
\label{group__MAT_gab9d6ef9e3ddca78a317b173f01d53fbb}


Matlab array flags. 

Matlab array flags \begin{Desc}
\item[Enumerator: ]\par
\begin{description}
\index{MAT\_\-F\_\-COMPLEX@{MAT\_\-F\_\-COMPLEX}!MAT@{MAT}}\index{MAT@{MAT}!MAT\_\-F\_\-COMPLEX@{MAT\_\-F\_\-COMPLEX}}\item[{\em 
\hypertarget{group__MAT_ggab9d6ef9e3ddca78a317b173f01d53fbbacd7b091a11184aad7fc6078c04470780}{
MAT\_\-F\_\-COMPLEX}
\label{group__MAT_ggab9d6ef9e3ddca78a317b173f01d53fbbacd7b091a11184aad7fc6078c04470780}
}]Complex bit flag. \index{MAT\_\-F\_\-GLOBAL@{MAT\_\-F\_\-GLOBAL}!MAT@{MAT}}\index{MAT@{MAT}!MAT\_\-F\_\-GLOBAL@{MAT\_\-F\_\-GLOBAL}}\item[{\em 
\hypertarget{group__MAT_ggab9d6ef9e3ddca78a317b173f01d53fbba49084e0c796aa7963e53f7539525d40d}{
MAT\_\-F\_\-GLOBAL}
\label{group__MAT_ggab9d6ef9e3ddca78a317b173f01d53fbba49084e0c796aa7963e53f7539525d40d}
}]Global bit flag. \index{MAT\_\-F\_\-LOGICAL@{MAT\_\-F\_\-LOGICAL}!MAT@{MAT}}\index{MAT@{MAT}!MAT\_\-F\_\-LOGICAL@{MAT\_\-F\_\-LOGICAL}}\item[{\em 
\hypertarget{group__MAT_ggab9d6ef9e3ddca78a317b173f01d53fbba57eb5c6e200bcbc0f1b7982f29a169c2}{
MAT\_\-F\_\-LOGICAL}
\label{group__MAT_ggab9d6ef9e3ddca78a317b173f01d53fbba57eb5c6e200bcbc0f1b7982f29a169c2}
}]Logical bit flag. \index{MAT\_\-F\_\-CLASS\_\-T@{MAT\_\-F\_\-CLASS\_\-T}!MAT@{MAT}}\index{MAT@{MAT}!MAT\_\-F\_\-CLASS\_\-T@{MAT\_\-F\_\-CLASS\_\-T}}\item[{\em 
\hypertarget{group__MAT_ggab9d6ef9e3ddca78a317b173f01d53fbba3a88beaec448e0485ffe21b18a540c1d}{
MAT\_\-F\_\-CLASS\_\-T}
\label{group__MAT_ggab9d6ef9e3ddca78a317b173f01d53fbba3a88beaec448e0485ffe21b18a540c1d}
}]Class-\/Type bits flag. \end{description}
\end{Desc}

\hypertarget{group__MAT_gacf7b3b879282b7ab3a51190e49bf3453}{
\index{MAT@{MAT}!matio\_\-types@{matio\_\-types}}
\index{matio\_\-types@{matio\_\-types}!MAT@{MAT}}
\subsubsection[{matio\_\-types}]{\setlength{\rightskip}{0pt plus 5cm}enum {\bf matio\_\-types}}}
\label{group__MAT_gacf7b3b879282b7ab3a51190e49bf3453}


Matlab data types. 

Matlab data types \begin{Desc}
\item[Enumerator: ]\par
\begin{description}
\index{MAT\_\-T\_\-UNKNOWN@{MAT\_\-T\_\-UNKNOWN}!MAT@{MAT}}\index{MAT@{MAT}!MAT\_\-T\_\-UNKNOWN@{MAT\_\-T\_\-UNKNOWN}}\item[{\em 
\hypertarget{group__MAT_ggacf7b3b879282b7ab3a51190e49bf3453a2a7318fe8bf9464935e7ed8902618293}{
MAT\_\-T\_\-UNKNOWN}
\label{group__MAT_ggacf7b3b879282b7ab3a51190e49bf3453a2a7318fe8bf9464935e7ed8902618293}
}]UNKOWN data type. \index{MAT\_\-T\_\-INT8@{MAT\_\-T\_\-INT8}!MAT@{MAT}}\index{MAT@{MAT}!MAT\_\-T\_\-INT8@{MAT\_\-T\_\-INT8}}\item[{\em 
\hypertarget{group__MAT_ggacf7b3b879282b7ab3a51190e49bf3453a9807f5033ed4f9b548953742d9fd1658}{
MAT\_\-T\_\-INT8}
\label{group__MAT_ggacf7b3b879282b7ab3a51190e49bf3453a9807f5033ed4f9b548953742d9fd1658}
}]8-\/bit signed integer data type \index{MAT\_\-T\_\-UINT8@{MAT\_\-T\_\-UINT8}!MAT@{MAT}}\index{MAT@{MAT}!MAT\_\-T\_\-UINT8@{MAT\_\-T\_\-UINT8}}\item[{\em 
\hypertarget{group__MAT_ggacf7b3b879282b7ab3a51190e49bf3453a01c1bd7db68f90552862eb5d311be408}{
MAT\_\-T\_\-UINT8}
\label{group__MAT_ggacf7b3b879282b7ab3a51190e49bf3453a01c1bd7db68f90552862eb5d311be408}
}]8-\/bit unsigned integer data type \index{MAT\_\-T\_\-INT16@{MAT\_\-T\_\-INT16}!MAT@{MAT}}\index{MAT@{MAT}!MAT\_\-T\_\-INT16@{MAT\_\-T\_\-INT16}}\item[{\em 
\hypertarget{group__MAT_ggacf7b3b879282b7ab3a51190e49bf3453a8c5b2e381946e95ea8d81ac216743302}{
MAT\_\-T\_\-INT16}
\label{group__MAT_ggacf7b3b879282b7ab3a51190e49bf3453a8c5b2e381946e95ea8d81ac216743302}
}]16-\/bit signed integer data type \index{MAT\_\-T\_\-UINT16@{MAT\_\-T\_\-UINT16}!MAT@{MAT}}\index{MAT@{MAT}!MAT\_\-T\_\-UINT16@{MAT\_\-T\_\-UINT16}}\item[{\em 
\hypertarget{group__MAT_ggacf7b3b879282b7ab3a51190e49bf3453a05bc7af7680aa68be95126ae0a4c2e31}{
MAT\_\-T\_\-UINT16}
\label{group__MAT_ggacf7b3b879282b7ab3a51190e49bf3453a05bc7af7680aa68be95126ae0a4c2e31}
}]16-\/bit unsigned integer data type \index{MAT\_\-T\_\-INT32@{MAT\_\-T\_\-INT32}!MAT@{MAT}}\index{MAT@{MAT}!MAT\_\-T\_\-INT32@{MAT\_\-T\_\-INT32}}\item[{\em 
\hypertarget{group__MAT_ggacf7b3b879282b7ab3a51190e49bf3453a83e06a68320726c6572bfbb9f3addb1d}{
MAT\_\-T\_\-INT32}
\label{group__MAT_ggacf7b3b879282b7ab3a51190e49bf3453a83e06a68320726c6572bfbb9f3addb1d}
}]32-\/bit signed integer data type \index{MAT\_\-T\_\-UINT32@{MAT\_\-T\_\-UINT32}!MAT@{MAT}}\index{MAT@{MAT}!MAT\_\-T\_\-UINT32@{MAT\_\-T\_\-UINT32}}\item[{\em 
\hypertarget{group__MAT_ggacf7b3b879282b7ab3a51190e49bf3453aa397e285a23fe240368b752897652c6a}{
MAT\_\-T\_\-UINT32}
\label{group__MAT_ggacf7b3b879282b7ab3a51190e49bf3453aa397e285a23fe240368b752897652c6a}
}]32-\/bit unsigned integer data type \index{MAT\_\-T\_\-SINGLE@{MAT\_\-T\_\-SINGLE}!MAT@{MAT}}\index{MAT@{MAT}!MAT\_\-T\_\-SINGLE@{MAT\_\-T\_\-SINGLE}}\item[{\em 
\hypertarget{group__MAT_ggacf7b3b879282b7ab3a51190e49bf3453a3a3657d40e9212c923d9b9d03531b64c}{
MAT\_\-T\_\-SINGLE}
\label{group__MAT_ggacf7b3b879282b7ab3a51190e49bf3453a3a3657d40e9212c923d9b9d03531b64c}
}]IEEE 754 single precision data type. \index{MAT\_\-T\_\-DOUBLE@{MAT\_\-T\_\-DOUBLE}!MAT@{MAT}}\index{MAT@{MAT}!MAT\_\-T\_\-DOUBLE@{MAT\_\-T\_\-DOUBLE}}\item[{\em 
\hypertarget{group__MAT_ggacf7b3b879282b7ab3a51190e49bf3453a31e721ecf7e188196f83c32838288797}{
MAT\_\-T\_\-DOUBLE}
\label{group__MAT_ggacf7b3b879282b7ab3a51190e49bf3453a31e721ecf7e188196f83c32838288797}
}]IEEE 754 double precision data type. \index{MAT\_\-T\_\-INT64@{MAT\_\-T\_\-INT64}!MAT@{MAT}}\index{MAT@{MAT}!MAT\_\-T\_\-INT64@{MAT\_\-T\_\-INT64}}\item[{\em 
\hypertarget{group__MAT_ggacf7b3b879282b7ab3a51190e49bf3453a9e825b5d18b8f946eaf2b4b57e51c145}{
MAT\_\-T\_\-INT64}
\label{group__MAT_ggacf7b3b879282b7ab3a51190e49bf3453a9e825b5d18b8f946eaf2b4b57e51c145}
}]64-\/bit signed integer data type \index{MAT\_\-T\_\-UINT64@{MAT\_\-T\_\-UINT64}!MAT@{MAT}}\index{MAT@{MAT}!MAT\_\-T\_\-UINT64@{MAT\_\-T\_\-UINT64}}\item[{\em 
\hypertarget{group__MAT_ggacf7b3b879282b7ab3a51190e49bf3453a45547932c46be27118abe08302d7e29f}{
MAT\_\-T\_\-UINT64}
\label{group__MAT_ggacf7b3b879282b7ab3a51190e49bf3453a45547932c46be27118abe08302d7e29f}
}]64-\/bit unsigned integer data type \index{MAT\_\-T\_\-MATRIX@{MAT\_\-T\_\-MATRIX}!MAT@{MAT}}\index{MAT@{MAT}!MAT\_\-T\_\-MATRIX@{MAT\_\-T\_\-MATRIX}}\item[{\em 
\hypertarget{group__MAT_ggacf7b3b879282b7ab3a51190e49bf3453a32985fee89a4df8db4b3f5d3a48823d3}{
MAT\_\-T\_\-MATRIX}
\label{group__MAT_ggacf7b3b879282b7ab3a51190e49bf3453a32985fee89a4df8db4b3f5d3a48823d3}
}]matrix data type \index{MAT\_\-T\_\-COMPRESSED@{MAT\_\-T\_\-COMPRESSED}!MAT@{MAT}}\index{MAT@{MAT}!MAT\_\-T\_\-COMPRESSED@{MAT\_\-T\_\-COMPRESSED}}\item[{\em 
\hypertarget{group__MAT_ggacf7b3b879282b7ab3a51190e49bf3453a30437f2eb3becc2fa6e5d96599d7f724}{
MAT\_\-T\_\-COMPRESSED}
\label{group__MAT_ggacf7b3b879282b7ab3a51190e49bf3453a30437f2eb3becc2fa6e5d96599d7f724}
}]compressed data type \index{MAT\_\-T\_\-UTF8@{MAT\_\-T\_\-UTF8}!MAT@{MAT}}\index{MAT@{MAT}!MAT\_\-T\_\-UTF8@{MAT\_\-T\_\-UTF8}}\item[{\em 
\hypertarget{group__MAT_ggacf7b3b879282b7ab3a51190e49bf3453ac34ad81f5cbd3b7d0d95e57e5be0149b}{
MAT\_\-T\_\-UTF8}
\label{group__MAT_ggacf7b3b879282b7ab3a51190e49bf3453ac34ad81f5cbd3b7d0d95e57e5be0149b}
}]8-\/bit unicode text data type \index{MAT\_\-T\_\-UTF16@{MAT\_\-T\_\-UTF16}!MAT@{MAT}}\index{MAT@{MAT}!MAT\_\-T\_\-UTF16@{MAT\_\-T\_\-UTF16}}\item[{\em 
\hypertarget{group__MAT_ggacf7b3b879282b7ab3a51190e49bf3453a87ffc0412143c326a1fcc759d5d81bdc}{
MAT\_\-T\_\-UTF16}
\label{group__MAT_ggacf7b3b879282b7ab3a51190e49bf3453a87ffc0412143c326a1fcc759d5d81bdc}
}]16-\/bit unicode text data type \index{MAT\_\-T\_\-UTF32@{MAT\_\-T\_\-UTF32}!MAT@{MAT}}\index{MAT@{MAT}!MAT\_\-T\_\-UTF32@{MAT\_\-T\_\-UTF32}}\item[{\em 
\hypertarget{group__MAT_ggacf7b3b879282b7ab3a51190e49bf3453a11e43c0e0be79b1983090e02ae583109}{
MAT\_\-T\_\-UTF32}
\label{group__MAT_ggacf7b3b879282b7ab3a51190e49bf3453a11e43c0e0be79b1983090e02ae583109}
}]32-\/bit unicode text data type \index{MAT\_\-T\_\-STRING@{MAT\_\-T\_\-STRING}!MAT@{MAT}}\index{MAT@{MAT}!MAT\_\-T\_\-STRING@{MAT\_\-T\_\-STRING}}\item[{\em 
\hypertarget{group__MAT_ggacf7b3b879282b7ab3a51190e49bf3453a9456a83c0b22022af42461a09d63cdb2}{
MAT\_\-T\_\-STRING}
\label{group__MAT_ggacf7b3b879282b7ab3a51190e49bf3453a9456a83c0b22022af42461a09d63cdb2}
}]String data type. \index{MAT\_\-T\_\-CELL@{MAT\_\-T\_\-CELL}!MAT@{MAT}}\index{MAT@{MAT}!MAT\_\-T\_\-CELL@{MAT\_\-T\_\-CELL}}\item[{\em 
\hypertarget{group__MAT_ggacf7b3b879282b7ab3a51190e49bf3453a07599cf2cca6d2b2d059378563318ba5}{
MAT\_\-T\_\-CELL}
\label{group__MAT_ggacf7b3b879282b7ab3a51190e49bf3453a07599cf2cca6d2b2d059378563318ba5}
}]Cell array data type. \index{MAT\_\-T\_\-STRUCT@{MAT\_\-T\_\-STRUCT}!MAT@{MAT}}\index{MAT@{MAT}!MAT\_\-T\_\-STRUCT@{MAT\_\-T\_\-STRUCT}}\item[{\em 
\hypertarget{group__MAT_ggacf7b3b879282b7ab3a51190e49bf3453a4f4d5a6e1d42c6aa81ffb810e5da5c85}{
MAT\_\-T\_\-STRUCT}
\label{group__MAT_ggacf7b3b879282b7ab3a51190e49bf3453a4f4d5a6e1d42c6aa81ffb810e5da5c85}
}]Structure data type. \index{MAT\_\-T\_\-ARRAY@{MAT\_\-T\_\-ARRAY}!MAT@{MAT}}\index{MAT@{MAT}!MAT\_\-T\_\-ARRAY@{MAT\_\-T\_\-ARRAY}}\item[{\em 
\hypertarget{group__MAT_ggacf7b3b879282b7ab3a51190e49bf3453acf106b0c23021582375f59bc9fce89b1}{
MAT\_\-T\_\-ARRAY}
\label{group__MAT_ggacf7b3b879282b7ab3a51190e49bf3453acf106b0c23021582375f59bc9fce89b1}
}]Array data type. \index{MAT\_\-T\_\-FUNCTION@{MAT\_\-T\_\-FUNCTION}!MAT@{MAT}}\index{MAT@{MAT}!MAT\_\-T\_\-FUNCTION@{MAT\_\-T\_\-FUNCTION}}\item[{\em 
\hypertarget{group__MAT_ggacf7b3b879282b7ab3a51190e49bf3453ae76686f267dd1641cd55dce306af6d10}{
MAT\_\-T\_\-FUNCTION}
\label{group__MAT_ggacf7b3b879282b7ab3a51190e49bf3453ae76686f267dd1641cd55dce306af6d10}
}]Function data type. \end{description}
\end{Desc}



\subsection{Function Documentation}
\hypertarget{group__MAT_ga9b8d09f631538b14ca29792e0334e349}{
\index{MAT@{MAT}!Mat\_\-CalcSingleSubscript@{Mat\_\-CalcSingleSubscript}}
\index{Mat\_\-CalcSingleSubscript@{Mat\_\-CalcSingleSubscript}!MAT@{MAT}}
\subsubsection[{Mat\_\-CalcSingleSubscript}]{\setlength{\rightskip}{0pt plus 5cm}int Mat\_\-CalcSingleSubscript (
\begin{DoxyParamCaption}
\item[{int}]{rank, }
\item[{int $\ast$}]{dims, }
\item[{int $\ast$}]{subs}
\end{DoxyParamCaption}
)}}
\label{group__MAT_ga9b8d09f631538b14ca29792e0334e349}


Calculate a single subscript from a set of subscript values. 

Calculates a single linear subscript (0-\/relative) given a 1-\/relative subscript for each dimension. The calculation uses the formula below where index is the linear index, s is an array of length RANK where each element is the subscript for the correspondind dimension, D is an array whose elements are the dimensions of the variable. \[ index = \sum\limits_{k=0}^{RANK-1} [(s_k - 1) \prod\limits_{l=0}^{k} D_l ] \]


\begin{DoxyParams}{Parameters}
{\em rank} & Rank of the variable \\
\hline
{\em dims} & dimensions of the variable \\
\hline
{\em subs} & Dimension subscripts \\
\hline
\end{DoxyParams}
\begin{DoxyReturn}{Returns}
Single (linear) subscript 
\end{DoxyReturn}
\hypertarget{group__MAT_gabe2571a4b9b6cff3b31aa6f152deba61}{
\index{MAT@{MAT}!Mat\_\-CalcSubscripts@{Mat\_\-CalcSubscripts}}
\index{Mat\_\-CalcSubscripts@{Mat\_\-CalcSubscripts}!MAT@{MAT}}
\subsubsection[{Mat\_\-CalcSubscripts}]{\setlength{\rightskip}{0pt plus 5cm}int$\ast$ Mat\_\-CalcSubscripts (
\begin{DoxyParamCaption}
\item[{int}]{rank, }
\item[{int $\ast$}]{dims, }
\item[{int}]{index}
\end{DoxyParamCaption}
)}}
\label{group__MAT_gabe2571a4b9b6cff3b31aa6f152deba61}


Calculate a set of subscript values from a single(linear) subscript. 

Calculates 1-\/relative subscripts for each dimension given a 0-\/relative linear index. Subscripts are calculated as follows where s is the array of dimension subscripts, D is the array of dimensions, and index is the linear index. \[ s_k = \lfloor\frac{1}{L} \prod\limits_{l = 0}^{k} D_l\rfloor + 1 \] \[ L = index - \sum\limits_{l = k}^{RANK - 1} s_k \prod\limits_{m = 0}^{k} D_m \]


\begin{DoxyParams}{Parameters}
{\em rank} & Rank of the variable \\
\hline
{\em dims} & dimensions of the variable \\
\hline
{\em index} & linear index \\
\hline
\end{DoxyParams}
\begin{DoxyReturn}{Returns}
Array of dimension subscripts 
\end{DoxyReturn}
\hypertarget{group__MAT_ga101c92ff7bde4a2d4615661beba09262}{
\index{MAT@{MAT}!Mat\_\-Close@{Mat\_\-Close}}
\index{Mat\_\-Close@{Mat\_\-Close}!MAT@{MAT}}
\subsubsection[{Mat\_\-Close}]{\setlength{\rightskip}{0pt plus 5cm}int Mat\_\-Close (
\begin{DoxyParamCaption}
\item[{{\bf mat\_\-t} $\ast$}]{mat}
\end{DoxyParamCaption}
)}}
\label{group__MAT_ga101c92ff7bde4a2d4615661beba09262}


Closes an open Matlab MAT file. 

Closes the given Matlab MAT file and frees any memory with it.


\begin{DoxyParams}{Parameters}
{\em mat} & Pointer to the MAT file \\
\hline
\end{DoxyParams}

\begin{DoxyRetVals}{Return values}
{\em 0} & \\
\hline
\end{DoxyRetVals}
\hypertarget{group__MAT_ga22d404f203af7869c841400e7ad247cf}{
\index{MAT@{MAT}!Mat\_\-CreateVer@{Mat\_\-CreateVer}}
\index{Mat\_\-CreateVer@{Mat\_\-CreateVer}!MAT@{MAT}}
\subsubsection[{Mat\_\-CreateVer}]{\setlength{\rightskip}{0pt plus 5cm}{\bf mat\_\-t}$\ast$ Mat\_\-CreateVer (
\begin{DoxyParamCaption}
\item[{const char $\ast$}]{matname, }
\item[{const char $\ast$}]{hdr\_\-str, }
\item[{enum {\bf mat\_\-ft}}]{mat\_\-file\_\-ver}
\end{DoxyParamCaption}
)}}
\label{group__MAT_ga22d404f203af7869c841400e7ad247cf}


Creates a new Matlab MAT file. 

Tries to create a new Matlab MAT file with the given name and optional header string. If no header string is given, the default string is used containing the software, version, and date in it. If a header string is given, at most the first 116 characters is written to the file. The given header string need not be the full 116 characters, but MUST be NULL terminated.


\begin{DoxyParams}{Parameters}
{\em matname} & Name of MAT file to create \\
\hline
{\em hdr\_\-str} & Optional header string, NULL to use default \\
\hline
{\em mat\_\-file\_\-ver} & MAT file version to create \\
\hline
\end{DoxyParams}
\begin{DoxyReturn}{Returns}
A pointer to the MAT file or NULL if it failed. This is not a simple FILE $\ast$ and should not be used as one. 
\end{DoxyReturn}
\hypertarget{group__MAT_gafbfedb5636a99f0ef867520c47f77d18}{
\index{MAT@{MAT}!Mat\_\-Open@{Mat\_\-Open}}
\index{Mat\_\-Open@{Mat\_\-Open}!MAT@{MAT}}
\subsubsection[{Mat\_\-Open}]{\setlength{\rightskip}{0pt plus 5cm}{\bf mat\_\-t}$\ast$ Mat\_\-Open (
\begin{DoxyParamCaption}
\item[{const char $\ast$}]{matname, }
\item[{int}]{mode}
\end{DoxyParamCaption}
)}}
\label{group__MAT_gafbfedb5636a99f0ef867520c47f77d18}


Opens an existing Matlab MAT file. 

Tries to open a Matlab MAT file with the given name


\begin{DoxyParams}{Parameters}
{\em matname} & Name of MAT file to open \\
\hline
{\em mode} & File access mode (MAT\_\-ACC\_\-RDONLY,MAT\_\-ACC\_\-RDWR,etc). \\
\hline
\end{DoxyParams}
\begin{DoxyReturn}{Returns}
A pointer to the MAT file or NULL if it failed. This is not a simple FILE $\ast$ and should not be used as one. 
\end{DoxyReturn}
\hypertarget{group__MAT_ga4d6e3892d2e216c507a744ba0e070d0b}{
\index{MAT@{MAT}!Mat\_\-Rewind@{Mat\_\-Rewind}}
\index{Mat\_\-Rewind@{Mat\_\-Rewind}!MAT@{MAT}}
\subsubsection[{Mat\_\-Rewind}]{\setlength{\rightskip}{0pt plus 5cm}int Mat\_\-Rewind (
\begin{DoxyParamCaption}
\item[{{\bf mat\_\-t} $\ast$}]{mat}
\end{DoxyParamCaption}
)}}
\label{group__MAT_ga4d6e3892d2e216c507a744ba0e070d0b}


Rewinds a Matlab MAT file to the first variable. 

Rewinds a Matlab MAT file to the first variable


\begin{DoxyParams}{Parameters}
{\em mat} & Pointer to the MAT file \\
\hline
\end{DoxyParams}

\begin{DoxyRetVals}{Return values}
{\em 0} & on success \\
\hline
\end{DoxyRetVals}
\hypertarget{group__MAT_ga2bf682f015b22fa796a8885e997661e7}{
\index{MAT@{MAT}!Mat\_\-SizeOfClass@{Mat\_\-SizeOfClass}}
\index{Mat\_\-SizeOfClass@{Mat\_\-SizeOfClass}!MAT@{MAT}}
\subsubsection[{Mat\_\-SizeOfClass}]{\setlength{\rightskip}{0pt plus 5cm}size\_\-t Mat\_\-SizeOfClass (
\begin{DoxyParamCaption}
\item[{int}]{class\_\-type}
\end{DoxyParamCaption}
)}}
\label{group__MAT_ga2bf682f015b22fa796a8885e997661e7}


Returns the size of a Matlab Class. 

Returns the size (in bytes) of the matlab class class\_\-type


\begin{DoxyParams}{Parameters}
{\em class\_\-type} & Matlab class type (MAT\_\-C\_\-$\ast$) \\
\hline
\end{DoxyParams}
\begin{DoxyReturn}{Returns}
Size of the class 
\end{DoxyReturn}
\hypertarget{group__MAT_ga9f8ab8a7e4206c16cb20437acc6960d2}{
\index{MAT@{MAT}!Mat\_\-VarAddStructField@{Mat\_\-VarAddStructField}}
\index{Mat\_\-VarAddStructField@{Mat\_\-VarAddStructField}!MAT@{MAT}}
\subsubsection[{Mat\_\-VarAddStructField}]{\setlength{\rightskip}{0pt plus 5cm}int Mat\_\-VarAddStructField (
\begin{DoxyParamCaption}
\item[{{\bf matvar\_\-t} $\ast$}]{matvar, }
\item[{{\bf matvar\_\-t} $\ast$$\ast$}]{fields}
\end{DoxyParamCaption}
)}}
\label{group__MAT_ga9f8ab8a7e4206c16cb20437acc6960d2}


Adds a field to a structure. 

Adds the given field to the structure. fields should be an array of \hyperlink{structmatvar__t}{matvar\_\-t} pointers of the same size as the structure (i.e. 1 field per structure element).


\begin{DoxyParams}{Parameters}
{\em matvar} & Pointer to the Structure MAT variable \\
\hline
{\em fields} & Array of fields to be added \\
\hline
\end{DoxyParams}

\begin{DoxyRetVals}{Return values}
{\em 0} & on success \\
\hline
\end{DoxyRetVals}
\hypertarget{group__MAT_gae7c9c3699f6e9c31a9c490300013098c}{
\index{MAT@{MAT}!Mat\_\-VarCalloc@{Mat\_\-VarCalloc}}
\index{Mat\_\-VarCalloc@{Mat\_\-VarCalloc}!MAT@{MAT}}
\subsubsection[{Mat\_\-VarCalloc}]{\setlength{\rightskip}{0pt plus 5cm}{\bf matvar\_\-t}$\ast$ Mat\_\-VarCalloc (
\begin{DoxyParamCaption}
\item[{void}]{}
\end{DoxyParamCaption}
)}}
\label{group__MAT_gae7c9c3699f6e9c31a9c490300013098c}


Allocates memory for a new \hyperlink{structmatvar__t}{matvar\_\-t} and initializes all the fields. 

\begin{DoxyReturn}{Returns}
A newly allocated \hyperlink{structmatvar__t}{matvar\_\-t} 
\end{DoxyReturn}
\hypertarget{group__MAT_ga1c54a84bb4d810c6fccdb8869489eac4}{
\index{MAT@{MAT}!Mat\_\-VarCreate@{Mat\_\-VarCreate}}
\index{Mat\_\-VarCreate@{Mat\_\-VarCreate}!MAT@{MAT}}
\subsubsection[{Mat\_\-VarCreate}]{\setlength{\rightskip}{0pt plus 5cm}{\bf matvar\_\-t}$\ast$ Mat\_\-VarCreate (
\begin{DoxyParamCaption}
\item[{const char $\ast$}]{name, }
\item[{enum {\bf matio\_\-classes}}]{class\_\-type, }
\item[{enum {\bf matio\_\-types}}]{data\_\-type, }
\item[{int}]{rank, }
\item[{size\_\-t $\ast$}]{dims, }
\item[{void $\ast$}]{data, }
\item[{int}]{opt}
\end{DoxyParamCaption}
)}}
\label{group__MAT_ga1c54a84bb4d810c6fccdb8869489eac4}


Creates a MAT Variable with the given name and (optionally) data. 

Creates a MAT variable that can be written to a Matlab MAT file with the given name, data type, dimensions and data. Rank should always be 2 or more. i.e. Scalar values would have rank=2 and dims\mbox{[}2\mbox{]} = \{1,1\}. Data type is one of the MAT\_\-T types. MAT adds MAT\_\-T\_\-STRUCT and MAT\_\-T\_\-CELL to create Structures and Cell Arrays respectively. For MAT\_\-T\_\-STRUCT, data should be a NULL terminated array of \hyperlink{structmatvar__t}{matvar\_\-t} $\ast$ variables (i.e. for a 3x2 structure with 10 fields, there should be 61 \hyperlink{structmatvar__t}{matvar\_\-t} $\ast$ variables where the last one is NULL). For cell arrays, the NULL termination isn't necessary. So to create a cell array of size 3x2, data would be the address of an array of 6 \hyperlink{structmatvar__t}{matvar\_\-t} $\ast$ variables.

EXAMPLE: To create a struct of size 3x2 with 3 fields: 
\begin{DoxyCode}
     int rank=2, dims[2] = {3,2}, nfields = 3;
     matvar_t **vars;

     vars = malloc((3*2*nfields+1)*sizeof(matvar_t *));
     vars[0]             = Mat_VarCreate(...);
        :
     vars[3*2*nfields-1] = Mat_VarCreate(...);
     vars[3*2*nfields]   = NULL;
\end{DoxyCode}


EXAMPLE: To create a cell array of size 3x2: 
\begin{DoxyCode}
     int rank=2, dims[2] = {3,2};
     matvar_t **vars;

     vars = malloc(3*2*sizeof(matvar_t *));
     vars[0]             = Mat_VarCreate(...);
        :
     vars[5] = Mat_VarCreate(...);
\end{DoxyCode}



\begin{DoxyParams}{Parameters}
{\em name} & Name of the variable to create \\
\hline
{\em class\_\-type} & class type of the variable in Matlab(one of the mx Classes) \\
\hline
{\em data\_\-type} & data type of the variable (one of the MAT\_\-T\_\- Types) \\
\hline
{\em rank} & Rank of the variable \\
\hline
{\em dims} & array of dimensions of the variable of size rank \\
\hline
{\em data} & pointer to the data \\
\hline
{\em opt} & 0, or bitwise or of the following options:
\begin{DoxyItemize}
\item MEM\_\-CONSERVE to just use the pointer to the data and not copy the data itself. Note that the pointer should not be freed until you are done with the mat variable. The Mat\_\-VarFree function will NOT free data that was created with MEM\_\-CONSERVE, so free it yourself.
\item MAT\_\-F\_\-COMPLEX to specify that the data is complex. The data variable should be a pointer to a struct \hyperlink{structComplexSplit}{ComplexSplit} type.
\item MAT\_\-F\_\-GLOBAL to assign the variable as a global variable
\item MAT\_\-F\_\-LOGICAL to specify that it is a logical variable 
\end{DoxyItemize}\\
\hline
\end{DoxyParams}
\begin{DoxyReturn}{Returns}
A MAT variable that can be written to a file or otherwise used 
\end{DoxyReturn}
\hypertarget{group__MAT_gabf139e48d48177e5069338fa2919c60a}{
\index{MAT@{MAT}!Mat\_\-VarDelete@{Mat\_\-VarDelete}}
\index{Mat\_\-VarDelete@{Mat\_\-VarDelete}!MAT@{MAT}}
\subsubsection[{Mat\_\-VarDelete}]{\setlength{\rightskip}{0pt plus 5cm}int Mat\_\-VarDelete (
\begin{DoxyParamCaption}
\item[{{\bf mat\_\-t} $\ast$}]{mat, }
\item[{const char $\ast$}]{name}
\end{DoxyParamCaption}
)}}
\label{group__MAT_gabf139e48d48177e5069338fa2919c60a}


Deletes a variable from a file. 


\begin{DoxyParams}{Parameters}
{\em mat} & Pointer to the mat\_\-t file structure \\
\hline
{\em name} & Name of the variable to delete \\
\hline
\end{DoxyParams}
\begin{DoxyReturn}{Returns}
0 on success 
\end{DoxyReturn}
\hypertarget{group__MAT_ga7ef80c5d99d7918b2b09db3bea106ecc}{
\index{MAT@{MAT}!Mat\_\-VarDuplicate@{Mat\_\-VarDuplicate}}
\index{Mat\_\-VarDuplicate@{Mat\_\-VarDuplicate}!MAT@{MAT}}
\subsubsection[{Mat\_\-VarDuplicate}]{\setlength{\rightskip}{0pt plus 5cm}{\bf matvar\_\-t}$\ast$ Mat\_\-VarDuplicate (
\begin{DoxyParamCaption}
\item[{const {\bf matvar\_\-t} $\ast$}]{in, }
\item[{int}]{opt}
\end{DoxyParamCaption}
)}}
\label{group__MAT_ga7ef80c5d99d7918b2b09db3bea106ecc}


Duplicates a \hyperlink{structmatvar__t}{matvar\_\-t} structure. 

Provides a clean function for duplicating a \hyperlink{structmatvar__t}{matvar\_\-t} structure.


\begin{DoxyParams}{Parameters}
{\em in} & pointer to the \hyperlink{structmatvar__t}{matvar\_\-t} structure to be duplicated \\
\hline
{\em opt} & 0 does a shallow duplicate and only assigns the data pointer to the duplicated array. 1 will do a deep duplicate and actually duplicate the contents of the data. Warning: If you do a shallow copy and free both structures, the data will be freed twice and memory will be corrupted. This may be fixed in a later release. \\
\hline
\end{DoxyParams}
\begin{DoxyReturn}{Returns}
Pointer to the duplicated \hyperlink{structmatvar__t}{matvar\_\-t} structure. 
\end{DoxyReturn}
\hypertarget{group__MAT_ga1d14716f7450530fd1c9d02413787f0e}{
\index{MAT@{MAT}!Mat\_\-VarFree@{Mat\_\-VarFree}}
\index{Mat\_\-VarFree@{Mat\_\-VarFree}!MAT@{MAT}}
\subsubsection[{Mat\_\-VarFree}]{\setlength{\rightskip}{0pt plus 5cm}void Mat\_\-VarFree (
\begin{DoxyParamCaption}
\item[{{\bf matvar\_\-t} $\ast$}]{matvar}
\end{DoxyParamCaption}
)}}
\label{group__MAT_ga1d14716f7450530fd1c9d02413787f0e}


Frees all the allocated memory associated with the structure. 

Frees memory used by a MAT variable. Frees the data associated with a MAT variable if it's non-\/NULL and MEM\_\-CONSERVE was not used.


\begin{DoxyParams}{Parameters}
{\em matvar} & Pointer to the \hyperlink{structmatvar__t}{matvar\_\-t} structure \\
\hline
\end{DoxyParams}
\hypertarget{group__MAT_gac1e15063439c0bd3eb0c986514c742dc}{
\index{MAT@{MAT}!Mat\_\-VarGetCell@{Mat\_\-VarGetCell}}
\index{Mat\_\-VarGetCell@{Mat\_\-VarGetCell}!MAT@{MAT}}
\subsubsection[{Mat\_\-VarGetCell}]{\setlength{\rightskip}{0pt plus 5cm}{\bf matvar\_\-t}$\ast$ Mat\_\-VarGetCell (
\begin{DoxyParamCaption}
\item[{{\bf matvar\_\-t} $\ast$}]{matvar, }
\item[{int}]{index}
\end{DoxyParamCaption}
)}}
\label{group__MAT_gac1e15063439c0bd3eb0c986514c742dc}


Returns a pointer to the Cell array at a specific index. 

Returns a pointer to the Cell Array Field at the given 1-\/relative index. MAT file must be a version 5 matlab file.


\begin{DoxyParams}{Parameters}
{\em matvar} & Pointer to the Cell Array MAT variable \\
\hline
{\em index} & linear index of cell to return \\
\hline
\end{DoxyParams}
\begin{DoxyReturn}{Returns}
Pointer to the Cell Array Field on success, NULL on error 
\end{DoxyReturn}
\hypertarget{group__MAT_ga0732b0a6c40975b036068b9a14422d45}{
\index{MAT@{MAT}!Mat\_\-VarGetCells@{Mat\_\-VarGetCells}}
\index{Mat\_\-VarGetCells@{Mat\_\-VarGetCells}!MAT@{MAT}}
\subsubsection[{Mat\_\-VarGetCells}]{\setlength{\rightskip}{0pt plus 5cm}{\bf matvar\_\-t}$\ast$$\ast$ Mat\_\-VarGetCells (
\begin{DoxyParamCaption}
\item[{{\bf matvar\_\-t} $\ast$}]{matvar, }
\item[{int $\ast$}]{start, }
\item[{int $\ast$}]{stride, }
\item[{int $\ast$}]{edge}
\end{DoxyParamCaption}
)}}
\label{group__MAT_ga0732b0a6c40975b036068b9a14422d45}


Indexes a cell array. 

Finds cells of a cell array given a start, stride, and edge for each. dimension. The cells are placed in a pointer array. The cells should not be freed, but the array of pointers should be. If copies are needed, use Mat\_\-VarDuplicate on each cell. MAT File version must be 5.


\begin{DoxyParams}{Parameters}
{\em matvar} & Cell Array matlab variable \\
\hline
{\em start} & vector of length rank with 0-\/relative starting coordinates for each diemnsion. \\
\hline
{\em stride} & vector of length rank with strides for each diemnsion. \\
\hline
{\em edge} & vector of length rank with the number of elements to read in each diemnsion. \\
\hline
\end{DoxyParams}
\begin{DoxyReturn}{Returns}
an array of pointers to the cells 
\end{DoxyReturn}
\hypertarget{group__MAT_ga004987d665654409f74eaf8e82bb1380}{
\index{MAT@{MAT}!Mat\_\-VarGetCellsLinear@{Mat\_\-VarGetCellsLinear}}
\index{Mat\_\-VarGetCellsLinear@{Mat\_\-VarGetCellsLinear}!MAT@{MAT}}
\subsubsection[{Mat\_\-VarGetCellsLinear}]{\setlength{\rightskip}{0pt plus 5cm}{\bf matvar\_\-t}$\ast$$\ast$ Mat\_\-VarGetCellsLinear (
\begin{DoxyParamCaption}
\item[{{\bf matvar\_\-t} $\ast$}]{matvar, }
\item[{int}]{start, }
\item[{int}]{stride, }
\item[{int}]{edge}
\end{DoxyParamCaption}
)}}
\label{group__MAT_ga004987d665654409f74eaf8e82bb1380}


Indexes a cell array. 

Finds cells of a cell array given a linear indexed start, stride, and edge. The cells are placed in a pointer array. The cells themself should not be freed as they are part of the original cell array, but the pointer array should be. If copies are needed, use Mat\_\-VarDuplicate on each of the cells. MAT file version must be 5.


\begin{DoxyParams}{Parameters}
{\em matvar} & Cell Array matlab variable \\
\hline
{\em start} & starting index \\
\hline
{\em stride} & stride \\
\hline
{\em edge} & Number of cells to get \\
\hline
\end{DoxyParams}
\begin{DoxyReturn}{Returns}
an array of pointers to the cells 
\end{DoxyReturn}
\hypertarget{group__MAT_ga56b9a545990a0f253164018e37111741}{
\index{MAT@{MAT}!Mat\_\-VarGetNumberOfFields@{Mat\_\-VarGetNumberOfFields}}
\index{Mat\_\-VarGetNumberOfFields@{Mat\_\-VarGetNumberOfFields}!MAT@{MAT}}
\subsubsection[{Mat\_\-VarGetNumberOfFields}]{\setlength{\rightskip}{0pt plus 5cm}int Mat\_\-VarGetNumberOfFields (
\begin{DoxyParamCaption}
\item[{{\bf matvar\_\-t} $\ast$}]{matvar}
\end{DoxyParamCaption}
)}}
\label{group__MAT_ga56b9a545990a0f253164018e37111741}


Returns the number of fields in a structure variable. 

Returns the number of fields in the given structure. MAT file version must be 5.


\begin{DoxyParams}{Parameters}
{\em matvar} & Structure matlab variable \\
\hline
\end{DoxyParams}
\begin{DoxyReturn}{Returns}
Number of fields, or a negative number on error 
\end{DoxyReturn}
\hypertarget{group__MAT_gaeeb798fead2f765bddfb19016c7fdbcc}{
\index{MAT@{MAT}!Mat\_\-VarGetSize@{Mat\_\-VarGetSize}}
\index{Mat\_\-VarGetSize@{Mat\_\-VarGetSize}!MAT@{MAT}}
\subsubsection[{Mat\_\-VarGetSize}]{\setlength{\rightskip}{0pt plus 5cm}size\_\-t Mat\_\-VarGetSize (
\begin{DoxyParamCaption}
\item[{{\bf matvar\_\-t} $\ast$}]{matvar}
\end{DoxyParamCaption}
)}}
\label{group__MAT_gaeeb798fead2f765bddfb19016c7fdbcc}


Calculates the size of a matlab variable in bytes. 


\begin{DoxyParams}{Parameters}
{\em matvar} & matlab variable \\
\hline
\end{DoxyParams}
\begin{DoxyReturn}{Returns}
size of the variable in bytes 
\end{DoxyReturn}
\hypertarget{group__MAT_ga7018bfe6934b96ae32e8f2a1712eefab}{
\index{MAT@{MAT}!Mat\_\-VarGetStructField@{Mat\_\-VarGetStructField}}
\index{Mat\_\-VarGetStructField@{Mat\_\-VarGetStructField}!MAT@{MAT}}
\subsubsection[{Mat\_\-VarGetStructField}]{\setlength{\rightskip}{0pt plus 5cm}{\bf matvar\_\-t}$\ast$ Mat\_\-VarGetStructField (
\begin{DoxyParamCaption}
\item[{{\bf matvar\_\-t} $\ast$}]{matvar, }
\item[{void $\ast$}]{name\_\-or\_\-index, }
\item[{int}]{opt, }
\item[{int}]{index}
\end{DoxyParamCaption}
)}}
\label{group__MAT_ga7018bfe6934b96ae32e8f2a1712eefab}


Finds a field of a structure. 

Returns a pointer to the structure field at the given 0-\/relative index. MAT file version must be 5.


\begin{DoxyParams}{Parameters}
{\em matvar} & Pointer to the Structure MAT variable \\
\hline
{\em name\_\-or\_\-index} & Name of the field, or the 1-\/relative index of the field. If the index is used, it should be the address of an integer variable whose value is the index number. \\
\hline
{\em opt} & BY\_\-NAME if the name\_\-or\_\-index is the name or BY\_\-INDEX if the index was passed. \\
\hline
{\em index} & linear index of the structure to find the field of \\
\hline
\end{DoxyParams}
\begin{DoxyReturn}{Returns}
Pointer to the Structure Field on success, NULL on error 
\end{DoxyReturn}
\hypertarget{group__MAT_ga509178d7dc15faf9f7cd0440df6009b9}{
\index{MAT@{MAT}!Mat\_\-VarGetStructs@{Mat\_\-VarGetStructs}}
\index{Mat\_\-VarGetStructs@{Mat\_\-VarGetStructs}!MAT@{MAT}}
\subsubsection[{Mat\_\-VarGetStructs}]{\setlength{\rightskip}{0pt plus 5cm}{\bf matvar\_\-t}$\ast$ Mat\_\-VarGetStructs (
\begin{DoxyParamCaption}
\item[{{\bf matvar\_\-t} $\ast$}]{matvar, }
\item[{int $\ast$}]{start, }
\item[{int $\ast$}]{stride, }
\item[{int $\ast$}]{edge, }
\item[{int}]{copy\_\-fields}
\end{DoxyParamCaption}
)}}
\label{group__MAT_ga509178d7dc15faf9f7cd0440df6009b9}


Indexes a structure. 

Finds structures of a structure array given a start, stride, and edge for each dimension. The structures are placed in a new structure array. If copy\_\-fields is non-\/zero, the indexed structures are copied and should be freed, but if copy\_\-fields is zero, the indexed structures are pointers to the original, but should still be freed since the mem\_\-conserve flag is set so that the structures are not freed. MAT File version must be 5.


\begin{DoxyParams}{Parameters}
{\em matvar} & Structure matlab variable \\
\hline
{\em start} & vector of length rank with 0-\/relative starting coordinates for each diemnsion. \\
\hline
{\em stride} & vector of length rank with strides for each diemnsion. \\
\hline
{\em edge} & vector of length rank with the number of elements to read in each diemnsion. \\
\hline
{\em copy\_\-fields} & 1 to copy the fields, 0 to just set pointers to them. If 0 is used, the fields should not be freed themselves. \\
\hline
\end{DoxyParams}
\begin{DoxyReturn}{Returns}
A new structure with fields indexed from matvar. 
\end{DoxyReturn}
\hypertarget{group__MAT_gaa56680fb7b2cd3d410f659e945da8141}{
\index{MAT@{MAT}!Mat\_\-VarGetStructsLinear@{Mat\_\-VarGetStructsLinear}}
\index{Mat\_\-VarGetStructsLinear@{Mat\_\-VarGetStructsLinear}!MAT@{MAT}}
\subsubsection[{Mat\_\-VarGetStructsLinear}]{\setlength{\rightskip}{0pt plus 5cm}{\bf matvar\_\-t}$\ast$ Mat\_\-VarGetStructsLinear (
\begin{DoxyParamCaption}
\item[{{\bf matvar\_\-t} $\ast$}]{matvar, }
\item[{int}]{start, }
\item[{int}]{stride, }
\item[{int}]{edge, }
\item[{int}]{copy\_\-fields}
\end{DoxyParamCaption}
)}}
\label{group__MAT_gaa56680fb7b2cd3d410f659e945da8141}


Indexes a structure. 

Finds structures of a structure array given a single (linear)start, stride, and edge. The structures are placed in a new structure array. If copy\_\-fields is non-\/zero, the indexed structures are copied and should be freed, but if copy\_\-fields is zero, the indexed structures are pointers to the original, but should still be freed since the mem\_\-conserve flag is set so that the structures are not freed. MAT File version must be 5.


\begin{DoxyParams}{Parameters}
{\em matvar} & Structure matlab variable \\
\hline
{\em start} & starting index \\
\hline
{\em stride} & stride \\
\hline
{\em edge} & Number of structures to get \\
\hline
{\em copy\_\-fields} & 1 to copy the fields, 0 to just set pointers to them. If 0 is used, the fields should not be freed themselves. \\
\hline
\end{DoxyParams}
\begin{DoxyReturn}{Returns}
A new structure with fields indexed from matvar 
\end{DoxyReturn}
\hypertarget{group__MAT_ga9100c145e338b84b55d5d0795d5d390a}{
\index{MAT@{MAT}!Mat\_\-VarPrint@{Mat\_\-VarPrint}}
\index{Mat\_\-VarPrint@{Mat\_\-VarPrint}!MAT@{MAT}}
\subsubsection[{Mat\_\-VarPrint}]{\setlength{\rightskip}{0pt plus 5cm}void Mat\_\-VarPrint (
\begin{DoxyParamCaption}
\item[{{\bf matvar\_\-t} $\ast$}]{matvar, }
\item[{int}]{printdata}
\end{DoxyParamCaption}
)}}
\label{group__MAT_ga9100c145e338b84b55d5d0795d5d390a}


Prints the variable information. 

Prints to stdout the values of the \hyperlink{structmatvar__t}{matvar\_\-t} structure


\begin{DoxyParams}{Parameters}
{\em matvar} & Pointer to the \hyperlink{structmatvar__t}{matvar\_\-t} structure \\
\hline
{\em printdata} & set to 1 if the Variables data should be printed, else 0 \\
\hline
\end{DoxyParams}
\hypertarget{group__MAT_ga3505f63029763eaa73d5a19f1115eb42}{
\index{MAT@{MAT}!Mat\_\-VarRead@{Mat\_\-VarRead}}
\index{Mat\_\-VarRead@{Mat\_\-VarRead}!MAT@{MAT}}
\subsubsection[{Mat\_\-VarRead}]{\setlength{\rightskip}{0pt plus 5cm}{\bf matvar\_\-t}$\ast$ Mat\_\-VarRead (
\begin{DoxyParamCaption}
\item[{{\bf mat\_\-t} $\ast$}]{mat, }
\item[{const char $\ast$}]{name}
\end{DoxyParamCaption}
)}}
\label{group__MAT_ga3505f63029763eaa73d5a19f1115eb42}


Reads the variable with the given name from a MAT file. 

Reads the next variable in the Matlab MAT file


\begin{DoxyParams}{Parameters}
{\em mat} & Pointer to the MAT file \\
\hline
{\em name} & Name of the variable to read \\
\hline
\end{DoxyParams}
\begin{DoxyReturn}{Returns}
Pointer to the \hyperlink{structmatvar__t}{matvar\_\-t} structure containing the MAT variable information 
\end{DoxyReturn}
\hypertarget{group__MAT_ga1845000f4fc6252ec5ff11c4b9f0759f}{
\index{MAT@{MAT}!Mat\_\-VarReadData@{Mat\_\-VarReadData}}
\index{Mat\_\-VarReadData@{Mat\_\-VarReadData}!MAT@{MAT}}
\subsubsection[{Mat\_\-VarReadData}]{\setlength{\rightskip}{0pt plus 5cm}int Mat\_\-VarReadData (
\begin{DoxyParamCaption}
\item[{{\bf mat\_\-t} $\ast$}]{mat, }
\item[{{\bf matvar\_\-t} $\ast$}]{matvar, }
\item[{void $\ast$}]{data, }
\item[{int $\ast$}]{start, }
\item[{int $\ast$}]{stride, }
\item[{int $\ast$}]{edge}
\end{DoxyParamCaption}
)}}
\label{group__MAT_ga1845000f4fc6252ec5ff11c4b9f0759f}


Reads MAT variable data from a file. 

Reads data from a MAT variable. The variable must have been read by Mat\_\-VarReadInfo.


\begin{DoxyParams}{Parameters}
{\em mat} & MAT file to read data from \\
\hline
{\em matvar} & MAT variable information \\
\hline
{\em data} & pointer to store data in (must be pre-\/allocated) \\
\hline
{\em start} & array of starting indeces \\
\hline
{\em stride} & stride of data \\
\hline
{\em edge} & array specifying the number to read in each direction \\
\hline
\end{DoxyParams}

\begin{DoxyRetVals}{Return values}
{\em 0} & on success \\
\hline
\end{DoxyRetVals}
\hypertarget{group__MAT_gaa8060d7c8e5da0aa9ee5f96e5f1eb30a}{
\index{MAT@{MAT}!Mat\_\-VarReadDataAll@{Mat\_\-VarReadDataAll}}
\index{Mat\_\-VarReadDataAll@{Mat\_\-VarReadDataAll}!MAT@{MAT}}
\subsubsection[{Mat\_\-VarReadDataAll}]{\setlength{\rightskip}{0pt plus 5cm}int Mat\_\-VarReadDataAll (
\begin{DoxyParamCaption}
\item[{{\bf mat\_\-t} $\ast$}]{mat, }
\item[{{\bf matvar\_\-t} $\ast$}]{matvar}
\end{DoxyParamCaption}
)}}
\label{group__MAT_gaa8060d7c8e5da0aa9ee5f96e5f1eb30a}


Reads all the data for a matlab variable. 

Allocates memory for an reads the data for a given matlab variable.


\begin{DoxyParams}{Parameters}
{\em mat} & Matlab MAT file structure pointer \\
\hline
{\em matvar} & Variable whose data is to be read \\
\hline
\end{DoxyParams}
\begin{DoxyReturn}{Returns}
non-\/zero on error 
\end{DoxyReturn}
\hypertarget{group__MAT_gaad61c8449a2106afa697280ff0ee9dd8}{
\index{MAT@{MAT}!Mat\_\-VarReadDataLinear@{Mat\_\-VarReadDataLinear}}
\index{Mat\_\-VarReadDataLinear@{Mat\_\-VarReadDataLinear}!MAT@{MAT}}
\subsubsection[{Mat\_\-VarReadDataLinear}]{\setlength{\rightskip}{0pt plus 5cm}int Mat\_\-VarReadDataLinear (
\begin{DoxyParamCaption}
\item[{{\bf mat\_\-t} $\ast$}]{mat, }
\item[{{\bf matvar\_\-t} $\ast$}]{matvar, }
\item[{void $\ast$}]{data, }
\item[{int}]{start, }
\item[{int}]{stride, }
\item[{int}]{edge}
\end{DoxyParamCaption}
)}}
\label{group__MAT_gaad61c8449a2106afa697280ff0ee9dd8}


Reads MAT variable data from a file. 

Reads data from a MAT variable using a linear indexingmode. The variable must have been read by Mat\_\-VarReadInfo.


\begin{DoxyParams}{Parameters}
{\em mat} & MAT file to read data from \\
\hline
{\em matvar} & MAT variable information \\
\hline
{\em data} & pointer to store data in (must be pre-\/allocated) \\
\hline
{\em start} & starting index \\
\hline
{\em stride} & stride of data \\
\hline
{\em edge} & number of elements to read \\
\hline
\end{DoxyParams}

\begin{DoxyRetVals}{Return values}
{\em 0} & on success \\
\hline
\end{DoxyRetVals}
\hypertarget{group__MAT_ga46da2e45ed96d3f1a6ec643757f2b086}{
\index{MAT@{MAT}!Mat\_\-VarReadInfo@{Mat\_\-VarReadInfo}}
\index{Mat\_\-VarReadInfo@{Mat\_\-VarReadInfo}!MAT@{MAT}}
\subsubsection[{Mat\_\-VarReadInfo}]{\setlength{\rightskip}{0pt plus 5cm}{\bf matvar\_\-t}$\ast$ Mat\_\-VarReadInfo (
\begin{DoxyParamCaption}
\item[{{\bf mat\_\-t} $\ast$}]{mat, }
\item[{const char $\ast$}]{name}
\end{DoxyParamCaption}
)}}
\label{group__MAT_ga46da2e45ed96d3f1a6ec643757f2b086}


Reads the information of a variable with the given name from a MAT file. 

Reads the named variable (or the next variable if name is NULL) information (class,flags-\/complex/global/logical,rank,dimensions,and name) from the Matlab MAT file


\begin{DoxyParams}{Parameters}
{\em mat} & Pointer to the MAT file \\
\hline
{\em name} & Name of the variable to read \\
\hline
\end{DoxyParams}
\begin{DoxyReturn}{Returns}
Pointer to the \hyperlink{structmatvar__t}{matvar\_\-t} structure containing the MAT variable information 
\end{DoxyReturn}
\hypertarget{group__MAT_ga7c321d6aafd93916ba6c5655ad78e9ca}{
\index{MAT@{MAT}!Mat\_\-VarReadNext@{Mat\_\-VarReadNext}}
\index{Mat\_\-VarReadNext@{Mat\_\-VarReadNext}!MAT@{MAT}}
\subsubsection[{Mat\_\-VarReadNext}]{\setlength{\rightskip}{0pt plus 5cm}{\bf matvar\_\-t}$\ast$ Mat\_\-VarReadNext (
\begin{DoxyParamCaption}
\item[{{\bf mat\_\-t} $\ast$}]{mat}
\end{DoxyParamCaption}
)}}
\label{group__MAT_ga7c321d6aafd93916ba6c5655ad78e9ca}


Reads the next variable in a MAT file. 

Reads the next variable in the Matlab MAT file


\begin{DoxyParams}{Parameters}
{\em mat} & Pointer to the MAT file \\
\hline
\end{DoxyParams}
\begin{DoxyReturn}{Returns}
Pointer to the \hyperlink{structmatvar__t}{matvar\_\-t} structure containing the MAT variable information 
\end{DoxyReturn}
\hypertarget{group__MAT_ga72dd99330507b17177e22f9ed3bea5e6}{
\index{MAT@{MAT}!Mat\_\-VarReadNextInfo@{Mat\_\-VarReadNextInfo}}
\index{Mat\_\-VarReadNextInfo@{Mat\_\-VarReadNextInfo}!MAT@{MAT}}
\subsubsection[{Mat\_\-VarReadNextInfo}]{\setlength{\rightskip}{0pt plus 5cm}{\bf matvar\_\-t}$\ast$ Mat\_\-VarReadNextInfo (
\begin{DoxyParamCaption}
\item[{{\bf mat\_\-t} $\ast$}]{mat}
\end{DoxyParamCaption}
)}}
\label{group__MAT_ga72dd99330507b17177e22f9ed3bea5e6}


Reads the information of the next variable in a MAT file. 

Reads the next variable's information (class,flags-\/complex/global/logical, rank,dimensions, name, etc) from the Matlab MAT file. After reading, the MAT file is positioned past the current variable.


\begin{DoxyParams}{Parameters}
{\em mat} & Pointer to the MAT file \\
\hline
\end{DoxyParams}
\begin{DoxyReturn}{Returns}
Pointer to the \hyperlink{structmatvar__t}{matvar\_\-t} structure containing the MAT variable information 
\end{DoxyReturn}
\hypertarget{group__MAT_ga77c5ad24d45047830046fe3ed25da8ad}{
\index{MAT@{MAT}!Mat\_\-VarWrite@{Mat\_\-VarWrite}}
\index{Mat\_\-VarWrite@{Mat\_\-VarWrite}!MAT@{MAT}}
\subsubsection[{Mat\_\-VarWrite}]{\setlength{\rightskip}{0pt plus 5cm}int Mat\_\-VarWrite (
\begin{DoxyParamCaption}
\item[{{\bf mat\_\-t} $\ast$}]{mat, }
\item[{{\bf matvar\_\-t} $\ast$}]{matvar, }
\item[{int}]{compress}
\end{DoxyParamCaption}
)}}
\label{group__MAT_ga77c5ad24d45047830046fe3ed25da8ad}


Writes the given MAT variable to a MAT file. 

Writes the MAT variable information stored in matvar to the given MAT file. The variable will be written to the end of the file.


\begin{DoxyParams}{Parameters}
{\em mat} & MAT file to write to \\
\hline
{\em matvar} & MAT variable information to write \\
\hline
{\em compress} & Whether or not to compress the data (Only valid for version 5 MAT files and variables with numeric data) \\
\hline
\end{DoxyParams}

\begin{DoxyRetVals}{Return values}
{\em 0} & on success \\
\hline
\end{DoxyRetVals}
\hypertarget{group__MAT_ga43179b930fb30c025a153a55a083a98a}{
\index{MAT@{MAT}!Mat\_\-VarWriteData@{Mat\_\-VarWriteData}}
\index{Mat\_\-VarWriteData@{Mat\_\-VarWriteData}!MAT@{MAT}}
\subsubsection[{Mat\_\-VarWriteData}]{\setlength{\rightskip}{0pt plus 5cm}int Mat\_\-VarWriteData (
\begin{DoxyParamCaption}
\item[{{\bf mat\_\-t} $\ast$}]{mat, }
\item[{{\bf matvar\_\-t} $\ast$}]{matvar, }
\item[{void $\ast$}]{data, }
\item[{int $\ast$}]{start, }
\item[{int $\ast$}]{stride, }
\item[{int $\ast$}]{edge}
\end{DoxyParamCaption}
)}}
\label{group__MAT_ga43179b930fb30c025a153a55a083a98a}


Writes the given data to the MAT variable. 

Writes data to a MAT variable. The variable must have previously been written with Mat\_\-VarWriteInfo.


\begin{DoxyParams}{Parameters}
{\em mat} & MAT file to write to \\
\hline
{\em matvar} & MAT variable information to write \\
\hline
{\em data} & pointer to the data to write \\
\hline
{\em start} & array of starting indeces \\
\hline
{\em stride} & stride of data \\
\hline
{\em edge} & array specifying the number to read in each direction \\
\hline
\end{DoxyParams}

\begin{DoxyRetVals}{Return values}
{\em 0} & on success \\
\hline
\end{DoxyRetVals}
\hypertarget{group__MAT_ga1ae164415dfd98cdf48ad07033b6e0bb}{
\index{MAT@{MAT}!Mat\_\-VarWriteInfo@{Mat\_\-VarWriteInfo}}
\index{Mat\_\-VarWriteInfo@{Mat\_\-VarWriteInfo}!MAT@{MAT}}
\subsubsection[{Mat\_\-VarWriteInfo}]{\setlength{\rightskip}{0pt plus 5cm}int Mat\_\-VarWriteInfo (
\begin{DoxyParamCaption}
\item[{{\bf mat\_\-t} $\ast$}]{mat, }
\item[{{\bf matvar\_\-t} $\ast$}]{matvar}
\end{DoxyParamCaption}
)}}
\label{group__MAT_ga1ae164415dfd98cdf48ad07033b6e0bb}


Writes the given MAT variable to a MAT file. 

Writes the MAT variable information stored in matvar to the given MAT file. The variable will be written to the end of the file.


\begin{DoxyParams}{Parameters}
{\em mat} & MAT file to write to \\
\hline
{\em matvar} & MAT variable information to write \\
\hline
\end{DoxyParams}

\begin{DoxyRetVals}{Return values}
{\em 0} & on success \\
\hline
\end{DoxyRetVals}

\hypertarget{group__mat__util}{
\section{MAT File I/O Utitlity Functions}
\label{group__mat__util}\index{MAT File I/O Utitlity Functions@{MAT File I/O Utitlity Functions}}
}
\subsection*{Functions}
\begin{DoxyCompactItemize}
\item 
char $\ast$ \hyperlink{group__mat__util_ga2b342987d3b664345cb233640b611fe9}{strdup\_\-vprintf} (const char $\ast$format, va\_\-list ap)
\begin{DoxyCompactList}\small\item\em Allocates and prints to a new string. \item\end{DoxyCompactList}\item 
char $\ast$ \hyperlink{group__mat__util_ga291b08f933c75fb70e3736b669896ebd}{strdup\_\-printf} (const char $\ast$format,...)
\begin{DoxyCompactList}\small\item\em Allocates and prints to a new string using printf format. \item\end{DoxyCompactList}\item 
int \hyperlink{group__mat__util_gaf348b811ee26bfc923924878cea3c9ba}{Mat\_\-SetVerbose} (int verb, int s)
\begin{DoxyCompactList}\small\item\em Sets verbose parameters. \item\end{DoxyCompactList}\item 
int \hyperlink{group__mat__util_gad75e2962dcaf2ac366f2420bb5b13094}{Mat\_\-SetDebug} (int d)
\begin{DoxyCompactList}\small\item\em Sets verbose parameters. \item\end{DoxyCompactList}\item 
int \hyperlink{group__mat__util_gae7dfa394b111bc908a616f8f5bddaa97}{Mat\_\-Message} (const char $\ast$format,...)
\begin{DoxyCompactList}\small\item\em Log a message unless silent. \item\end{DoxyCompactList}\item 
int \hyperlink{group__mat__util_ga26e00cfb07551be5201fd9e0f04066d9}{Mat\_\-DebugMessage} (int level, const char $\ast$format,...)
\begin{DoxyCompactList}\small\item\em Log a message based on verbose level. \item\end{DoxyCompactList}\item 
int \hyperlink{group__mat__util_ga64a176ea7e27e38d4242a24f3e3bad24}{Mat\_\-VerbMessage} (int level, const char $\ast$format,...)
\begin{DoxyCompactList}\small\item\em Log a message based on verbose level. \item\end{DoxyCompactList}\item 
void \hyperlink{group__mat__util_gaf51f2bfbb5580f575e4dd79757e2b80c}{Mat\_\-Critical} (const char $\ast$format,...)
\begin{DoxyCompactList}\small\item\em Logs a Critical message and returns to the user. \item\end{DoxyCompactList}\item 
void \hyperlink{group__mat__util_ga058b1cb9a4ca36712857d2b3c4de7ffc}{Mat\_\-Error} (const char $\ast$format,...)
\begin{DoxyCompactList}\small\item\em Logs a Critical message and aborts the program. \item\end{DoxyCompactList}\item 
void \hyperlink{group__mat__util_gaa4039c185e807ed2e9682b66fe2ea331}{Mat\_\-Help} (const char $\ast$helpstr\mbox{[}$\,$\mbox{]})
\begin{DoxyCompactList}\small\item\em Prints a helpstring to stdout and exits with status 1. \item\end{DoxyCompactList}\item 
int \hyperlink{group__mat__util_ga333d15dbd2e7a691621a2af8fc7adc3d}{Mat\_\-LogClose} (void)
\begin{DoxyCompactList}\small\item\em Closes the logging system. \item\end{DoxyCompactList}\item 
int \hyperlink{group__mat__util_ga0d30e03216ceaab7c0a4ff878b26f89f}{Mat\_\-LogInit} (const char $\ast$prog\_\-name)
\begin{DoxyCompactList}\small\item\em Intializes the logging system. \item\end{DoxyCompactList}\item 
int \hyperlink{group__mat__util_ga93f4dd8d36413ae7f49260d757e3ab9f}{Mat\_\-LogInitFunc} (const char $\ast$prog\_\-name, void($\ast$log\_\-func)(int log\_\-level, char $\ast$message))
\begin{DoxyCompactList}\small\item\em Intializes the logging system. \item\end{DoxyCompactList}\item 
void \hyperlink{group__mat__util_gafcedc83eb7e4759a8ea5c974c4f801c3}{Mat\_\-Warning} (const char $\ast$format,...)
\begin{DoxyCompactList}\small\item\em Prints a warning message to stdout. \item\end{DoxyCompactList}\item 
size\_\-t \hyperlink{group__mat__util_gab6774aabdc124c540c1e7686d0804940}{Mat\_\-SizeOf} (enum \hyperlink{group__MAT_gacf7b3b879282b7ab3a51190e49bf3453}{matio\_\-types} data\_\-type)
\begin{DoxyCompactList}\small\item\em Calculate the size of MAT data types. \item\end{DoxyCompactList}\end{DoxyCompactItemize}


\subsection{Function Documentation}
\hypertarget{group__mat__util_gaf51f2bfbb5580f575e4dd79757e2b80c}{
\index{mat\_\-util@{mat\_\-util}!Mat\_\-Critical@{Mat\_\-Critical}}
\index{Mat\_\-Critical@{Mat\_\-Critical}!mat_util@{mat\_\-util}}
\subsubsection[{Mat\_\-Critical}]{\setlength{\rightskip}{0pt plus 5cm}void Mat\_\-Critical (
\begin{DoxyParamCaption}
\item[{const char $\ast$}]{format, }
\item[{}]{...}
\end{DoxyParamCaption}
)}}
\label{group__mat__util_gaf51f2bfbb5580f575e4dd79757e2b80c}


Logs a Critical message and returns to the user. 

Logs a Critical message and returns to the user. If the program should stop running, use \hyperlink{group__mat__util_ga058b1cb9a4ca36712857d2b3c4de7ffc}{Mat\_\-Error}


\begin{DoxyParams}{Parameters}
{\em format} & format string identical to printf format \\
\hline
{\em ...} & arguments to the format string \\
\hline
\end{DoxyParams}
\hypertarget{group__mat__util_ga26e00cfb07551be5201fd9e0f04066d9}{
\index{mat\_\-util@{mat\_\-util}!Mat\_\-DebugMessage@{Mat\_\-DebugMessage}}
\index{Mat\_\-DebugMessage@{Mat\_\-DebugMessage}!mat_util@{mat\_\-util}}
\subsubsection[{Mat\_\-DebugMessage}]{\setlength{\rightskip}{0pt plus 5cm}int Mat\_\-DebugMessage (
\begin{DoxyParamCaption}
\item[{int}]{level, }
\item[{const char $\ast$}]{format, }
\item[{}]{...}
\end{DoxyParamCaption}
)}}
\label{group__mat__util_ga26e00cfb07551be5201fd9e0f04066d9}


Log a message based on verbose level. 

If {\itshape level\/} is less than or equal to the set verbose level, the message is printed. If the level is higher than the set verbose level nothing is displayed.


\begin{DoxyParams}{Parameters}
{\em level} & verbose level \\
\hline
{\em format} & message format \\
\hline
\end{DoxyParams}
\hypertarget{group__mat__util_ga058b1cb9a4ca36712857d2b3c4de7ffc}{
\index{mat\_\-util@{mat\_\-util}!Mat\_\-Error@{Mat\_\-Error}}
\index{Mat\_\-Error@{Mat\_\-Error}!mat_util@{mat\_\-util}}
\subsubsection[{Mat\_\-Error}]{\setlength{\rightskip}{0pt plus 5cm}void Mat\_\-Error (
\begin{DoxyParamCaption}
\item[{const char $\ast$}]{format, }
\item[{}]{...}
\end{DoxyParamCaption}
)}}
\label{group__mat__util_ga058b1cb9a4ca36712857d2b3c4de7ffc}


Logs a Critical message and aborts the program. 

Logs an Error message and aborts


\begin{DoxyParams}{Parameters}
{\em format} & format string identical to printf format \\
\hline
{\em ...} & arguments to the format string \\
\hline
\end{DoxyParams}
\hypertarget{group__mat__util_gaa4039c185e807ed2e9682b66fe2ea331}{
\index{mat\_\-util@{mat\_\-util}!Mat\_\-Help@{Mat\_\-Help}}
\index{Mat\_\-Help@{Mat\_\-Help}!mat_util@{mat\_\-util}}
\subsubsection[{Mat\_\-Help}]{\setlength{\rightskip}{0pt plus 5cm}void Mat\_\-Help (
\begin{DoxyParamCaption}
\item[{const char $\ast$}]{helpstr\mbox{[}$\,$\mbox{]}}
\end{DoxyParamCaption}
)}}
\label{group__mat__util_gaa4039c185e807ed2e9682b66fe2ea331}


Prints a helpstring to stdout and exits with status 1. 

Prints the array of strings to stdout and exits with status 1. The array of strings should have NULL as its last element 
\begin{DoxyCode}
 char *helpstr[] = {"My Help string line1","My help string line 2",NULL};
 Mat_Help(helpstr);
\end{DoxyCode}



\begin{DoxyParams}{Parameters}
{\em helpstr} & array of strings with NULL as its last element \\
\hline
\end{DoxyParams}
\hypertarget{group__mat__util_ga333d15dbd2e7a691621a2af8fc7adc3d}{
\index{mat\_\-util@{mat\_\-util}!Mat\_\-LogClose@{Mat\_\-LogClose}}
\index{Mat\_\-LogClose@{Mat\_\-LogClose}!mat_util@{mat\_\-util}}
\subsubsection[{Mat\_\-LogClose}]{\setlength{\rightskip}{0pt plus 5cm}int Mat\_\-LogClose (
\begin{DoxyParamCaption}
\item[{void}]{}
\end{DoxyParamCaption}
)}}
\label{group__mat__util_ga333d15dbd2e7a691621a2af8fc7adc3d}


Closes the logging system. 


\begin{DoxyRetVals}{Return values}
{\em 1} & \\
\hline
\end{DoxyRetVals}
\hypertarget{group__mat__util_ga0d30e03216ceaab7c0a4ff878b26f89f}{
\index{mat\_\-util@{mat\_\-util}!Mat\_\-LogInit@{Mat\_\-LogInit}}
\index{Mat\_\-LogInit@{Mat\_\-LogInit}!mat_util@{mat\_\-util}}
\subsubsection[{Mat\_\-LogInit}]{\setlength{\rightskip}{0pt plus 5cm}int Mat\_\-LogInit (
\begin{DoxyParamCaption}
\item[{const char $\ast$}]{prog\_\-name}
\end{DoxyParamCaption}
)}}
\label{group__mat__util_ga0d30e03216ceaab7c0a4ff878b26f89f}


Intializes the logging system. 


\begin{DoxyParams}{Parameters}
{\em prog\_\-name} & Name of the program initializing the logging functions \\
\hline
\end{DoxyParams}
\begin{DoxyReturn}{Returns}
0 on success 
\end{DoxyReturn}
\hypertarget{group__mat__util_ga93f4dd8d36413ae7f49260d757e3ab9f}{
\index{mat\_\-util@{mat\_\-util}!Mat\_\-LogInitFunc@{Mat\_\-LogInitFunc}}
\index{Mat\_\-LogInitFunc@{Mat\_\-LogInitFunc}!mat_util@{mat\_\-util}}
\subsubsection[{Mat\_\-LogInitFunc}]{\setlength{\rightskip}{0pt plus 5cm}int Mat\_\-LogInitFunc (
\begin{DoxyParamCaption}
\item[{const char $\ast$}]{prog\_\-name, }
\item[{void($\ast$)(int log\_\-level, char $\ast$message)}]{log\_\-func}
\end{DoxyParamCaption}
)}}
\label{group__mat__util_ga93f4dd8d36413ae7f49260d757e3ab9f}


Intializes the logging system. 


\begin{DoxyParams}{Parameters}
{\em prog\_\-name} & Name of the program initializing the logging functions \\
\hline
{\em log\_\-func} & pointer to the function to do the logging \\
\hline
\end{DoxyParams}
\begin{DoxyReturn}{Returns}
0 on success 
\end{DoxyReturn}
\hypertarget{group__mat__util_gae7dfa394b111bc908a616f8f5bddaa97}{
\index{mat\_\-util@{mat\_\-util}!Mat\_\-Message@{Mat\_\-Message}}
\index{Mat\_\-Message@{Mat\_\-Message}!mat_util@{mat\_\-util}}
\subsubsection[{Mat\_\-Message}]{\setlength{\rightskip}{0pt plus 5cm}int Mat\_\-Message (
\begin{DoxyParamCaption}
\item[{const char $\ast$}]{format, }
\item[{}]{...}
\end{DoxyParamCaption}
)}}
\label{group__mat__util_gae7dfa394b111bc908a616f8f5bddaa97}


Log a message unless silent. 

Logs the message unless the silent option is set (See SetVerbose). To log a message based on the verbose level, use \hyperlink{group__mat__util_ga64a176ea7e27e38d4242a24f3e3bad24}{Mat\_\-VerbMessage}


\begin{DoxyParams}{Parameters}
{\em format} & message format \\
\hline
\end{DoxyParams}
\hypertarget{group__mat__util_gad75e2962dcaf2ac366f2420bb5b13094}{
\index{mat\_\-util@{mat\_\-util}!Mat\_\-SetDebug@{Mat\_\-SetDebug}}
\index{Mat\_\-SetDebug@{Mat\_\-SetDebug}!mat_util@{mat\_\-util}}
\subsubsection[{Mat\_\-SetDebug}]{\setlength{\rightskip}{0pt plus 5cm}int Mat\_\-SetDebug (
\begin{DoxyParamCaption}
\item[{int}]{d}
\end{DoxyParamCaption}
)}}
\label{group__mat__util_gad75e2962dcaf2ac366f2420bb5b13094}


Sets verbose parameters. 

Sets the verbose level and silent level. These values are used by programs to determine what information should be printed to the screen


\begin{DoxyParams}{Parameters}
{\em verb} & sets logging verbosity level \\
\hline
{\em s} & sets logging silent level \\
\hline
\end{DoxyParams}
\hypertarget{group__mat__util_gaf348b811ee26bfc923924878cea3c9ba}{
\index{mat\_\-util@{mat\_\-util}!Mat\_\-SetVerbose@{Mat\_\-SetVerbose}}
\index{Mat\_\-SetVerbose@{Mat\_\-SetVerbose}!mat_util@{mat\_\-util}}
\subsubsection[{Mat\_\-SetVerbose}]{\setlength{\rightskip}{0pt plus 5cm}int Mat\_\-SetVerbose (
\begin{DoxyParamCaption}
\item[{int}]{verb, }
\item[{int}]{s}
\end{DoxyParamCaption}
)}}
\label{group__mat__util_gaf348b811ee26bfc923924878cea3c9ba}


Sets verbose parameters. 

Sets the verbose level and silent level. These values are used by programs to determine what information should be printed to the screen


\begin{DoxyParams}{Parameters}
{\em verb} & sets logging verbosity level \\
\hline
{\em s} & sets logging silent level \\
\hline
\end{DoxyParams}
\hypertarget{group__mat__util_gab6774aabdc124c540c1e7686d0804940}{
\index{mat\_\-util@{mat\_\-util}!Mat\_\-SizeOf@{Mat\_\-SizeOf}}
\index{Mat\_\-SizeOf@{Mat\_\-SizeOf}!mat_util@{mat\_\-util}}
\subsubsection[{Mat\_\-SizeOf}]{\setlength{\rightskip}{0pt plus 5cm}size\_\-t Mat\_\-SizeOf (
\begin{DoxyParamCaption}
\item[{enum {\bf matio\_\-types}}]{data\_\-type}
\end{DoxyParamCaption}
)}}
\label{group__mat__util_gab6774aabdc124c540c1e7686d0804940}


Calculate the size of MAT data types. 


\begin{DoxyParams}{Parameters}
{\em data\_\-type} & Data type enumeration \\
\hline
\end{DoxyParams}
\begin{DoxyReturn}{Returns}
size of the data type in bytes 
\end{DoxyReturn}
\hypertarget{group__mat__util_ga64a176ea7e27e38d4242a24f3e3bad24}{
\index{mat\_\-util@{mat\_\-util}!Mat\_\-VerbMessage@{Mat\_\-VerbMessage}}
\index{Mat\_\-VerbMessage@{Mat\_\-VerbMessage}!mat_util@{mat\_\-util}}
\subsubsection[{Mat\_\-VerbMessage}]{\setlength{\rightskip}{0pt plus 5cm}int Mat\_\-VerbMessage (
\begin{DoxyParamCaption}
\item[{int}]{level, }
\item[{const char $\ast$}]{format, }
\item[{}]{...}
\end{DoxyParamCaption}
)}}
\label{group__mat__util_ga64a176ea7e27e38d4242a24f3e3bad24}


Log a message based on verbose level. 

If {\itshape level\/} is less than or equal to the set verbose level, the message is printed. If the level is higher than the set verbose level nothing is displayed.


\begin{DoxyParams}{Parameters}
{\em level} & verbose level \\
\hline
{\em format} & message format \\
\hline
\end{DoxyParams}
\hypertarget{group__mat__util_gafcedc83eb7e4759a8ea5c974c4f801c3}{
\index{mat\_\-util@{mat\_\-util}!Mat\_\-Warning@{Mat\_\-Warning}}
\index{Mat\_\-Warning@{Mat\_\-Warning}!mat_util@{mat\_\-util}}
\subsubsection[{Mat\_\-Warning}]{\setlength{\rightskip}{0pt plus 5cm}void Mat\_\-Warning (
\begin{DoxyParamCaption}
\item[{const char $\ast$}]{format, }
\item[{}]{...}
\end{DoxyParamCaption}
)}}
\label{group__mat__util_gafcedc83eb7e4759a8ea5c974c4f801c3}


Prints a warning message to stdout. 

Logs a warning message then returns


\begin{DoxyParams}{Parameters}
{\em format} & format string identical to printf format \\
\hline
{\em ...} & arguments to the format string \\
\hline
\end{DoxyParams}
\hypertarget{group__mat__util_ga291b08f933c75fb70e3736b669896ebd}{
\index{mat\_\-util@{mat\_\-util}!strdup\_\-printf@{strdup\_\-printf}}
\index{strdup\_\-printf@{strdup\_\-printf}!mat_util@{mat\_\-util}}
\subsubsection[{strdup\_\-printf}]{\setlength{\rightskip}{0pt plus 5cm}char$\ast$ strdup\_\-printf (
\begin{DoxyParamCaption}
\item[{const char $\ast$}]{format, }
\item[{}]{...}
\end{DoxyParamCaption}
)}}
\label{group__mat__util_ga291b08f933c75fb70e3736b669896ebd}


Allocates and prints to a new string using printf format. 


\begin{DoxyParams}{Parameters}
{\em format} & format string \\
\hline
\end{DoxyParams}
\begin{DoxyReturn}{Returns}
Pointer to resulting string, or NULL if there was an error 
\end{DoxyReturn}
\hypertarget{group__mat__util_ga2b342987d3b664345cb233640b611fe9}{
\index{mat\_\-util@{mat\_\-util}!strdup\_\-vprintf@{strdup\_\-vprintf}}
\index{strdup\_\-vprintf@{strdup\_\-vprintf}!mat_util@{mat\_\-util}}
\subsubsection[{strdup\_\-vprintf}]{\setlength{\rightskip}{0pt plus 5cm}char$\ast$ strdup\_\-vprintf (
\begin{DoxyParamCaption}
\item[{const char $\ast$}]{format, }
\item[{va\_\-list}]{ap}
\end{DoxyParamCaption}
)}}
\label{group__mat__util_ga2b342987d3b664345cb233640b611fe9}


Allocates and prints to a new string. 


\begin{DoxyParams}{Parameters}
{\em format} & format string \\
\hline
{\em ap} & variable argument list \\
\hline
\end{DoxyParams}
\begin{DoxyReturn}{Returns}
Newly allocated string with format printed to it 
\end{DoxyReturn}

\chapter{Class Documentation}
\hypertarget{struct__mat__t}{\section{\-\_\-mat\-\_\-t Struct Reference}
\label{struct__mat__t}\index{\-\_\-mat\-\_\-t@{\-\_\-mat\-\_\-t}}
}
\subsection*{Public Attributes}
\begin{DoxyCompactItemize}
\item 
void $\ast$ \hyperlink{struct__mat__t_a85f562e407ca9ad4d2a6e14f839432b7}{fp}
\item 
char $\ast$ \hyperlink{struct__mat__t_a5ed5d0e4e3c4d76b626a8a1772d579c4}{header}
\item 
char $\ast$ \hyperlink{struct__mat__t_a19317c01209959d755d69311960d3eec}{subsys\-\_\-offset}
\item 
char $\ast$ \hyperlink{struct__mat__t_a340b191598135edd03b6dec847f0b1b1}{filename}
\item 
int \hyperlink{struct__mat__t_a729c2bc0afc97485057a5af425635b1a}{version}
\item 
int \hyperlink{struct__mat__t_a99d207977af5e04941ace56d71817a40}{byteswap}
\item 
int \hyperlink{struct__mat__t_aa43288b63b8edb7cadf0b79e2d1df2ee}{mode}
\item 
long \hyperlink{struct__mat__t_a0f87794a6113bd568fe591953e20ddf3}{bof}
\item 
long \hyperlink{struct__mat__t_ab673fe0b330cf4666a66924e37d908d8}{next\-\_\-index}
\item 
long \hyperlink{struct__mat__t_afa714cbc14c9846e8e62df7cae0a9181}{num\-\_\-datasets}
\end{DoxyCompactItemize}


\subsection{Detailed Description}


\subsection{Member Data Documentation}
\hypertarget{struct__mat__t_a0f87794a6113bd568fe591953e20ddf3}{\index{\-\_\-mat\-\_\-t@{\-\_\-mat\-\_\-t}!bof@{bof}}
\index{bof@{bof}!_mat_t@{\-\_\-mat\-\_\-t}}
\subsubsection[{bof}]{\setlength{\rightskip}{0pt plus 5cm}long \-\_\-mat\-\_\-t\-::bof}}\label{struct__mat__t_a0f87794a6113bd568fe591953e20ddf3}
Beginning of file not including any header \hypertarget{struct__mat__t_a99d207977af5e04941ace56d71817a40}{\index{\-\_\-mat\-\_\-t@{\-\_\-mat\-\_\-t}!byteswap@{byteswap}}
\index{byteswap@{byteswap}!_mat_t@{\-\_\-mat\-\_\-t}}
\subsubsection[{byteswap}]{\setlength{\rightskip}{0pt plus 5cm}int \-\_\-mat\-\_\-t\-::byteswap}}\label{struct__mat__t_a99d207977af5e04941ace56d71817a40}
1 if byte swapping is required, 0 otherwise \hypertarget{struct__mat__t_a340b191598135edd03b6dec847f0b1b1}{\index{\-\_\-mat\-\_\-t@{\-\_\-mat\-\_\-t}!filename@{filename}}
\index{filename@{filename}!_mat_t@{\-\_\-mat\-\_\-t}}
\subsubsection[{filename}]{\setlength{\rightskip}{0pt plus 5cm}char$\ast$ \-\_\-mat\-\_\-t\-::filename}}\label{struct__mat__t_a340b191598135edd03b6dec847f0b1b1}
Filename of the M\-A\-T file \hypertarget{struct__mat__t_a85f562e407ca9ad4d2a6e14f839432b7}{\index{\-\_\-mat\-\_\-t@{\-\_\-mat\-\_\-t}!fp@{fp}}
\index{fp@{fp}!_mat_t@{\-\_\-mat\-\_\-t}}
\subsubsection[{fp}]{\setlength{\rightskip}{0pt plus 5cm}void$\ast$ \-\_\-mat\-\_\-t\-::fp}}\label{struct__mat__t_a85f562e407ca9ad4d2a6e14f839432b7}
File pointer for the M\-A\-T file \hypertarget{struct__mat__t_a5ed5d0e4e3c4d76b626a8a1772d579c4}{\index{\-\_\-mat\-\_\-t@{\-\_\-mat\-\_\-t}!header@{header}}
\index{header@{header}!_mat_t@{\-\_\-mat\-\_\-t}}
\subsubsection[{header}]{\setlength{\rightskip}{0pt plus 5cm}char$\ast$ \-\_\-mat\-\_\-t\-::header}}\label{struct__mat__t_a5ed5d0e4e3c4d76b626a8a1772d579c4}
M\-A\-T File header string \hypertarget{struct__mat__t_aa43288b63b8edb7cadf0b79e2d1df2ee}{\index{\-\_\-mat\-\_\-t@{\-\_\-mat\-\_\-t}!mode@{mode}}
\index{mode@{mode}!_mat_t@{\-\_\-mat\-\_\-t}}
\subsubsection[{mode}]{\setlength{\rightskip}{0pt plus 5cm}int \-\_\-mat\-\_\-t\-::mode}}\label{struct__mat__t_aa43288b63b8edb7cadf0b79e2d1df2ee}
Access mode \hypertarget{struct__mat__t_ab673fe0b330cf4666a66924e37d908d8}{\index{\-\_\-mat\-\_\-t@{\-\_\-mat\-\_\-t}!next\-\_\-index@{next\-\_\-index}}
\index{next\-\_\-index@{next\-\_\-index}!_mat_t@{\-\_\-mat\-\_\-t}}
\subsubsection[{next\-\_\-index}]{\setlength{\rightskip}{0pt plus 5cm}long \-\_\-mat\-\_\-t\-::next\-\_\-index}}\label{struct__mat__t_ab673fe0b330cf4666a66924e37d908d8}
Index/\-File position of next variable to read \hypertarget{struct__mat__t_afa714cbc14c9846e8e62df7cae0a9181}{\index{\-\_\-mat\-\_\-t@{\-\_\-mat\-\_\-t}!num\-\_\-datasets@{num\-\_\-datasets}}
\index{num\-\_\-datasets@{num\-\_\-datasets}!_mat_t@{\-\_\-mat\-\_\-t}}
\subsubsection[{num\-\_\-datasets}]{\setlength{\rightskip}{0pt plus 5cm}long \-\_\-mat\-\_\-t\-::num\-\_\-datasets}}\label{struct__mat__t_afa714cbc14c9846e8e62df7cae0a9181}
Number of datasets in the file \hypertarget{struct__mat__t_a19317c01209959d755d69311960d3eec}{\index{\-\_\-mat\-\_\-t@{\-\_\-mat\-\_\-t}!subsys\-\_\-offset@{subsys\-\_\-offset}}
\index{subsys\-\_\-offset@{subsys\-\_\-offset}!_mat_t@{\-\_\-mat\-\_\-t}}
\subsubsection[{subsys\-\_\-offset}]{\setlength{\rightskip}{0pt plus 5cm}char$\ast$ \-\_\-mat\-\_\-t\-::subsys\-\_\-offset}}\label{struct__mat__t_a19317c01209959d755d69311960d3eec}
offset \hypertarget{struct__mat__t_a729c2bc0afc97485057a5af425635b1a}{\index{\-\_\-mat\-\_\-t@{\-\_\-mat\-\_\-t}!version@{version}}
\index{version@{version}!_mat_t@{\-\_\-mat\-\_\-t}}
\subsubsection[{version}]{\setlength{\rightskip}{0pt plus 5cm}int \-\_\-mat\-\_\-t\-::version}}\label{struct__mat__t_a729c2bc0afc97485057a5af425635b1a}
M\-A\-T File version 

The documentation for this struct was generated from the following file\-:\begin{DoxyCompactItemize}
\item 
/home/brawner/catkin\-\_\-ws/src/baxter\-\_\-h2r\-\_\-packages/meldon\-\_\-detection/src/kdes/dependencies/matio/src/matio\-\_\-private.\-h\end{DoxyCompactItemize}

\hypertarget{structComplexSplit}{\section{Complex\-Split Struct Reference}
\label{structComplexSplit}\index{Complex\-Split@{Complex\-Split}}
}


Complex data type using split storage.  




{\ttfamily \#include $<$matio.\-h$>$}

\subsection*{Public Attributes}
\begin{DoxyCompactItemize}
\item 
void $\ast$ \hyperlink{structComplexSplit_ab9c3f2544c4325a372300d4546e374a8}{Re}
\item 
void $\ast$ \hyperlink{structComplexSplit_abdf9792203bd776fb6be4ceebf078402}{Im}
\end{DoxyCompactItemize}


\subsection{Detailed Description}
Complex data type using split storage. 

Complex data type using split real/imaginary pointers 

\subsection{Member Data Documentation}
\hypertarget{structComplexSplit_abdf9792203bd776fb6be4ceebf078402}{\index{Complex\-Split@{Complex\-Split}!Im@{Im}}
\index{Im@{Im}!ComplexSplit@{Complex\-Split}}
\subsubsection[{Im}]{\setlength{\rightskip}{0pt plus 5cm}void$\ast$ Complex\-Split\-::\-Im}}\label{structComplexSplit_abdf9792203bd776fb6be4ceebf078402}
Pointer to the imaginary part \hypertarget{structComplexSplit_ab9c3f2544c4325a372300d4546e374a8}{\index{Complex\-Split@{Complex\-Split}!Re@{Re}}
\index{Re@{Re}!ComplexSplit@{Complex\-Split}}
\subsubsection[{Re}]{\setlength{\rightskip}{0pt plus 5cm}void$\ast$ Complex\-Split\-::\-Re}}\label{structComplexSplit_ab9c3f2544c4325a372300d4546e374a8}
Pointer to the real part 

The documentation for this struct was generated from the following file\-:\begin{DoxyCompactItemize}
\item 
/home/brawner/catkin\-\_\-ws/src/baxter\-\_\-h2r\-\_\-packages/meldon\-\_\-detection/src/kdes/dependencies/matio/src/\hyperlink{matio_8h}{matio.\-h}\end{DoxyCompactItemize}

\hypertarget{structmatvar__internal}{\section{matvar\-\_\-internal Struct Reference}
\label{structmatvar__internal}\index{matvar\-\_\-internal@{matvar\-\_\-internal}}
}
\subsection*{Public Attributes}
\begin{DoxyCompactItemize}
\item 
\hypertarget{structmatvar__internal_a7379b7a8a1bc2e7720cd1df061259a88}{char $\ast$ {\bfseries hdf5\-\_\-name}}\label{structmatvar__internal_a7379b7a8a1bc2e7720cd1df061259a88}

\item 
\hypertarget{structmatvar__internal_af9d2edbde9bb8e2306ec4d4df67c89a2}{hobj\-\_\-ref\-\_\-t {\bfseries hdf5\-\_\-ref}}\label{structmatvar__internal_af9d2edbde9bb8e2306ec4d4df67c89a2}

\item 
\hypertarget{structmatvar__internal_ab821990e65e91f405bbf939010947839}{hid\-\_\-t {\bfseries id}}\label{structmatvar__internal_ab821990e65e91f405bbf939010947839}

\item 
long \hyperlink{structmatvar__internal_af64eef69fa4be3b068789d816dedd619}{fpos}
\item 
long \hyperlink{structmatvar__internal_afd3bfaab126a160bd6855563e1ea0a7e}{datapos}
\item 
\hyperlink{group__MAT_gab0fc888f5a5d79943b16284b1f91c2e8}{mat\-\_\-t} $\ast$ \hyperlink{structmatvar__internal_a73f9faaa71fa20ca1cb48e32bcc67351}{fp}
\item 
z\-\_\-stream $\ast$ \hyperlink{structmatvar__internal_a65fb91d70ebba50dd69f6433a5ef2461}{z}
\end{DoxyCompactItemize}


\subsection{Detailed Description}


\subsection{Member Data Documentation}
\hypertarget{structmatvar__internal_afd3bfaab126a160bd6855563e1ea0a7e}{\index{matvar\-\_\-internal@{matvar\-\_\-internal}!datapos@{datapos}}
\index{datapos@{datapos}!matvar_internal@{matvar\-\_\-internal}}
\subsubsection[{datapos}]{\setlength{\rightskip}{0pt plus 5cm}long matvar\-\_\-internal\-::datapos}}\label{structmatvar__internal_afd3bfaab126a160bd6855563e1ea0a7e}
Offset from the beginning of the M\-A\-T file to the data \hypertarget{structmatvar__internal_a73f9faaa71fa20ca1cb48e32bcc67351}{\index{matvar\-\_\-internal@{matvar\-\_\-internal}!fp@{fp}}
\index{fp@{fp}!matvar_internal@{matvar\-\_\-internal}}
\subsubsection[{fp}]{\setlength{\rightskip}{0pt plus 5cm}{\bf mat\-\_\-t}$\ast$ matvar\-\_\-internal\-::fp}}\label{structmatvar__internal_a73f9faaa71fa20ca1cb48e32bcc67351}
Pointer to the M\-A\-T file structure (mat\-\_\-t) \hypertarget{structmatvar__internal_af64eef69fa4be3b068789d816dedd619}{\index{matvar\-\_\-internal@{matvar\-\_\-internal}!fpos@{fpos}}
\index{fpos@{fpos}!matvar_internal@{matvar\-\_\-internal}}
\subsubsection[{fpos}]{\setlength{\rightskip}{0pt plus 5cm}long matvar\-\_\-internal\-::fpos}}\label{structmatvar__internal_af64eef69fa4be3b068789d816dedd619}
Offset from the beginning of the M\-A\-T file to the variable \hypertarget{structmatvar__internal_a65fb91d70ebba50dd69f6433a5ef2461}{\index{matvar\-\_\-internal@{matvar\-\_\-internal}!z@{z}}
\index{z@{z}!matvar_internal@{matvar\-\_\-internal}}
\subsubsection[{z}]{\setlength{\rightskip}{0pt plus 5cm}z\-\_\-stream$\ast$ matvar\-\_\-internal\-::z}}\label{structmatvar__internal_a65fb91d70ebba50dd69f6433a5ef2461}
zlib compression state 

The documentation for this struct was generated from the following file\-:\begin{DoxyCompactItemize}
\item 
/home/meldon/catkin\-\_\-ws/src/meldon\-\_\-detection/src/kdes/dependencies/matio/src/matio\-\_\-private.\-h\end{DoxyCompactItemize}

\hypertarget{structmatvar__t}{
\section{matvar\_\-t Struct Reference}
\label{structmatvar__t}\index{matvar\_\-t@{matvar\_\-t}}
}


Matlab variable information.  




{\ttfamily \#include $<$matio.h$>$}

\subsection*{Public Attributes}
\begin{DoxyCompactItemize}
\item 
int \hyperlink{structmatvar__t_ae6e0987fef1e35a7e4d0a78b27648035}{nbytes}
\item 
int \hyperlink{structmatvar__t_a84ba70c96ded13cc555fa75b768d9921}{rank}
\item 
enum \hyperlink{group__MAT_gacf7b3b879282b7ab3a51190e49bf3453}{matio\_\-types} \hyperlink{structmatvar__t_ab6aafe9bd77f0f077852593dec438144}{data\_\-type}
\item 
int \hyperlink{structmatvar__t_a9ad1c82e2b568da617e12dc73a26e1f9}{data\_\-size}
\item 
enum \hyperlink{group__MAT_gad4d60ae7b709fc81bfd744fb4c857c40}{matio\_\-classes} \hyperlink{structmatvar__t_aff13035bf3265dd7d9425e5d40c839d4}{class\_\-type}
\item 
int \hyperlink{structmatvar__t_aeb03b3a69f108dc05470b00443a43739}{isComplex}
\item 
int \hyperlink{structmatvar__t_af26c71c4c0ddb14931d15910dddac1bc}{isGlobal}
\item 
int \hyperlink{structmatvar__t_a866c1539e68073a837833d74cd4a65be}{isLogical}
\item 
size\_\-t $\ast$ \hyperlink{structmatvar__t_a86a0006fff01d51e04d598d8fd48e619}{dims}
\item 
char $\ast$ \hyperlink{structmatvar__t_a5f03073a500dae5824d0c7895ae60df9}{name}
\item 
void $\ast$ \hyperlink{structmatvar__t_a70d5c21dc70558757770d4d72ff5d3f4}{data}
\item 
int \hyperlink{structmatvar__t_aff20e87a00691c97340ab07656a13ee7}{mem\_\-conserve}
\item 
int \hyperlink{structmatvar__t_a327abc8f4be853b9f26bc054ce1029e5}{compression}
\item 
struct \hyperlink{structmatvar__internal}{matvar\_\-internal} $\ast$ \hyperlink{structmatvar__t_aa69173e24ef1ac2853548d763f97006c}{internal}
\end{DoxyCompactItemize}


\subsection{Detailed Description}
Matlab variable information. Contains information about a Matlab variable 

\subsection{Member Data Documentation}
\hypertarget{structmatvar__t_aff13035bf3265dd7d9425e5d40c839d4}{
\index{matvar\_\-t@{matvar\_\-t}!class\_\-type@{class\_\-type}}
\index{class\_\-type@{class\_\-type}!matvar_t@{matvar\_\-t}}
\subsubsection[{class\_\-type}]{\setlength{\rightskip}{0pt plus 5cm}enum {\bf matio\_\-classes} {\bf matvar\_\-t::class\_\-type}}}
\label{structmatvar__t_aff13035bf3265dd7d9425e5d40c839d4}
Class type in Matlab(MAT\_\-C\_\-DOUBLE, etc) \hypertarget{structmatvar__t_a327abc8f4be853b9f26bc054ce1029e5}{
\index{matvar\_\-t@{matvar\_\-t}!compression@{compression}}
\index{compression@{compression}!matvar_t@{matvar\_\-t}}
\subsubsection[{compression}]{\setlength{\rightskip}{0pt plus 5cm}int {\bf matvar\_\-t::compression}}}
\label{structmatvar__t_a327abc8f4be853b9f26bc054ce1029e5}
Compression (0=$>$None,1=$>$ZLIB) \hypertarget{structmatvar__t_a70d5c21dc70558757770d4d72ff5d3f4}{
\index{matvar\_\-t@{matvar\_\-t}!data@{data}}
\index{data@{data}!matvar_t@{matvar\_\-t}}
\subsubsection[{data}]{\setlength{\rightskip}{0pt plus 5cm}void$\ast$ {\bf matvar\_\-t::data}}}
\label{structmatvar__t_a70d5c21dc70558757770d4d72ff5d3f4}
Pointer to the data \hypertarget{structmatvar__t_a9ad1c82e2b568da617e12dc73a26e1f9}{
\index{matvar\_\-t@{matvar\_\-t}!data\_\-size@{data\_\-size}}
\index{data\_\-size@{data\_\-size}!matvar_t@{matvar\_\-t}}
\subsubsection[{data\_\-size}]{\setlength{\rightskip}{0pt plus 5cm}int {\bf matvar\_\-t::data\_\-size}}}
\label{structmatvar__t_a9ad1c82e2b568da617e12dc73a26e1f9}
Bytes / element for the data \hypertarget{structmatvar__t_ab6aafe9bd77f0f077852593dec438144}{
\index{matvar\_\-t@{matvar\_\-t}!data\_\-type@{data\_\-type}}
\index{data\_\-type@{data\_\-type}!matvar_t@{matvar\_\-t}}
\subsubsection[{data\_\-type}]{\setlength{\rightskip}{0pt plus 5cm}enum {\bf matio\_\-types} {\bf matvar\_\-t::data\_\-type}}}
\label{structmatvar__t_ab6aafe9bd77f0f077852593dec438144}
Data type(MAT\_\-T\_\-$\ast$) \hypertarget{structmatvar__t_a86a0006fff01d51e04d598d8fd48e619}{
\index{matvar\_\-t@{matvar\_\-t}!dims@{dims}}
\index{dims@{dims}!matvar_t@{matvar\_\-t}}
\subsubsection[{dims}]{\setlength{\rightskip}{0pt plus 5cm}size\_\-t$\ast$ {\bf matvar\_\-t::dims}}}
\label{structmatvar__t_a86a0006fff01d51e04d598d8fd48e619}
Array of lengths for each dimension \hypertarget{structmatvar__t_aa69173e24ef1ac2853548d763f97006c}{
\index{matvar\_\-t@{matvar\_\-t}!internal@{internal}}
\index{internal@{internal}!matvar_t@{matvar\_\-t}}
\subsubsection[{internal}]{\setlength{\rightskip}{0pt plus 5cm}struct {\bf matvar\_\-internal}$\ast$ {\bf matvar\_\-t::internal}}}
\label{structmatvar__t_aa69173e24ef1ac2853548d763f97006c}
matio internal data \hypertarget{structmatvar__t_aeb03b3a69f108dc05470b00443a43739}{
\index{matvar\_\-t@{matvar\_\-t}!isComplex@{isComplex}}
\index{isComplex@{isComplex}!matvar_t@{matvar\_\-t}}
\subsubsection[{isComplex}]{\setlength{\rightskip}{0pt plus 5cm}int {\bf matvar\_\-t::isComplex}}}
\label{structmatvar__t_aeb03b3a69f108dc05470b00443a43739}
non-\/zero if the data is complex, 0 if real \hypertarget{structmatvar__t_af26c71c4c0ddb14931d15910dddac1bc}{
\index{matvar\_\-t@{matvar\_\-t}!isGlobal@{isGlobal}}
\index{isGlobal@{isGlobal}!matvar_t@{matvar\_\-t}}
\subsubsection[{isGlobal}]{\setlength{\rightskip}{0pt plus 5cm}int {\bf matvar\_\-t::isGlobal}}}
\label{structmatvar__t_af26c71c4c0ddb14931d15910dddac1bc}
non-\/zero if the variable is global \hypertarget{structmatvar__t_a866c1539e68073a837833d74cd4a65be}{
\index{matvar\_\-t@{matvar\_\-t}!isLogical@{isLogical}}
\index{isLogical@{isLogical}!matvar_t@{matvar\_\-t}}
\subsubsection[{isLogical}]{\setlength{\rightskip}{0pt plus 5cm}int {\bf matvar\_\-t::isLogical}}}
\label{structmatvar__t_a866c1539e68073a837833d74cd4a65be}
non-\/zero if the variable is logical \hypertarget{structmatvar__t_aff20e87a00691c97340ab07656a13ee7}{
\index{matvar\_\-t@{matvar\_\-t}!mem\_\-conserve@{mem\_\-conserve}}
\index{mem\_\-conserve@{mem\_\-conserve}!matvar_t@{matvar\_\-t}}
\subsubsection[{mem\_\-conserve}]{\setlength{\rightskip}{0pt plus 5cm}int {\bf matvar\_\-t::mem\_\-conserve}}}
\label{structmatvar__t_aff20e87a00691c97340ab07656a13ee7}
1 if Memory was conserved with data \hypertarget{structmatvar__t_a5f03073a500dae5824d0c7895ae60df9}{
\index{matvar\_\-t@{matvar\_\-t}!name@{name}}
\index{name@{name}!matvar_t@{matvar\_\-t}}
\subsubsection[{name}]{\setlength{\rightskip}{0pt plus 5cm}char$\ast$ {\bf matvar\_\-t::name}}}
\label{structmatvar__t_a5f03073a500dae5824d0c7895ae60df9}
Name of the variable \hypertarget{structmatvar__t_ae6e0987fef1e35a7e4d0a78b27648035}{
\index{matvar\_\-t@{matvar\_\-t}!nbytes@{nbytes}}
\index{nbytes@{nbytes}!matvar_t@{matvar\_\-t}}
\subsubsection[{nbytes}]{\setlength{\rightskip}{0pt plus 5cm}int {\bf matvar\_\-t::nbytes}}}
\label{structmatvar__t_ae6e0987fef1e35a7e4d0a78b27648035}
Number of bytes for the MAT variable \hypertarget{structmatvar__t_a84ba70c96ded13cc555fa75b768d9921}{
\index{matvar\_\-t@{matvar\_\-t}!rank@{rank}}
\index{rank@{rank}!matvar_t@{matvar\_\-t}}
\subsubsection[{rank}]{\setlength{\rightskip}{0pt plus 5cm}int {\bf matvar\_\-t::rank}}}
\label{structmatvar__t_a84ba70c96ded13cc555fa75b768d9921}
Rank (Number of dimensions) of the data 

The documentation for this struct was generated from the following file:\begin{DoxyCompactItemize}
\item 
/home/xren/work/kerneldescriptor/KernelDescriptors\_\-CPU/dependencies/matio/src/\hyperlink{matio_8h}{matio.h}\end{DoxyCompactItemize}

\hypertarget{structsparse__t}{
\section{sparse\_\-t Struct Reference}
\label{structsparse__t}\index{sparse\_\-t@{sparse\_\-t}}
}


sparse data information  




{\ttfamily \#include $<$matio.h$>$}

\subsection*{Public Attributes}
\begin{DoxyCompactItemize}
\item 
int \hyperlink{structsparse__t_afd116055ad6a18c27b80333fdac45827}{nzmax}
\item 
int $\ast$ \hyperlink{structsparse__t_ae3b4cd31e90bb824e9b00f7c3dec7af4}{ir}
\item 
int \hyperlink{structsparse__t_a6aa1a08cc2760a36771edd65df8cf111}{nir}
\item 
int $\ast$ \hyperlink{structsparse__t_aeea61c5d15e5cc015a8baf55cc130ee1}{jc}
\item 
int \hyperlink{structsparse__t_aa0ef6a0c8be3ad0e3a222371e68f7dd4}{njc}
\item 
int \hyperlink{structsparse__t_ab692009004070fda2da8274767a0788d}{ndata}
\item 
void $\ast$ \hyperlink{structsparse__t_a8194f9468a7d77514db5ed70b54017bb}{data}
\end{DoxyCompactItemize}


\subsection{Detailed Description}
sparse data information Contains information and data for a sparse matrix 

\subsection{Member Data Documentation}
\hypertarget{structsparse__t_a8194f9468a7d77514db5ed70b54017bb}{
\index{sparse\_\-t@{sparse\_\-t}!data@{data}}
\index{data@{data}!sparse_t@{sparse\_\-t}}
\subsubsection[{data}]{\setlength{\rightskip}{0pt plus 5cm}void$\ast$ {\bf sparse\_\-t::data}}}
\label{structsparse__t_a8194f9468a7d77514db5ed70b54017bb}
Array of data elements \hypertarget{structsparse__t_ae3b4cd31e90bb824e9b00f7c3dec7af4}{
\index{sparse\_\-t@{sparse\_\-t}!ir@{ir}}
\index{ir@{ir}!sparse_t@{sparse\_\-t}}
\subsubsection[{ir}]{\setlength{\rightskip}{0pt plus 5cm}int$\ast$ {\bf sparse\_\-t::ir}}}
\label{structsparse__t_ae3b4cd31e90bb824e9b00f7c3dec7af4}
Array of size nzmax where ir\mbox{[}k\mbox{]} is the row of data\mbox{[}k\mbox{]}. 0 $<$= k $<$= nzmax \hypertarget{structsparse__t_aeea61c5d15e5cc015a8baf55cc130ee1}{
\index{sparse\_\-t@{sparse\_\-t}!jc@{jc}}
\index{jc@{jc}!sparse_t@{sparse\_\-t}}
\subsubsection[{jc}]{\setlength{\rightskip}{0pt plus 5cm}int$\ast$ {\bf sparse\_\-t::jc}}}
\label{structsparse__t_aeea61c5d15e5cc015a8baf55cc130ee1}
Array size N+1 (N is number of columsn) with jc\mbox{[}k\mbox{]} being the index into ir/data of the first non-\/zero element for row k. \hypertarget{structsparse__t_ab692009004070fda2da8274767a0788d}{
\index{sparse\_\-t@{sparse\_\-t}!ndata@{ndata}}
\index{ndata@{ndata}!sparse_t@{sparse\_\-t}}
\subsubsection[{ndata}]{\setlength{\rightskip}{0pt plus 5cm}int {\bf sparse\_\-t::ndata}}}
\label{structsparse__t_ab692009004070fda2da8274767a0788d}
Number of complex/real data values \hypertarget{structsparse__t_a6aa1a08cc2760a36771edd65df8cf111}{
\index{sparse\_\-t@{sparse\_\-t}!nir@{nir}}
\index{nir@{nir}!sparse_t@{sparse\_\-t}}
\subsubsection[{nir}]{\setlength{\rightskip}{0pt plus 5cm}int {\bf sparse\_\-t::nir}}}
\label{structsparse__t_a6aa1a08cc2760a36771edd65df8cf111}
number of elements in ir \hypertarget{structsparse__t_aa0ef6a0c8be3ad0e3a222371e68f7dd4}{
\index{sparse\_\-t@{sparse\_\-t}!njc@{njc}}
\index{njc@{njc}!sparse_t@{sparse\_\-t}}
\subsubsection[{njc}]{\setlength{\rightskip}{0pt plus 5cm}int {\bf sparse\_\-t::njc}}}
\label{structsparse__t_aa0ef6a0c8be3ad0e3a222371e68f7dd4}
Number of elements in jc \hypertarget{structsparse__t_afd116055ad6a18c27b80333fdac45827}{
\index{sparse\_\-t@{sparse\_\-t}!nzmax@{nzmax}}
\index{nzmax@{nzmax}!sparse_t@{sparse\_\-t}}
\subsubsection[{nzmax}]{\setlength{\rightskip}{0pt plus 5cm}int {\bf sparse\_\-t::nzmax}}}
\label{structsparse__t_afd116055ad6a18c27b80333fdac45827}
Maximum number of non-\/zero elements 

The documentation for this struct was generated from the following file:\begin{DoxyCompactItemize}
\item 
/home/xren/work/kerneldescriptor/KernelDescriptors\_\-CPU/dependencies/matio/src/\hyperlink{matio_8h}{matio.h}\end{DoxyCompactItemize}

\chapter{File Documentation}
\hypertarget{endian_8c}{\section{/home/meldon/catkin\-\_\-ws/src/meldon\-\_\-detection/src/kdes/dependencies/matio/src/endian.c File Reference}
\label{endian_8c}\index{/home/meldon/catkin\-\_\-ws/src/meldon\-\_\-detection/src/kdes/dependencies/matio/src/endian.\-c@{/home/meldon/catkin\-\_\-ws/src/meldon\-\_\-detection/src/kdes/dependencies/matio/src/endian.\-c}}
}


Functions to handle endian specifics.  


{\ttfamily \#include $<$stdlib.\-h$>$}\\*
{\ttfamily \#include \char`\"{}matio\-\_\-private.\-h\char`\"{}}\\*
\subsection*{Macros}
\begin{DoxyCompactItemize}
\item 
\hypertarget{endian_8c_a3ca5ecd34b04d6a243c054ac3a57f68d}{\#define \hyperlink{endian_8c_a3ca5ecd34b04d6a243c054ac3a57f68d}{swap}(a, b)~a$^\wedge$=b;b$^\wedge$=a;a$^\wedge$=b}\label{endian_8c_a3ca5ecd34b04d6a243c054ac3a57f68d}

\begin{DoxyCompactList}\small\item\em swap the bytes {\ttfamily a} and {\ttfamily b} \end{DoxyCompactList}\end{DoxyCompactItemize}
\subsection*{Functions}
\begin{DoxyCompactItemize}
\item 
mat\-\_\-int64\-\_\-t \hyperlink{endian_8c_ade92156b3144177bc008b2b34d604eb0}{Mat\-\_\-int64\-Swap} (mat\-\_\-int64\-\_\-t $\ast$a)
\begin{DoxyCompactList}\small\item\em swap the bytes of a 64-\/bit signed integer \end{DoxyCompactList}\item 
mat\-\_\-uint64\-\_\-t \hyperlink{endian_8c_a7920da873d225c6eb9f891eea4782d2f}{Mat\-\_\-uint64\-Swap} (mat\-\_\-uint64\-\_\-t $\ast$a)
\begin{DoxyCompactList}\small\item\em swap the bytes of a 64-\/bit unsigned integer \end{DoxyCompactList}\item 
mat\-\_\-int32\-\_\-t \hyperlink{endian_8c_a2e0153996243f0a34df9a5286087cfa3}{Mat\-\_\-int32\-Swap} (mat\-\_\-int32\-\_\-t $\ast$a)
\begin{DoxyCompactList}\small\item\em swap the bytes of a 32-\/bit signed integer \end{DoxyCompactList}\item 
mat\-\_\-uint32\-\_\-t \hyperlink{endian_8c_a8cb0d0750e2eaf9840d95db531934f4f}{Mat\-\_\-uint32\-Swap} (mat\-\_\-uint32\-\_\-t $\ast$a)
\begin{DoxyCompactList}\small\item\em swap the bytes of a 32-\/bit unsigned integer \end{DoxyCompactList}\item 
mat\-\_\-int16\-\_\-t \hyperlink{endian_8c_a741eb8019dbc3e8addfc0e75adb0dd90}{Mat\-\_\-int16\-Swap} (mat\-\_\-int16\-\_\-t $\ast$a)
\begin{DoxyCompactList}\small\item\em swap the bytes of a 16-\/bit signed integer \end{DoxyCompactList}\item 
mat\-\_\-uint16\-\_\-t \hyperlink{endian_8c_a0fd527794c69f2872e80a6f20cd09fd2}{Mat\-\_\-uint16\-Swap} (mat\-\_\-uint16\-\_\-t $\ast$a)
\begin{DoxyCompactList}\small\item\em swap the bytes of a 16-\/bit unsigned integer \end{DoxyCompactList}\item 
float \hyperlink{endian_8c_aec590b585dd84bbbae74a857922fced2}{Mat\-\_\-float\-Swap} (float $\ast$a)
\begin{DoxyCompactList}\small\item\em swap the bytes of a 4 byte single-\/precision float \end{DoxyCompactList}\item 
double \hyperlink{endian_8c_a7f548ab23c3b06fa90ef646ed43dc558}{Mat\-\_\-double\-Swap} (double $\ast$a)
\begin{DoxyCompactList}\small\item\em swap the bytes of a 4 or 8 byte double-\/precision float \end{DoxyCompactList}\end{DoxyCompactItemize}


\subsection{Detailed Description}
Functions to handle endian specifics. 

\subsection{Function Documentation}
\hypertarget{endian_8c_a7f548ab23c3b06fa90ef646ed43dc558}{\index{endian.\-c@{endian.\-c}!Mat\-\_\-double\-Swap@{Mat\-\_\-double\-Swap}}
\index{Mat\-\_\-double\-Swap@{Mat\-\_\-double\-Swap}!endian.c@{endian.\-c}}
\subsubsection[{Mat\-\_\-double\-Swap}]{\setlength{\rightskip}{0pt plus 5cm}double Mat\-\_\-double\-Swap (
\begin{DoxyParamCaption}
\item[{double $\ast$}]{a}
\end{DoxyParamCaption}
)}}\label{endian_8c_a7f548ab23c3b06fa90ef646ed43dc558}


swap the bytes of a 4 or 8 byte double-\/precision float 


\begin{DoxyParams}{Parameters}
{\em a} & pointer to integer to swap \\
\hline
\end{DoxyParams}
\begin{DoxyReturn}{Returns}
the swapped integer 
\end{DoxyReturn}
\hypertarget{endian_8c_aec590b585dd84bbbae74a857922fced2}{\index{endian.\-c@{endian.\-c}!Mat\-\_\-float\-Swap@{Mat\-\_\-float\-Swap}}
\index{Mat\-\_\-float\-Swap@{Mat\-\_\-float\-Swap}!endian.c@{endian.\-c}}
\subsubsection[{Mat\-\_\-float\-Swap}]{\setlength{\rightskip}{0pt plus 5cm}float Mat\-\_\-float\-Swap (
\begin{DoxyParamCaption}
\item[{float $\ast$}]{a}
\end{DoxyParamCaption}
)}}\label{endian_8c_aec590b585dd84bbbae74a857922fced2}


swap the bytes of a 4 byte single-\/precision float 


\begin{DoxyParams}{Parameters}
{\em a} & pointer to integer to swap \\
\hline
\end{DoxyParams}
\begin{DoxyReturn}{Returns}
the swapped integer 
\end{DoxyReturn}
\hypertarget{endian_8c_a741eb8019dbc3e8addfc0e75adb0dd90}{\index{endian.\-c@{endian.\-c}!Mat\-\_\-int16\-Swap@{Mat\-\_\-int16\-Swap}}
\index{Mat\-\_\-int16\-Swap@{Mat\-\_\-int16\-Swap}!endian.c@{endian.\-c}}
\subsubsection[{Mat\-\_\-int16\-Swap}]{\setlength{\rightskip}{0pt plus 5cm}mat\-\_\-int16\-\_\-t Mat\-\_\-int16\-Swap (
\begin{DoxyParamCaption}
\item[{mat\-\_\-int16\-\_\-t $\ast$}]{a}
\end{DoxyParamCaption}
)}}\label{endian_8c_a741eb8019dbc3e8addfc0e75adb0dd90}


swap the bytes of a 16-\/bit signed integer 


\begin{DoxyParams}{Parameters}
{\em a} & pointer to integer to swap \\
\hline
\end{DoxyParams}
\begin{DoxyReturn}{Returns}
the swapped integer 
\end{DoxyReturn}
\hypertarget{endian_8c_a2e0153996243f0a34df9a5286087cfa3}{\index{endian.\-c@{endian.\-c}!Mat\-\_\-int32\-Swap@{Mat\-\_\-int32\-Swap}}
\index{Mat\-\_\-int32\-Swap@{Mat\-\_\-int32\-Swap}!endian.c@{endian.\-c}}
\subsubsection[{Mat\-\_\-int32\-Swap}]{\setlength{\rightskip}{0pt plus 5cm}mat\-\_\-int32\-\_\-t Mat\-\_\-int32\-Swap (
\begin{DoxyParamCaption}
\item[{mat\-\_\-int32\-\_\-t $\ast$}]{a}
\end{DoxyParamCaption}
)}}\label{endian_8c_a2e0153996243f0a34df9a5286087cfa3}


swap the bytes of a 32-\/bit signed integer 


\begin{DoxyParams}{Parameters}
{\em a} & pointer to integer to swap \\
\hline
\end{DoxyParams}
\begin{DoxyReturn}{Returns}
the swapped integer 
\end{DoxyReturn}
\hypertarget{endian_8c_ade92156b3144177bc008b2b34d604eb0}{\index{endian.\-c@{endian.\-c}!Mat\-\_\-int64\-Swap@{Mat\-\_\-int64\-Swap}}
\index{Mat\-\_\-int64\-Swap@{Mat\-\_\-int64\-Swap}!endian.c@{endian.\-c}}
\subsubsection[{Mat\-\_\-int64\-Swap}]{\setlength{\rightskip}{0pt plus 5cm}mat\-\_\-int64\-\_\-t Mat\-\_\-int64\-Swap (
\begin{DoxyParamCaption}
\item[{mat\-\_\-int64\-\_\-t $\ast$}]{a}
\end{DoxyParamCaption}
)}}\label{endian_8c_ade92156b3144177bc008b2b34d604eb0}


swap the bytes of a 64-\/bit signed integer 


\begin{DoxyParams}{Parameters}
{\em a} & pointer to integer to swap \\
\hline
\end{DoxyParams}
\begin{DoxyReturn}{Returns}
the swapped integer 
\end{DoxyReturn}
\hypertarget{endian_8c_a0fd527794c69f2872e80a6f20cd09fd2}{\index{endian.\-c@{endian.\-c}!Mat\-\_\-uint16\-Swap@{Mat\-\_\-uint16\-Swap}}
\index{Mat\-\_\-uint16\-Swap@{Mat\-\_\-uint16\-Swap}!endian.c@{endian.\-c}}
\subsubsection[{Mat\-\_\-uint16\-Swap}]{\setlength{\rightskip}{0pt plus 5cm}mat\-\_\-uint16\-\_\-t Mat\-\_\-uint16\-Swap (
\begin{DoxyParamCaption}
\item[{mat\-\_\-uint16\-\_\-t $\ast$}]{a}
\end{DoxyParamCaption}
)}}\label{endian_8c_a0fd527794c69f2872e80a6f20cd09fd2}


swap the bytes of a 16-\/bit unsigned integer 


\begin{DoxyParams}{Parameters}
{\em a} & pointer to integer to swap \\
\hline
\end{DoxyParams}
\begin{DoxyReturn}{Returns}
the swapped integer 
\end{DoxyReturn}
\hypertarget{endian_8c_a8cb0d0750e2eaf9840d95db531934f4f}{\index{endian.\-c@{endian.\-c}!Mat\-\_\-uint32\-Swap@{Mat\-\_\-uint32\-Swap}}
\index{Mat\-\_\-uint32\-Swap@{Mat\-\_\-uint32\-Swap}!endian.c@{endian.\-c}}
\subsubsection[{Mat\-\_\-uint32\-Swap}]{\setlength{\rightskip}{0pt plus 5cm}mat\-\_\-uint32\-\_\-t Mat\-\_\-uint32\-Swap (
\begin{DoxyParamCaption}
\item[{mat\-\_\-uint32\-\_\-t $\ast$}]{a}
\end{DoxyParamCaption}
)}}\label{endian_8c_a8cb0d0750e2eaf9840d95db531934f4f}


swap the bytes of a 32-\/bit unsigned integer 


\begin{DoxyParams}{Parameters}
{\em a} & pointer to integer to swap \\
\hline
\end{DoxyParams}
\begin{DoxyReturn}{Returns}
the swapped integer 
\end{DoxyReturn}
\hypertarget{endian_8c_a7920da873d225c6eb9f891eea4782d2f}{\index{endian.\-c@{endian.\-c}!Mat\-\_\-uint64\-Swap@{Mat\-\_\-uint64\-Swap}}
\index{Mat\-\_\-uint64\-Swap@{Mat\-\_\-uint64\-Swap}!endian.c@{endian.\-c}}
\subsubsection[{Mat\-\_\-uint64\-Swap}]{\setlength{\rightskip}{0pt plus 5cm}mat\-\_\-uint64\-\_\-t Mat\-\_\-uint64\-Swap (
\begin{DoxyParamCaption}
\item[{mat\-\_\-uint64\-\_\-t $\ast$}]{a}
\end{DoxyParamCaption}
)}}\label{endian_8c_a7920da873d225c6eb9f891eea4782d2f}


swap the bytes of a 64-\/bit unsigned integer 


\begin{DoxyParams}{Parameters}
{\em a} & pointer to integer to swap \\
\hline
\end{DoxyParams}
\begin{DoxyReturn}{Returns}
the swapped integer 
\end{DoxyReturn}

\hypertarget{inflate_8c}{\section{/home/meldon/catkin\-\_\-ws/src/meldon\-\_\-detection/src/kdes/dependencies/matio/src/inflate.c File Reference}
\label{inflate_8c}\index{/home/meldon/catkin\-\_\-ws/src/meldon\-\_\-detection/src/kdes/dependencies/matio/src/inflate.\-c@{/home/meldon/catkin\-\_\-ws/src/meldon\-\_\-detection/src/kdes/dependencies/matio/src/inflate.\-c}}
}


Functions to inflate data/tags.  


{\ttfamily \#include $<$stdlib.\-h$>$}\\*
{\ttfamily \#include \char`\"{}matio\-\_\-private.\-h\char`\"{}}\\*


\subsection{Detailed Description}
Functions to inflate data/tags. 
\hypertarget{io_8c}{\section{/home/meldon/catkin\-\_\-ws/src/meldon\-\_\-detection/src/kdes/dependencies/matio/src/io.c File Reference}
\label{io_8c}\index{/home/meldon/catkin\-\_\-ws/src/meldon\-\_\-detection/src/kdes/dependencies/matio/src/io.\-c@{/home/meldon/catkin\-\_\-ws/src/meldon\-\_\-detection/src/kdes/dependencies/matio/src/io.\-c}}
}
{\ttfamily \#include $<$stdlib.\-h$>$}\\*
{\ttfamily \#include $<$stdarg.\-h$>$}\\*
{\ttfamily \#include $<$stdio.\-h$>$}\\*
{\ttfamily \#include $<$string.\-h$>$}\\*
{\ttfamily \#include \char`\"{}matio\-\_\-private.\-h\char`\"{}}\\*
\subsection*{Macros}
\begin{DoxyCompactItemize}
\item 
\hypertarget{io_8c_a742fc70e331d7e568bd893c514756a29}{\#define {\bfseries L\-O\-G\-\_\-\-L\-E\-V\-E\-L\-\_\-\-E\-R\-R\-O\-R}~1}\label{io_8c_a742fc70e331d7e568bd893c514756a29}

\item 
\hypertarget{io_8c_af6fcf65d545308183defd25bffce73e3}{\#define {\bfseries L\-O\-G\-\_\-\-L\-E\-V\-E\-L\-\_\-\-C\-R\-I\-T\-I\-C\-A\-L}~1 $<$$<$ 1}\label{io_8c_af6fcf65d545308183defd25bffce73e3}

\item 
\hypertarget{io_8c_af539a66abed2a7a15e3443d70a3cf1e1}{\#define {\bfseries L\-O\-G\-\_\-\-L\-E\-V\-E\-L\-\_\-\-W\-A\-R\-N\-I\-N\-G}~1 $<$$<$ 2}\label{io_8c_af539a66abed2a7a15e3443d70a3cf1e1}

\item 
\hypertarget{io_8c_ada4ad405cae329c67ee7506fbe6463a3}{\#define {\bfseries L\-O\-G\-\_\-\-L\-E\-V\-E\-L\-\_\-\-M\-E\-S\-S\-A\-G\-E}~1 $<$$<$ 3}\label{io_8c_ada4ad405cae329c67ee7506fbe6463a3}

\item 
\hypertarget{io_8c_a130224df8c6bf22a688e3cb74a45689a}{\#define {\bfseries L\-O\-G\-\_\-\-L\-E\-V\-E\-L\-\_\-\-D\-E\-B\-U\-G}~1 $<$$<$ 4}\label{io_8c_a130224df8c6bf22a688e3cb74a45689a}

\end{DoxyCompactItemize}
\subsection*{Functions}
\begin{DoxyCompactItemize}
\item 
char $\ast$ \hyperlink{group__mat__util_ga2b342987d3b664345cb233640b611fe9}{strdup\-\_\-vprintf} (const char $\ast$format, va\-\_\-list ap)
\begin{DoxyCompactList}\small\item\em Allocates and prints to a new string. \end{DoxyCompactList}\item 
char $\ast$ \hyperlink{group__mat__util_ga291b08f933c75fb70e3736b669896ebd}{strdup\-\_\-printf} (const char $\ast$format,...)
\begin{DoxyCompactList}\small\item\em Allocates and prints to a new string using printf format. \end{DoxyCompactList}\item 
int \hyperlink{group__mat__util_gaf348b811ee26bfc923924878cea3c9ba}{Mat\-\_\-\-Set\-Verbose} (int verb, int s)
\begin{DoxyCompactList}\small\item\em Sets verbose parameters. \end{DoxyCompactList}\item 
int \hyperlink{group__mat__util_gad75e2962dcaf2ac366f2420bb5b13094}{Mat\-\_\-\-Set\-Debug} (int d)
\begin{DoxyCompactList}\small\item\em Sets verbose parameters. \end{DoxyCompactList}\item 
int \hyperlink{group__mat__util_gae7dfa394b111bc908a616f8f5bddaa97}{Mat\-\_\-\-Message} (const char $\ast$format,...)
\begin{DoxyCompactList}\small\item\em Log a message unless silent. \end{DoxyCompactList}\item 
int \hyperlink{group__mat__util_ga26e00cfb07551be5201fd9e0f04066d9}{Mat\-\_\-\-Debug\-Message} (int level, const char $\ast$format,...)
\begin{DoxyCompactList}\small\item\em Log a message based on verbose level. \end{DoxyCompactList}\item 
int \hyperlink{group__mat__util_ga64a176ea7e27e38d4242a24f3e3bad24}{Mat\-\_\-\-Verb\-Message} (int level, const char $\ast$format,...)
\begin{DoxyCompactList}\small\item\em Log a message based on verbose level. \end{DoxyCompactList}\item 
void \hyperlink{group__mat__util_gaf51f2bfbb5580f575e4dd79757e2b80c}{Mat\-\_\-\-Critical} (const char $\ast$format,...)
\begin{DoxyCompactList}\small\item\em Logs a Critical message and returns to the user. \end{DoxyCompactList}\item 
void \hyperlink{group__mat__util_ga058b1cb9a4ca36712857d2b3c4de7ffc}{Mat\-\_\-\-Error} (const char $\ast$format,...)
\begin{DoxyCompactList}\small\item\em Logs a Critical message and aborts the program. \end{DoxyCompactList}\item 
void \hyperlink{group__mat__util_gaa4039c185e807ed2e9682b66fe2ea331}{Mat\-\_\-\-Help} (const char $\ast$helpstr\mbox{[}$\,$\mbox{]})
\begin{DoxyCompactList}\small\item\em Prints a helpstring to stdout and exits with status 1. \end{DoxyCompactList}\item 
int \hyperlink{group__mat__util_ga333d15dbd2e7a691621a2af8fc7adc3d}{Mat\-\_\-\-Log\-Close} (void)
\begin{DoxyCompactList}\small\item\em Closes the logging system. \end{DoxyCompactList}\item 
int \hyperlink{group__mat__util_ga0d30e03216ceaab7c0a4ff878b26f89f}{Mat\-\_\-\-Log\-Init} (const char $\ast$prog\-\_\-name)
\begin{DoxyCompactList}\small\item\em Intializes the logging system. \end{DoxyCompactList}\item 
int \hyperlink{group__mat__util_ga93f4dd8d36413ae7f49260d757e3ab9f}{Mat\-\_\-\-Log\-Init\-Func} (const char $\ast$prog\-\_\-name, void($\ast$log\-\_\-func)(int log\-\_\-level, char $\ast$message))
\begin{DoxyCompactList}\small\item\em Intializes the logging system. \end{DoxyCompactList}\item 
void \hyperlink{group__mat__util_gafcedc83eb7e4759a8ea5c974c4f801c3}{Mat\-\_\-\-Warning} (const char $\ast$format,...)
\begin{DoxyCompactList}\small\item\em Prints a warning message to stdout. \end{DoxyCompactList}\item 
size\-\_\-t \hyperlink{group__mat__util_gab6774aabdc124c540c1e7686d0804940}{Mat\-\_\-\-Size\-Of} (enum \hyperlink{group__MAT_gacf7b3b879282b7ab3a51190e49bf3453}{matio\-\_\-types} data\-\_\-type)
\begin{DoxyCompactList}\small\item\em Calculate the size of M\-A\-T data types. \end{DoxyCompactList}\end{DoxyCompactItemize}


\subsection{Detailed Description}
M\-A\-T File I/\-O Utility Functions 
\hypertarget{mat_8c}{
\section{/home/xren/work/kerneldescriptor/KernelDescriptors\_\-CPU/dependencies/matio/src/mat.c File Reference}
\label{mat_8c}\index{/home/xren/work/kerneldescriptor/KernelDescriptors\_\-CPU/dependencies/matio/src/mat.c@{/home/xren/work/kerneldescriptor/KernelDescriptors\_\-CPU/dependencies/matio/src/mat.c}}
}
{\ttfamily \#include $<$stdlib.h$>$}\par
{\ttfamily \#include $<$string.h$>$}\par
{\ttfamily \#include $<$stdio.h$>$}\par
{\ttfamily \#include $<$math.h$>$}\par
{\ttfamily \#include $<$time.h$>$}\par
{\ttfamily \#include \char`\"{}matio\_\-private.h\char`\"{}}\par
{\ttfamily \#include \char`\"{}mat5.h\char`\"{}}\par
{\ttfamily \#include \char`\"{}mat4.h\char`\"{}}\par
{\ttfamily \#include \char`\"{}mat73.h\char`\"{}}\par
\subsection*{Functions}
\begin{DoxyCompactItemize}
\item 
\hyperlink{struct__mat__t}{mat\_\-t} $\ast$ \hyperlink{group__MAT_ga22d404f203af7869c841400e7ad247cf}{Mat\_\-CreateVer} (const char $\ast$matname, const char $\ast$hdr\_\-str, enum \hyperlink{group__MAT_gad03442b8378999189d510e3745c702b7}{mat\_\-ft} mat\_\-file\_\-ver)
\begin{DoxyCompactList}\small\item\em Creates a new Matlab MAT file. \item\end{DoxyCompactList}\item 
\hyperlink{struct__mat__t}{mat\_\-t} $\ast$ \hyperlink{group__MAT_gafbfedb5636a99f0ef867520c47f77d18}{Mat\_\-Open} (const char $\ast$matname, int mode)
\begin{DoxyCompactList}\small\item\em Opens an existing Matlab MAT file. \item\end{DoxyCompactList}\item 
int \hyperlink{group__MAT_ga101c92ff7bde4a2d4615661beba09262}{Mat\_\-Close} (\hyperlink{struct__mat__t}{mat\_\-t} $\ast$mat)
\begin{DoxyCompactList}\small\item\em Closes an open Matlab MAT file. \item\end{DoxyCompactList}\item 
int \hyperlink{group__MAT_ga4d6e3892d2e216c507a744ba0e070d0b}{Mat\_\-Rewind} (\hyperlink{struct__mat__t}{mat\_\-t} $\ast$mat)
\begin{DoxyCompactList}\small\item\em Rewinds a Matlab MAT file to the first variable. \item\end{DoxyCompactList}\item 
size\_\-t \hyperlink{group__MAT_ga2bf682f015b22fa796a8885e997661e7}{Mat\_\-SizeOfClass} (int class\_\-type)
\begin{DoxyCompactList}\small\item\em Returns the size of a Matlab Class. \item\end{DoxyCompactList}\item 
\hyperlink{structmatvar__t}{matvar\_\-t} $\ast$ \hyperlink{group__MAT_gae7c9c3699f6e9c31a9c490300013098c}{Mat\_\-VarCalloc} (void)
\begin{DoxyCompactList}\small\item\em Allocates memory for a new \hyperlink{structmatvar__t}{matvar\_\-t} and initializes all the fields. \item\end{DoxyCompactList}\item 
\hyperlink{structmatvar__t}{matvar\_\-t} $\ast$ \hyperlink{group__MAT_ga1c54a84bb4d810c6fccdb8869489eac4}{Mat\_\-VarCreate} (const char $\ast$name, enum \hyperlink{group__MAT_gad4d60ae7b709fc81bfd744fb4c857c40}{matio\_\-classes} class\_\-type, enum \hyperlink{group__MAT_gacf7b3b879282b7ab3a51190e49bf3453}{matio\_\-types} data\_\-type, int rank, size\_\-t $\ast$dims, void $\ast$data, int opt)
\begin{DoxyCompactList}\small\item\em Creates a MAT Variable with the given name and (optionally) data. \item\end{DoxyCompactList}\item 
int \hyperlink{group__MAT_gabf139e48d48177e5069338fa2919c60a}{Mat\_\-VarDelete} (\hyperlink{struct__mat__t}{mat\_\-t} $\ast$mat, const char $\ast$name)
\begin{DoxyCompactList}\small\item\em Deletes a variable from a file. \item\end{DoxyCompactList}\item 
\hyperlink{structmatvar__t}{matvar\_\-t} $\ast$ \hyperlink{group__MAT_ga7ef80c5d99d7918b2b09db3bea106ecc}{Mat\_\-VarDuplicate} (const \hyperlink{structmatvar__t}{matvar\_\-t} $\ast$in, int opt)
\begin{DoxyCompactList}\small\item\em Duplicates a \hyperlink{structmatvar__t}{matvar\_\-t} structure. \item\end{DoxyCompactList}\item 
void \hyperlink{group__MAT_ga1d14716f7450530fd1c9d02413787f0e}{Mat\_\-VarFree} (\hyperlink{structmatvar__t}{matvar\_\-t} $\ast$matvar)
\begin{DoxyCompactList}\small\item\em Frees all the allocated memory associated with the structure. \item\end{DoxyCompactList}\item 
\hypertarget{mat_8c_ad5ef8f43e9f073fc984dcd57db459908}{
void {\bfseries Mat\_\-VarFree2} (\hyperlink{structmatvar__t}{matvar\_\-t} $\ast$matvar)}
\label{mat_8c_ad5ef8f43e9f073fc984dcd57db459908}

\item 
int \hyperlink{group__MAT_ga9b8d09f631538b14ca29792e0334e349}{Mat\_\-CalcSingleSubscript} (int rank, int $\ast$dims, int $\ast$subs)
\begin{DoxyCompactList}\small\item\em Calculate a single subscript from a set of subscript values. \item\end{DoxyCompactList}\item 
int $\ast$ \hyperlink{group__MAT_gabe2571a4b9b6cff3b31aa6f152deba61}{Mat\_\-CalcSubscripts} (int rank, int $\ast$dims, int index)
\begin{DoxyCompactList}\small\item\em Calculate a set of subscript values from a single(linear) subscript. \item\end{DoxyCompactList}\item 
\hyperlink{structmatvar__t}{matvar\_\-t} $\ast$ \hyperlink{group__MAT_gac1e15063439c0bd3eb0c986514c742dc}{Mat\_\-VarGetCell} (\hyperlink{structmatvar__t}{matvar\_\-t} $\ast$matvar, int index)
\begin{DoxyCompactList}\small\item\em Returns a pointer to the Cell array at a specific index. \item\end{DoxyCompactList}\item 
\hyperlink{structmatvar__t}{matvar\_\-t} $\ast$$\ast$ \hyperlink{group__MAT_ga0732b0a6c40975b036068b9a14422d45}{Mat\_\-VarGetCells} (\hyperlink{structmatvar__t}{matvar\_\-t} $\ast$matvar, int $\ast$start, int $\ast$stride, int $\ast$edge)
\begin{DoxyCompactList}\small\item\em Indexes a cell array. \item\end{DoxyCompactList}\item 
\hyperlink{structmatvar__t}{matvar\_\-t} $\ast$$\ast$ \hyperlink{group__MAT_ga004987d665654409f74eaf8e82bb1380}{Mat\_\-VarGetCellsLinear} (\hyperlink{structmatvar__t}{matvar\_\-t} $\ast$matvar, int start, int stride, int edge)
\begin{DoxyCompactList}\small\item\em Indexes a cell array. \item\end{DoxyCompactList}\item 
size\_\-t \hyperlink{group__MAT_gaeeb798fead2f765bddfb19016c7fdbcc}{Mat\_\-VarGetSize} (\hyperlink{structmatvar__t}{matvar\_\-t} $\ast$matvar)
\begin{DoxyCompactList}\small\item\em Calculates the size of a matlab variable in bytes. \item\end{DoxyCompactList}\item 
int \hyperlink{group__MAT_ga9f8ab8a7e4206c16cb20437acc6960d2}{Mat\_\-VarAddStructField} (\hyperlink{structmatvar__t}{matvar\_\-t} $\ast$matvar, \hyperlink{structmatvar__t}{matvar\_\-t} $\ast$$\ast$fields)
\begin{DoxyCompactList}\small\item\em Adds a field to a structure. \item\end{DoxyCompactList}\item 
int \hyperlink{group__MAT_ga56b9a545990a0f253164018e37111741}{Mat\_\-VarGetNumberOfFields} (\hyperlink{structmatvar__t}{matvar\_\-t} $\ast$matvar)
\begin{DoxyCompactList}\small\item\em Returns the number of fields in a structure variable. \item\end{DoxyCompactList}\item 
\hyperlink{structmatvar__t}{matvar\_\-t} $\ast$ \hyperlink{group__MAT_ga7018bfe6934b96ae32e8f2a1712eefab}{Mat\_\-VarGetStructField} (\hyperlink{structmatvar__t}{matvar\_\-t} $\ast$matvar, void $\ast$name\_\-or\_\-index, int opt, int index)
\begin{DoxyCompactList}\small\item\em Finds a field of a structure. \item\end{DoxyCompactList}\item 
\hyperlink{structmatvar__t}{matvar\_\-t} $\ast$ \hyperlink{group__MAT_ga509178d7dc15faf9f7cd0440df6009b9}{Mat\_\-VarGetStructs} (\hyperlink{structmatvar__t}{matvar\_\-t} $\ast$matvar, int $\ast$start, int $\ast$stride, int $\ast$edge, int copy\_\-fields)
\begin{DoxyCompactList}\small\item\em Indexes a structure. \item\end{DoxyCompactList}\item 
\hyperlink{structmatvar__t}{matvar\_\-t} $\ast$ \hyperlink{group__MAT_gaa56680fb7b2cd3d410f659e945da8141}{Mat\_\-VarGetStructsLinear} (\hyperlink{structmatvar__t}{matvar\_\-t} $\ast$matvar, int start, int stride, int edge, int copy\_\-fields)
\begin{DoxyCompactList}\small\item\em Indexes a structure. \item\end{DoxyCompactList}\item 
void \hyperlink{group__MAT_ga9100c145e338b84b55d5d0795d5d390a}{Mat\_\-VarPrint} (\hyperlink{structmatvar__t}{matvar\_\-t} $\ast$matvar, int printdata)
\begin{DoxyCompactList}\small\item\em Prints the variable information. \item\end{DoxyCompactList}\item 
int \hyperlink{group__MAT_ga1845000f4fc6252ec5ff11c4b9f0759f}{Mat\_\-VarReadData} (\hyperlink{struct__mat__t}{mat\_\-t} $\ast$mat, \hyperlink{structmatvar__t}{matvar\_\-t} $\ast$matvar, void $\ast$data, int $\ast$start, int $\ast$stride, int $\ast$edge)
\begin{DoxyCompactList}\small\item\em Reads MAT variable data from a file. \item\end{DoxyCompactList}\item 
int \hyperlink{group__MAT_gaa8060d7c8e5da0aa9ee5f96e5f1eb30a}{Mat\_\-VarReadDataAll} (\hyperlink{struct__mat__t}{mat\_\-t} $\ast$mat, \hyperlink{structmatvar__t}{matvar\_\-t} $\ast$matvar)
\begin{DoxyCompactList}\small\item\em Reads all the data for a matlab variable. \item\end{DoxyCompactList}\item 
int \hyperlink{group__MAT_gaad61c8449a2106afa697280ff0ee9dd8}{Mat\_\-VarReadDataLinear} (\hyperlink{struct__mat__t}{mat\_\-t} $\ast$mat, \hyperlink{structmatvar__t}{matvar\_\-t} $\ast$matvar, void $\ast$data, int start, int stride, int edge)
\begin{DoxyCompactList}\small\item\em Reads MAT variable data from a file. \item\end{DoxyCompactList}\item 
\hyperlink{structmatvar__t}{matvar\_\-t} $\ast$ \hyperlink{group__MAT_ga72dd99330507b17177e22f9ed3bea5e6}{Mat\_\-VarReadNextInfo} (\hyperlink{struct__mat__t}{mat\_\-t} $\ast$mat)
\begin{DoxyCompactList}\small\item\em Reads the information of the next variable in a MAT file. \item\end{DoxyCompactList}\item 
\hyperlink{structmatvar__t}{matvar\_\-t} $\ast$ \hyperlink{group__MAT_ga46da2e45ed96d3f1a6ec643757f2b086}{Mat\_\-VarReadInfo} (\hyperlink{struct__mat__t}{mat\_\-t} $\ast$mat, const char $\ast$name)
\begin{DoxyCompactList}\small\item\em Reads the information of a variable with the given name from a MAT file. \item\end{DoxyCompactList}\item 
\hyperlink{structmatvar__t}{matvar\_\-t} $\ast$ \hyperlink{group__MAT_ga3505f63029763eaa73d5a19f1115eb42}{Mat\_\-VarRead} (\hyperlink{struct__mat__t}{mat\_\-t} $\ast$mat, const char $\ast$name)
\begin{DoxyCompactList}\small\item\em Reads the variable with the given name from a MAT file. \item\end{DoxyCompactList}\item 
\hyperlink{structmatvar__t}{matvar\_\-t} $\ast$ \hyperlink{group__MAT_ga7c321d6aafd93916ba6c5655ad78e9ca}{Mat\_\-VarReadNext} (\hyperlink{struct__mat__t}{mat\_\-t} $\ast$mat)
\begin{DoxyCompactList}\small\item\em Reads the next variable in a MAT file. \item\end{DoxyCompactList}\item 
int \hyperlink{group__MAT_ga1ae164415dfd98cdf48ad07033b6e0bb}{Mat\_\-VarWriteInfo} (\hyperlink{struct__mat__t}{mat\_\-t} $\ast$mat, \hyperlink{structmatvar__t}{matvar\_\-t} $\ast$matvar)
\begin{DoxyCompactList}\small\item\em Writes the given MAT variable to a MAT file. \item\end{DoxyCompactList}\item 
int \hyperlink{group__MAT_ga43179b930fb30c025a153a55a083a98a}{Mat\_\-VarWriteData} (\hyperlink{struct__mat__t}{mat\_\-t} $\ast$mat, \hyperlink{structmatvar__t}{matvar\_\-t} $\ast$matvar, void $\ast$data, int $\ast$start, int $\ast$stride, int $\ast$edge)
\begin{DoxyCompactList}\small\item\em Writes the given data to the MAT variable. \item\end{DoxyCompactList}\item 
int \hyperlink{group__MAT_ga77c5ad24d45047830046fe3ed25da8ad}{Mat\_\-VarWrite} (\hyperlink{struct__mat__t}{mat\_\-t} $\ast$mat, \hyperlink{structmatvar__t}{matvar\_\-t} $\ast$matvar, int compress)
\begin{DoxyCompactList}\small\item\em Writes the given MAT variable to a MAT file. \item\end{DoxyCompactList}\end{DoxyCompactItemize}


\subsection{Detailed Description}
Matlab MAT version 5 file functions 
\hypertarget{mat4_8c}{
\section{/home/xren/work/kerneldescriptor/KernelDescriptors\_\-CPU/dependencies/matio/src/mat4.c File Reference}
\label{mat4_8c}\index{/home/xren/work/kerneldescriptor/KernelDescriptors\_\-CPU/dependencies/matio/src/mat4.c@{/home/xren/work/kerneldescriptor/KernelDescriptors\_\-CPU/dependencies/matio/src/mat4.c}}
}
{\ttfamily \#include $<$stdlib.h$>$}\par
{\ttfamily \#include $<$stdio.h$>$}\par
{\ttfamily \#include $<$math.h$>$}\par
{\ttfamily \#include \char`\"{}matio\_\-private.h\char`\"{}}\par
{\ttfamily \#include \char`\"{}mat4.h\char`\"{}}\par
\subsection*{Functions}
\begin{DoxyCompactItemize}
\item 
void \hyperlink{mat4_8c_a48f37c81ddb08c15bd949ccd31a8aee9}{Read4} (\hyperlink{struct__mat__t}{mat\_\-t} $\ast$mat, \hyperlink{structmatvar__t}{matvar\_\-t} $\ast$matvar)
\item 
int \hyperlink{mat4_8c_a6c3186af92bf92da475cd4142c325ba5}{ReadData4} (\hyperlink{struct__mat__t}{mat\_\-t} $\ast$mat, \hyperlink{structmatvar__t}{matvar\_\-t} $\ast$matvar, void $\ast$data, int $\ast$start, int $\ast$stride, int $\ast$edge)
\item 
\hyperlink{structmatvar__t}{matvar\_\-t} $\ast$ \hyperlink{mat4_8c_a494a64be8f002874f2048687561f65d3}{Mat\_\-VarReadNextInfo4} (\hyperlink{struct__mat__t}{mat\_\-t} $\ast$mat)
\item 
void \hyperlink{mat4_8c_a42abd591673bea010f4b71118f6242cf}{Mat\_\-VarPrint4} (\hyperlink{structmatvar__t}{matvar\_\-t} $\ast$matvar, int printdata)
\end{DoxyCompactItemize}


\subsection{Detailed Description}
Matlab MAT version 4 file functions 

\subsection{Function Documentation}
\hypertarget{mat4_8c_a42abd591673bea010f4b71118f6242cf}{
\index{mat4.c@{mat4.c}!Mat\_\-VarPrint4@{Mat\_\-VarPrint4}}
\index{Mat\_\-VarPrint4@{Mat\_\-VarPrint4}!mat4.c@{mat4.c}}
\subsubsection[{Mat\_\-VarPrint4}]{\setlength{\rightskip}{0pt plus 5cm}void Mat\_\-VarPrint4 (
\begin{DoxyParamCaption}
\item[{{\bf matvar\_\-t} $\ast$}]{matvar, }
\item[{int}]{printdata}
\end{DoxyParamCaption}
)}}
\label{mat4_8c_a42abd591673bea010f4b71118f6242cf}
\hypertarget{mat4_8c_a494a64be8f002874f2048687561f65d3}{
\index{mat4.c@{mat4.c}!Mat\_\-VarReadNextInfo4@{Mat\_\-VarReadNextInfo4}}
\index{Mat\_\-VarReadNextInfo4@{Mat\_\-VarReadNextInfo4}!mat4.c@{mat4.c}}
\subsubsection[{Mat\_\-VarReadNextInfo4}]{\setlength{\rightskip}{0pt plus 5cm}{\bf matvar\_\-t}$\ast$ Mat\_\-VarReadNextInfo4 (
\begin{DoxyParamCaption}
\item[{{\bf mat\_\-t} $\ast$}]{mat}
\end{DoxyParamCaption}
)}}
\label{mat4_8c_a494a64be8f002874f2048687561f65d3}
\hypertarget{mat4_8c_a48f37c81ddb08c15bd949ccd31a8aee9}{
\index{mat4.c@{mat4.c}!Read4@{Read4}}
\index{Read4@{Read4}!mat4.c@{mat4.c}}
\subsubsection[{Read4}]{\setlength{\rightskip}{0pt plus 5cm}void Read4 (
\begin{DoxyParamCaption}
\item[{{\bf mat\_\-t} $\ast$}]{mat, }
\item[{{\bf matvar\_\-t} $\ast$}]{matvar}
\end{DoxyParamCaption}
)}}
\label{mat4_8c_a48f37c81ddb08c15bd949ccd31a8aee9}
\hypertarget{mat4_8c_a6c3186af92bf92da475cd4142c325ba5}{
\index{mat4.c@{mat4.c}!ReadData4@{ReadData4}}
\index{ReadData4@{ReadData4}!mat4.c@{mat4.c}}
\subsubsection[{ReadData4}]{\setlength{\rightskip}{0pt plus 5cm}int ReadData4 (
\begin{DoxyParamCaption}
\item[{{\bf mat\_\-t} $\ast$}]{mat, }
\item[{{\bf matvar\_\-t} $\ast$}]{matvar, }
\item[{void $\ast$}]{data, }
\item[{int $\ast$}]{start, }
\item[{int $\ast$}]{stride, }
\item[{int $\ast$}]{edge}
\end{DoxyParamCaption}
)}}
\label{mat4_8c_a6c3186af92bf92da475cd4142c325ba5}

\hypertarget{mat5_8c}{
\section{/home/xren/work/kerneldescriptor/KernelDescriptors\_\-CPU/dependencies/matio/src/mat5.c File Reference}
\label{mat5_8c}\index{/home/xren/work/kerneldescriptor/KernelDescriptors\_\-CPU/dependencies/matio/src/mat5.c@{/home/xren/work/kerneldescriptor/KernelDescriptors\_\-CPU/dependencies/matio/src/mat5.c}}
}
{\ttfamily \#include $<$stdlib.h$>$}\par
{\ttfamily \#include $<$string.h$>$}\par
{\ttfamily \#include $<$stdio.h$>$}\par
{\ttfamily \#include $<$math.h$>$}\par
{\ttfamily \#include $<$time.h$>$}\par
{\ttfamily \#include \char`\"{}matio\_\-private.h\char`\"{}}\par
{\ttfamily \#include \char`\"{}mat5.h\char`\"{}}\par
\subsection*{Defines}
\begin{DoxyCompactItemize}
\item 
\hypertarget{mat5_8c_a82bacecc4afc633b61bc3dc8ef88d1ed}{
\#define {\bfseries TYPE\_\-FROM\_\-TAG}(a)~(enum \hyperlink{group__MAT_gacf7b3b879282b7ab3a51190e49bf3453}{matio\_\-types})((a) \& 0x000000ff)}
\label{mat5_8c_a82bacecc4afc633b61bc3dc8ef88d1ed}

\item 
\hypertarget{mat5_8c_a85a616d27707e89bda9fd2e9bbb6a586}{
\#define {\bfseries CLASS\_\-FROM\_\-ARRAY\_\-FLAGS}(a)~(enum \hyperlink{group__MAT_gad4d60ae7b709fc81bfd744fb4c857c40}{matio\_\-classes})((a) \& 0x000000ff)}
\label{mat5_8c_a85a616d27707e89bda9fd2e9bbb6a586}

\end{DoxyCompactItemize}
\subsection*{Functions}
\begin{DoxyCompactItemize}
\item 
\hyperlink{struct__mat__t}{mat\_\-t} $\ast$ \hyperlink{mat5_8c_aacff52cdf3427b35a54c111fa3d7bb21}{Mat\_\-Create5} (const char $\ast$matname, const char $\ast$hdr\_\-str)
\item 
int \hyperlink{mat5_8c_ab391ca08249b5e5738b88fd791022610}{WriteCharData} (\hyperlink{struct__mat__t}{mat\_\-t} $\ast$mat, void $\ast$data, int N, enum \hyperlink{group__MAT_gacf7b3b879282b7ab3a51190e49bf3453}{matio\_\-types} data\_\-type)
\item 
int \hyperlink{mat5_8c_a04eba34668a0de871c4f245702f802f7}{WriteDataSlab2} (\hyperlink{struct__mat__t}{mat\_\-t} $\ast$mat, void $\ast$data, enum \hyperlink{group__MAT_gacf7b3b879282b7ab3a51190e49bf3453}{matio\_\-types} data\_\-type, size\_\-t $\ast$dims, int $\ast$start, int $\ast$stride, int $\ast$edge)
\item 
int \hyperlink{mat5_8c_a5afeaabef5e10f73a5805ce6de4acc8b}{WriteCharDataSlab2} (\hyperlink{struct__mat__t}{mat\_\-t} $\ast$mat, void $\ast$data, enum \hyperlink{group__MAT_gacf7b3b879282b7ab3a51190e49bf3453}{matio\_\-types} data\_\-type, size\_\-t $\ast$dims, int $\ast$start, int $\ast$stride, int $\ast$edge)
\item 
int \hyperlink{mat5_8c_a1bec2ccf2a3b48706edd32e63744d364}{WriteData} (\hyperlink{struct__mat__t}{mat\_\-t} $\ast$mat, void $\ast$data, int N, enum \hyperlink{group__MAT_gacf7b3b879282b7ab3a51190e49bf3453}{matio\_\-types} data\_\-type)
\begin{DoxyCompactList}\small\item\em Writes the data buffer to the file. \item\end{DoxyCompactList}\item 
void \hyperlink{mat5_8c_abd8669832a02e759fe190bf2724f60ae}{Read5} (\hyperlink{struct__mat__t}{mat\_\-t} $\ast$mat, \hyperlink{structmatvar__t}{matvar\_\-t} $\ast$matvar)
\item 
int \hyperlink{mat5_8c_af8a259573b7b3a89555c5a9bc1860e19}{ReadData5} (\hyperlink{struct__mat__t}{mat\_\-t} $\ast$mat, \hyperlink{structmatvar__t}{matvar\_\-t} $\ast$matvar, void $\ast$data, int $\ast$start, int $\ast$stride, int $\ast$edge)
\item 
int \hyperlink{mat5_8c_a4d546a62c4b429d9b94fe873015e6a24}{Mat\_\-VarWrite5} (\hyperlink{struct__mat__t}{mat\_\-t} $\ast$mat, \hyperlink{structmatvar__t}{matvar\_\-t} $\ast$matvar, int compress)
\item 
void \hyperlink{mat5_8c_a80e4ba45ec110d05657f8f39ffd9ee27}{WriteInfo5} (\hyperlink{struct__mat__t}{mat\_\-t} $\ast$mat, \hyperlink{structmatvar__t}{matvar\_\-t} $\ast$matvar)
\item 
\hyperlink{structmatvar__t}{matvar\_\-t} $\ast$ \hyperlink{mat5_8c_a5defc934cf088b13347e50ea7f276ea3}{Mat\_\-VarReadNextInfo5} (\hyperlink{struct__mat__t}{mat\_\-t} $\ast$mat)
\end{DoxyCompactItemize}


\subsection{Detailed Description}
Matlab MAT version 5 file functions 

\subsection{Function Documentation}
\hypertarget{mat5_8c_aacff52cdf3427b35a54c111fa3d7bb21}{
\index{mat5.c@{mat5.c}!Mat\_\-Create5@{Mat\_\-Create5}}
\index{Mat\_\-Create5@{Mat\_\-Create5}!mat5.c@{mat5.c}}
\subsubsection[{Mat\_\-Create5}]{\setlength{\rightskip}{0pt plus 5cm}{\bf mat\_\-t}$\ast$ Mat\_\-Create5 (
\begin{DoxyParamCaption}
\item[{const char $\ast$}]{matname, }
\item[{const char $\ast$}]{hdr\_\-str}
\end{DoxyParamCaption}
)}}
\label{mat5_8c_aacff52cdf3427b35a54c111fa3d7bb21}
\hypertarget{mat5_8c_a5defc934cf088b13347e50ea7f276ea3}{
\index{mat5.c@{mat5.c}!Mat\_\-VarReadNextInfo5@{Mat\_\-VarReadNextInfo5}}
\index{Mat\_\-VarReadNextInfo5@{Mat\_\-VarReadNextInfo5}!mat5.c@{mat5.c}}
\subsubsection[{Mat\_\-VarReadNextInfo5}]{\setlength{\rightskip}{0pt plus 5cm}{\bf matvar\_\-t}$\ast$ Mat\_\-VarReadNextInfo5 (
\begin{DoxyParamCaption}
\item[{{\bf mat\_\-t} $\ast$}]{mat}
\end{DoxyParamCaption}
)}}
\label{mat5_8c_a5defc934cf088b13347e50ea7f276ea3}
\hypertarget{mat5_8c_a4d546a62c4b429d9b94fe873015e6a24}{
\index{mat5.c@{mat5.c}!Mat\_\-VarWrite5@{Mat\_\-VarWrite5}}
\index{Mat\_\-VarWrite5@{Mat\_\-VarWrite5}!mat5.c@{mat5.c}}
\subsubsection[{Mat\_\-VarWrite5}]{\setlength{\rightskip}{0pt plus 5cm}int Mat\_\-VarWrite5 (
\begin{DoxyParamCaption}
\item[{{\bf mat\_\-t} $\ast$}]{mat, }
\item[{{\bf matvar\_\-t} $\ast$}]{matvar, }
\item[{int}]{compress}
\end{DoxyParamCaption}
)}}
\label{mat5_8c_a4d546a62c4b429d9b94fe873015e6a24}
\hypertarget{mat5_8c_abd8669832a02e759fe190bf2724f60ae}{
\index{mat5.c@{mat5.c}!Read5@{Read5}}
\index{Read5@{Read5}!mat5.c@{mat5.c}}
\subsubsection[{Read5}]{\setlength{\rightskip}{0pt plus 5cm}void Read5 (
\begin{DoxyParamCaption}
\item[{{\bf mat\_\-t} $\ast$}]{mat, }
\item[{{\bf matvar\_\-t} $\ast$}]{matvar}
\end{DoxyParamCaption}
)}}
\label{mat5_8c_abd8669832a02e759fe190bf2724f60ae}
\hypertarget{mat5_8c_af8a259573b7b3a89555c5a9bc1860e19}{
\index{mat5.c@{mat5.c}!ReadData5@{ReadData5}}
\index{ReadData5@{ReadData5}!mat5.c@{mat5.c}}
\subsubsection[{ReadData5}]{\setlength{\rightskip}{0pt plus 5cm}int ReadData5 (
\begin{DoxyParamCaption}
\item[{{\bf mat\_\-t} $\ast$}]{mat, }
\item[{{\bf matvar\_\-t} $\ast$}]{matvar, }
\item[{void $\ast$}]{data, }
\item[{int $\ast$}]{start, }
\item[{int $\ast$}]{stride, }
\item[{int $\ast$}]{edge}
\end{DoxyParamCaption}
)}}
\label{mat5_8c_af8a259573b7b3a89555c5a9bc1860e19}
\hypertarget{mat5_8c_ab391ca08249b5e5738b88fd791022610}{
\index{mat5.c@{mat5.c}!WriteCharData@{WriteCharData}}
\index{WriteCharData@{WriteCharData}!mat5.c@{mat5.c}}
\subsubsection[{WriteCharData}]{\setlength{\rightskip}{0pt plus 5cm}int WriteCharData (
\begin{DoxyParamCaption}
\item[{{\bf mat\_\-t} $\ast$}]{mat, }
\item[{void $\ast$}]{data, }
\item[{int}]{N, }
\item[{enum {\bf matio\_\-types}}]{data\_\-type}
\end{DoxyParamCaption}
)}}
\label{mat5_8c_ab391ca08249b5e5738b88fd791022610}
\hypertarget{mat5_8c_a5afeaabef5e10f73a5805ce6de4acc8b}{
\index{mat5.c@{mat5.c}!WriteCharDataSlab2@{WriteCharDataSlab2}}
\index{WriteCharDataSlab2@{WriteCharDataSlab2}!mat5.c@{mat5.c}}
\subsubsection[{WriteCharDataSlab2}]{\setlength{\rightskip}{0pt plus 5cm}int WriteCharDataSlab2 (
\begin{DoxyParamCaption}
\item[{{\bf mat\_\-t} $\ast$}]{mat, }
\item[{void $\ast$}]{data, }
\item[{enum {\bf matio\_\-types}}]{data\_\-type, }
\item[{size\_\-t $\ast$}]{dims, }
\item[{int $\ast$}]{start, }
\item[{int $\ast$}]{stride, }
\item[{int $\ast$}]{edge}
\end{DoxyParamCaption}
)}}
\label{mat5_8c_a5afeaabef5e10f73a5805ce6de4acc8b}
\hypertarget{mat5_8c_a1bec2ccf2a3b48706edd32e63744d364}{
\index{mat5.c@{mat5.c}!WriteData@{WriteData}}
\index{WriteData@{WriteData}!mat5.c@{mat5.c}}
\subsubsection[{WriteData}]{\setlength{\rightskip}{0pt plus 5cm}int WriteData (
\begin{DoxyParamCaption}
\item[{{\bf mat\_\-t} $\ast$}]{mat, }
\item[{void $\ast$}]{data, }
\item[{int}]{N, }
\item[{enum {\bf matio\_\-types}}]{data\_\-type}
\end{DoxyParamCaption}
)}}
\label{mat5_8c_a1bec2ccf2a3b48706edd32e63744d364}


Writes the data buffer to the file. 


\begin{DoxyParams}{Parameters}
{\em mat} & MAT file pointer \\
\hline
{\em data} & pointer to the data to write \\
\hline
{\em N} & number of elements to write \\
\hline
{\em data\_\-type} & data type of the data \\
\hline
\end{DoxyParams}
\begin{DoxyReturn}{Returns}
number of bytes written 
\end{DoxyReturn}
\hypertarget{mat5_8c_a04eba34668a0de871c4f245702f802f7}{
\index{mat5.c@{mat5.c}!WriteDataSlab2@{WriteDataSlab2}}
\index{WriteDataSlab2@{WriteDataSlab2}!mat5.c@{mat5.c}}
\subsubsection[{WriteDataSlab2}]{\setlength{\rightskip}{0pt plus 5cm}int WriteDataSlab2 (
\begin{DoxyParamCaption}
\item[{{\bf mat\_\-t} $\ast$}]{mat, }
\item[{void $\ast$}]{data, }
\item[{enum {\bf matio\_\-types}}]{data\_\-type, }
\item[{size\_\-t $\ast$}]{dims, }
\item[{int $\ast$}]{start, }
\item[{int $\ast$}]{stride, }
\item[{int $\ast$}]{edge}
\end{DoxyParamCaption}
)}}
\label{mat5_8c_a04eba34668a0de871c4f245702f802f7}
\hypertarget{mat5_8c_a80e4ba45ec110d05657f8f39ffd9ee27}{
\index{mat5.c@{mat5.c}!WriteInfo5@{WriteInfo5}}
\index{WriteInfo5@{WriteInfo5}!mat5.c@{mat5.c}}
\subsubsection[{WriteInfo5}]{\setlength{\rightskip}{0pt plus 5cm}void WriteInfo5 (
\begin{DoxyParamCaption}
\item[{{\bf mat\_\-t} $\ast$}]{mat, }
\item[{{\bf matvar\_\-t} $\ast$}]{matvar}
\end{DoxyParamCaption}
)}}
\label{mat5_8c_a80e4ba45ec110d05657f8f39ffd9ee27}

\hypertarget{matio_8h}{\section{/home/meldon/catkin\-\_\-ws/src/meldon\-\_\-detection/src/kdes/dependencies/matio/src/matio.h File Reference}
\label{matio_8h}\index{/home/meldon/catkin\-\_\-ws/src/meldon\-\_\-detection/src/kdes/dependencies/matio/src/matio.\-h@{/home/meldon/catkin\-\_\-ws/src/meldon\-\_\-detection/src/kdes/dependencies/matio/src/matio.\-h}}
}
{\ttfamily \#include $<$stdlib.\-h$>$}\\*
{\ttfamily \#include $<$stdio.\-h$>$}\\*
{\ttfamily \#include \char`\"{}matio\-\_\-pubconf.\-h\char`\"{}}\\*
{\ttfamily \#include $<$stdarg.\-h$>$}\\*
\subsection*{Classes}
\begin{DoxyCompactItemize}
\item 
struct \hyperlink{structComplexSplit}{Complex\-Split}
\begin{DoxyCompactList}\small\item\em Complex data type using split storage. \end{DoxyCompactList}\item 
struct \hyperlink{structmatvar__t}{matvar\-\_\-t}
\begin{DoxyCompactList}\small\item\em Matlab variable information. \end{DoxyCompactList}\item 
struct \hyperlink{structsparse__t}{sparse\-\_\-t}
\begin{DoxyCompactList}\small\item\em sparse data information \end{DoxyCompactList}\end{DoxyCompactItemize}
\subsection*{Macros}
\begin{DoxyCompactItemize}
\item 
\hypertarget{matio_8h_a77366c1bd428629dc898e188bfd182a3}{\#define {\bfseries E\-X\-T\-E\-R\-N}~extern}\label{matio_8h_a77366c1bd428629dc898e188bfd182a3}

\item 
\hypertarget{matio_8h_adba468ccee1d50707f3d89cb64e9f16b}{\#define \hyperlink{matio_8h_adba468ccee1d50707f3d89cb64e9f16b}{M\-E\-M\-\_\-\-C\-O\-N\-S\-E\-R\-V\-E}~1}\label{matio_8h_adba468ccee1d50707f3d89cb64e9f16b}

\begin{DoxyCompactList}\small\item\em Conserve memory. \end{DoxyCompactList}\item 
\hypertarget{matio_8h_a1b68dc49516be488aedd5056c3078cd9}{\#define {\bfseries Mat\-\_\-\-Create}(a, b)~\hyperlink{group__MAT_ga22d404f203af7869c841400e7ad247cf}{Mat\-\_\-\-Create\-Ver}(a,b,M\-A\-T\-\_\-\-F\-T\-\_\-\-D\-E\-F\-A\-U\-L\-T)}\label{matio_8h_a1b68dc49516be488aedd5056c3078cd9}

\end{DoxyCompactItemize}
\subsection*{Typedefs}
\begin{DoxyCompactItemize}
\item 
\hypertarget{group__MAT_gab0fc888f5a5d79943b16284b1f91c2e8}{typedef struct \hyperlink{struct__mat__t}{\-\_\-mat\-\_\-t} \hyperlink{group__MAT_gab0fc888f5a5d79943b16284b1f91c2e8}{mat\-\_\-t}}\label{group__MAT_gab0fc888f5a5d79943b16284b1f91c2e8}

\begin{DoxyCompactList}\small\item\em Matlab M\-A\-T File information Contains information about a Matlab M\-A\-T file. \end{DoxyCompactList}\item 
typedef struct \hyperlink{structmatvar__t}{matvar\-\_\-t} \hyperlink{group__MAT_ga24775c96a2a6d073581639c780b7896c}{matvar\-\_\-t}
\begin{DoxyCompactList}\small\item\em Matlab variable information. \end{DoxyCompactList}\item 
typedef struct \hyperlink{structsparse__t}{sparse\-\_\-t} \hyperlink{group__MAT_ga3ce6ed53a1909e27e92f3eaffc2f92ed}{sparse\-\_\-t}
\begin{DoxyCompactList}\small\item\em sparse data information \end{DoxyCompactList}\end{DoxyCompactItemize}
\subsection*{Enumerations}
\begin{DoxyCompactItemize}
\item 
enum \hyperlink{group__MAT_gaa9dcbc70f538af79bd557593ff6b5cdb}{mat\-\_\-acc} \{ \hyperlink{group__MAT_ggaa9dcbc70f538af79bd557593ff6b5cdba8dd1457651b27ba9bea6cfba158c037c}{M\-A\-T\-\_\-\-A\-C\-C\-\_\-\-R\-D\-O\-N\-L\-Y} = 0, 
\hyperlink{group__MAT_ggaa9dcbc70f538af79bd557593ff6b5cdba0f65f27ea42fde32d62b702b82329c1f}{M\-A\-T\-\_\-\-A\-C\-C\-\_\-\-R\-D\-W\-R} = 1
 \}
\begin{DoxyCompactList}\small\item\em M\-A\-T file access types. \end{DoxyCompactList}\item 
enum \hyperlink{group__MAT_gad03442b8378999189d510e3745c702b7}{mat\-\_\-ft} \{ \hyperlink{group__MAT_ggad03442b8378999189d510e3745c702b7a765c5d1d5038947646260dc82483517e}{M\-A\-T\-\_\-\-F\-T\-\_\-\-M\-A\-T73} = 0x0200, 
\hyperlink{group__MAT_ggad03442b8378999189d510e3745c702b7a31ade1f6989411dc0299007e2c7d33b2}{M\-A\-T\-\_\-\-F\-T\-\_\-\-M\-A\-T5} = 0x0100, 
\hyperlink{group__MAT_ggad03442b8378999189d510e3745c702b7a858b4f5da65548219b1c3ad47aa478d3}{M\-A\-T\-\_\-\-F\-T\-\_\-\-M\-A\-T4} = 0x0010
 \}
\begin{DoxyCompactList}\small\item\em M\-A\-T file versions. \end{DoxyCompactList}\item 
enum \hyperlink{group__MAT_gacf7b3b879282b7ab3a51190e49bf3453}{matio\-\_\-types} \{ \\*
\hyperlink{group__MAT_ggacf7b3b879282b7ab3a51190e49bf3453a2a7318fe8bf9464935e7ed8902618293}{M\-A\-T\-\_\-\-T\-\_\-\-U\-N\-K\-N\-O\-W\-N} = 0, 
\hyperlink{group__MAT_ggacf7b3b879282b7ab3a51190e49bf3453a9807f5033ed4f9b548953742d9fd1658}{M\-A\-T\-\_\-\-T\-\_\-\-I\-N\-T8} = 1, 
\hyperlink{group__MAT_ggacf7b3b879282b7ab3a51190e49bf3453a01c1bd7db68f90552862eb5d311be408}{M\-A\-T\-\_\-\-T\-\_\-\-U\-I\-N\-T8} = 2, 
\hyperlink{group__MAT_ggacf7b3b879282b7ab3a51190e49bf3453a8c5b2e381946e95ea8d81ac216743302}{M\-A\-T\-\_\-\-T\-\_\-\-I\-N\-T16} = 3, 
\\*
\hyperlink{group__MAT_ggacf7b3b879282b7ab3a51190e49bf3453a05bc7af7680aa68be95126ae0a4c2e31}{M\-A\-T\-\_\-\-T\-\_\-\-U\-I\-N\-T16} = 4, 
\hyperlink{group__MAT_ggacf7b3b879282b7ab3a51190e49bf3453a83e06a68320726c6572bfbb9f3addb1d}{M\-A\-T\-\_\-\-T\-\_\-\-I\-N\-T32} = 5, 
\hyperlink{group__MAT_ggacf7b3b879282b7ab3a51190e49bf3453aa397e285a23fe240368b752897652c6a}{M\-A\-T\-\_\-\-T\-\_\-\-U\-I\-N\-T32} = 6, 
\hyperlink{group__MAT_ggacf7b3b879282b7ab3a51190e49bf3453a3a3657d40e9212c923d9b9d03531b64c}{M\-A\-T\-\_\-\-T\-\_\-\-S\-I\-N\-G\-L\-E} = 7, 
\\*
\hyperlink{group__MAT_ggacf7b3b879282b7ab3a51190e49bf3453a31e721ecf7e188196f83c32838288797}{M\-A\-T\-\_\-\-T\-\_\-\-D\-O\-U\-B\-L\-E} = 9, 
\hyperlink{group__MAT_ggacf7b3b879282b7ab3a51190e49bf3453a9e825b5d18b8f946eaf2b4b57e51c145}{M\-A\-T\-\_\-\-T\-\_\-\-I\-N\-T64} = 12, 
\hyperlink{group__MAT_ggacf7b3b879282b7ab3a51190e49bf3453a45547932c46be27118abe08302d7e29f}{M\-A\-T\-\_\-\-T\-\_\-\-U\-I\-N\-T64} = 13, 
\hyperlink{group__MAT_ggacf7b3b879282b7ab3a51190e49bf3453a32985fee89a4df8db4b3f5d3a48823d3}{M\-A\-T\-\_\-\-T\-\_\-\-M\-A\-T\-R\-I\-X} = 14, 
\\*
\hyperlink{group__MAT_ggacf7b3b879282b7ab3a51190e49bf3453a30437f2eb3becc2fa6e5d96599d7f724}{M\-A\-T\-\_\-\-T\-\_\-\-C\-O\-M\-P\-R\-E\-S\-S\-E\-D} = 15, 
\hyperlink{group__MAT_ggacf7b3b879282b7ab3a51190e49bf3453ac34ad81f5cbd3b7d0d95e57e5be0149b}{M\-A\-T\-\_\-\-T\-\_\-\-U\-T\-F8} = 16, 
\hyperlink{group__MAT_ggacf7b3b879282b7ab3a51190e49bf3453a87ffc0412143c326a1fcc759d5d81bdc}{M\-A\-T\-\_\-\-T\-\_\-\-U\-T\-F16} = 17, 
\hyperlink{group__MAT_ggacf7b3b879282b7ab3a51190e49bf3453a11e43c0e0be79b1983090e02ae583109}{M\-A\-T\-\_\-\-T\-\_\-\-U\-T\-F32} = 18, 
\\*
\hyperlink{group__MAT_ggacf7b3b879282b7ab3a51190e49bf3453a9456a83c0b22022af42461a09d63cdb2}{M\-A\-T\-\_\-\-T\-\_\-\-S\-T\-R\-I\-N\-G} = 20, 
\hyperlink{group__MAT_ggacf7b3b879282b7ab3a51190e49bf3453a07599cf2cca6d2b2d059378563318ba5}{M\-A\-T\-\_\-\-T\-\_\-\-C\-E\-L\-L} = 21, 
\hyperlink{group__MAT_ggacf7b3b879282b7ab3a51190e49bf3453a4f4d5a6e1d42c6aa81ffb810e5da5c85}{M\-A\-T\-\_\-\-T\-\_\-\-S\-T\-R\-U\-C\-T} = 22, 
\hyperlink{group__MAT_ggacf7b3b879282b7ab3a51190e49bf3453acf106b0c23021582375f59bc9fce89b1}{M\-A\-T\-\_\-\-T\-\_\-\-A\-R\-R\-A\-Y} = 23, 
\\*
\hyperlink{group__MAT_ggacf7b3b879282b7ab3a51190e49bf3453ae76686f267dd1641cd55dce306af6d10}{M\-A\-T\-\_\-\-T\-\_\-\-F\-U\-N\-C\-T\-I\-O\-N} = 24
 \}
\begin{DoxyCompactList}\small\item\em Matlab data types. \end{DoxyCompactList}\item 
enum \hyperlink{group__MAT_gad4d60ae7b709fc81bfd744fb4c857c40}{matio\-\_\-classes} \{ \\*
\hyperlink{group__MAT_ggad4d60ae7b709fc81bfd744fb4c857c40a5c76eef0ca0373d25abe49053be6fa9a}{M\-A\-T\-\_\-\-C\-\_\-\-E\-M\-P\-T\-Y} = 0, 
\hyperlink{group__MAT_ggad4d60ae7b709fc81bfd744fb4c857c40a2f7abb47a1c51e248bd4e5e03cc81b08}{M\-A\-T\-\_\-\-C\-\_\-\-C\-E\-L\-L} = 1, 
\hyperlink{group__MAT_ggad4d60ae7b709fc81bfd744fb4c857c40acb467c7749c80902b798134c729bb521}{M\-A\-T\-\_\-\-C\-\_\-\-S\-T\-R\-U\-C\-T} = 2, 
\hyperlink{group__MAT_ggad4d60ae7b709fc81bfd744fb4c857c40afe45104b68b044c83a2f99e349fa1ea6}{M\-A\-T\-\_\-\-C\-\_\-\-O\-B\-J\-E\-C\-T} = 3, 
\\*
\hyperlink{group__MAT_ggad4d60ae7b709fc81bfd744fb4c857c40aacdec5834df0861130b393697646119c}{M\-A\-T\-\_\-\-C\-\_\-\-C\-H\-A\-R} = 4, 
\hyperlink{group__MAT_ggad4d60ae7b709fc81bfd744fb4c857c40a0d5655b7e6178a2242cb3bb56ff4c8d2}{M\-A\-T\-\_\-\-C\-\_\-\-S\-P\-A\-R\-S\-E} = 5, 
\hyperlink{group__MAT_ggad4d60ae7b709fc81bfd744fb4c857c40a5d70e0862e5bdb7bd86bf7ba5948f307}{M\-A\-T\-\_\-\-C\-\_\-\-D\-O\-U\-B\-L\-E} = 6, 
\hyperlink{group__MAT_ggad4d60ae7b709fc81bfd744fb4c857c40a2825631e26a961cbe0f79db50a39cea2}{M\-A\-T\-\_\-\-C\-\_\-\-S\-I\-N\-G\-L\-E} = 7, 
\\*
\hyperlink{group__MAT_ggad4d60ae7b709fc81bfd744fb4c857c40a984ff310f9e906100fcff95f704f43c5}{M\-A\-T\-\_\-\-C\-\_\-\-I\-N\-T8} = 8, 
\hyperlink{group__MAT_ggad4d60ae7b709fc81bfd744fb4c857c40a81270f8093cb4808e992c1d29d84d4e3}{M\-A\-T\-\_\-\-C\-\_\-\-U\-I\-N\-T8} = 9, 
\hyperlink{group__MAT_ggad4d60ae7b709fc81bfd744fb4c857c40a40370e9de516c5036a67a5865c071006}{M\-A\-T\-\_\-\-C\-\_\-\-I\-N\-T16} = 10, 
\hyperlink{group__MAT_ggad4d60ae7b709fc81bfd744fb4c857c40a8bede21dbf6c1edc0bbccc1481bccae7}{M\-A\-T\-\_\-\-C\-\_\-\-U\-I\-N\-T16} = 11, 
\\*
\hyperlink{group__MAT_ggad4d60ae7b709fc81bfd744fb4c857c40adb44fc39694e3152ae5e69470a2fefe8}{M\-A\-T\-\_\-\-C\-\_\-\-I\-N\-T32} = 12, 
\hyperlink{group__MAT_ggad4d60ae7b709fc81bfd744fb4c857c40a9a17a7edd45b19ef68197db81b27e816}{M\-A\-T\-\_\-\-C\-\_\-\-U\-I\-N\-T32} = 13, 
\hyperlink{group__MAT_ggad4d60ae7b709fc81bfd744fb4c857c40a1ea83bcde49b35477494412973f82409}{M\-A\-T\-\_\-\-C\-\_\-\-I\-N\-T64} = 14, 
\hyperlink{group__MAT_ggad4d60ae7b709fc81bfd744fb4c857c40a86470e25c3763d9a24623f04326195dd}{M\-A\-T\-\_\-\-C\-\_\-\-U\-I\-N\-T64} = 15, 
\\*
\hyperlink{group__MAT_ggad4d60ae7b709fc81bfd744fb4c857c40aaa9bf08312779cd1ab8e504a162ddcea}{M\-A\-T\-\_\-\-C\-\_\-\-F\-U\-N\-C\-T\-I\-O\-N} = 16
 \}
\begin{DoxyCompactList}\small\item\em Matlab variable classes. \end{DoxyCompactList}\item 
enum \hyperlink{group__MAT_gab9d6ef9e3ddca78a317b173f01d53fbb}{matio\-\_\-flags} \{ \hyperlink{group__MAT_ggab9d6ef9e3ddca78a317b173f01d53fbbacd7b091a11184aad7fc6078c04470780}{M\-A\-T\-\_\-\-F\-\_\-\-C\-O\-M\-P\-L\-E\-X} = 0x0800, 
\hyperlink{group__MAT_ggab9d6ef9e3ddca78a317b173f01d53fbba49084e0c796aa7963e53f7539525d40d}{M\-A\-T\-\_\-\-F\-\_\-\-G\-L\-O\-B\-A\-L} = 0x0400, 
\hyperlink{group__MAT_ggab9d6ef9e3ddca78a317b173f01d53fbba57eb5c6e200bcbc0f1b7982f29a169c2}{M\-A\-T\-\_\-\-F\-\_\-\-L\-O\-G\-I\-C\-A\-L} = 0x0200, 
\hyperlink{group__MAT_ggab9d6ef9e3ddca78a317b173f01d53fbba3a88beaec448e0485ffe21b18a540c1d}{M\-A\-T\-\_\-\-F\-\_\-\-C\-L\-A\-S\-S\-\_\-\-T} = 0x00ff
 \}
\begin{DoxyCompactList}\small\item\em Matlab array flags. \end{DoxyCompactList}\item 
enum \hyperlink{group__MAT_ga768c318af97bd2567758ecb001ceb7f4}{matio\-\_\-compression} \{ \hyperlink{group__MAT_gga768c318af97bd2567758ecb001ceb7f4ac549b871996d1ef05d40056bf5bb52e5}{C\-O\-M\-P\-R\-E\-S\-S\-I\-O\-N\-\_\-\-N\-O\-N\-E} = 0, 
\hyperlink{group__MAT_gga768c318af97bd2567758ecb001ceb7f4a1f453c9a2c01b52294b37a1226837f86}{C\-O\-M\-P\-R\-E\-S\-S\-I\-O\-N\-\_\-\-Z\-L\-I\-B} = 1
 \}
\begin{DoxyCompactList}\small\item\em Matlab compression options. \end{DoxyCompactList}\item 
enum \{ \hyperlink{group__MAT_gga06fc87d81c62e9abb8790b6e5713c55ba8938378c70879fe916177141cce0417e}{B\-Y\-\_\-\-N\-A\-M\-E} = 1, 
\hyperlink{group__MAT_gga06fc87d81c62e9abb8790b6e5713c55ba5f4d5606de1ec27f80f4a50186909005}{B\-Y\-\_\-\-I\-N\-D\-E\-X} = 2
 \}
\end{DoxyCompactItemize}
\subsection*{Functions}
\begin{DoxyCompactItemize}
\item 
E\-X\-T\-E\-R\-N char $\ast$ \hyperlink{group__mat__util_ga2b342987d3b664345cb233640b611fe9}{strdup\-\_\-vprintf} (const char $\ast$format, va\-\_\-list ap)
\begin{DoxyCompactList}\small\item\em Allocates and prints to a new string. \end{DoxyCompactList}\item 
E\-X\-T\-E\-R\-N char $\ast$ \hyperlink{group__mat__util_ga291b08f933c75fb70e3736b669896ebd}{strdup\-\_\-printf} (const char $\ast$format,...)
\begin{DoxyCompactList}\small\item\em Allocates and prints to a new string using printf format. \end{DoxyCompactList}\item 
E\-X\-T\-E\-R\-N int \hyperlink{group__mat__util_gaf348b811ee26bfc923924878cea3c9ba}{Mat\-\_\-\-Set\-Verbose} (int verb, int s)
\begin{DoxyCompactList}\small\item\em Sets verbose parameters. \end{DoxyCompactList}\item 
E\-X\-T\-E\-R\-N int \hyperlink{group__mat__util_gad75e2962dcaf2ac366f2420bb5b13094}{Mat\-\_\-\-Set\-Debug} (int d)
\begin{DoxyCompactList}\small\item\em Sets verbose parameters. \end{DoxyCompactList}\item 
E\-X\-T\-E\-R\-N void \hyperlink{group__mat__util_gaf51f2bfbb5580f575e4dd79757e2b80c}{Mat\-\_\-\-Critical} (const char $\ast$format,...)
\begin{DoxyCompactList}\small\item\em Logs a Critical message and returns to the user. \end{DoxyCompactList}\item 
E\-X\-T\-E\-R\-N void \hyperlink{group__mat__util_ga058b1cb9a4ca36712857d2b3c4de7ffc}{Mat\-\_\-\-Error} (const char $\ast$format,...)
\begin{DoxyCompactList}\small\item\em Logs a Critical message and aborts the program. \end{DoxyCompactList}\item 
E\-X\-T\-E\-R\-N void \hyperlink{group__mat__util_gaa4039c185e807ed2e9682b66fe2ea331}{Mat\-\_\-\-Help} (const char $\ast$helpstr\mbox{[}$\,$\mbox{]})
\begin{DoxyCompactList}\small\item\em Prints a helpstring to stdout and exits with status 1. \end{DoxyCompactList}\item 
E\-X\-T\-E\-R\-N int \hyperlink{group__mat__util_ga0d30e03216ceaab7c0a4ff878b26f89f}{Mat\-\_\-\-Log\-Init} (const char $\ast$progname)
\begin{DoxyCompactList}\small\item\em Intializes the logging system. \end{DoxyCompactList}\item 
E\-X\-T\-E\-R\-N int \hyperlink{group__mat__util_ga333d15dbd2e7a691621a2af8fc7adc3d}{Mat\-\_\-\-Log\-Close} (void)
\begin{DoxyCompactList}\small\item\em Closes the logging system. \end{DoxyCompactList}\item 
E\-X\-T\-E\-R\-N int \hyperlink{group__mat__util_ga93f4dd8d36413ae7f49260d757e3ab9f}{Mat\-\_\-\-Log\-Init\-Func} (const char $\ast$prog\-\_\-name, void($\ast$log\-\_\-func)(int log\-\_\-level, char $\ast$message))
\begin{DoxyCompactList}\small\item\em Intializes the logging system. \end{DoxyCompactList}\item 
E\-X\-T\-E\-R\-N int \hyperlink{group__mat__util_gae7dfa394b111bc908a616f8f5bddaa97}{Mat\-\_\-\-Message} (const char $\ast$format,...)
\begin{DoxyCompactList}\small\item\em Log a message unless silent. \end{DoxyCompactList}\item 
E\-X\-T\-E\-R\-N int \hyperlink{group__mat__util_ga26e00cfb07551be5201fd9e0f04066d9}{Mat\-\_\-\-Debug\-Message} (int level, const char $\ast$format,...)
\begin{DoxyCompactList}\small\item\em Log a message based on verbose level. \end{DoxyCompactList}\item 
E\-X\-T\-E\-R\-N int \hyperlink{group__mat__util_ga64a176ea7e27e38d4242a24f3e3bad24}{Mat\-\_\-\-Verb\-Message} (int level, const char $\ast$format,...)
\begin{DoxyCompactList}\small\item\em Log a message based on verbose level. \end{DoxyCompactList}\item 
E\-X\-T\-E\-R\-N void \hyperlink{group__mat__util_gafcedc83eb7e4759a8ea5c974c4f801c3}{Mat\-\_\-\-Warning} (const char $\ast$format,...)
\begin{DoxyCompactList}\small\item\em Prints a warning message to stdout. \end{DoxyCompactList}\item 
E\-X\-T\-E\-R\-N size\-\_\-t \hyperlink{group__mat__util_gab6774aabdc124c540c1e7686d0804940}{Mat\-\_\-\-Size\-Of} (enum \hyperlink{group__MAT_gacf7b3b879282b7ab3a51190e49bf3453}{matio\-\_\-types} data\-\_\-type)
\begin{DoxyCompactList}\small\item\em Calculate the size of M\-A\-T data types. \end{DoxyCompactList}\item 
E\-X\-T\-E\-R\-N size\-\_\-t \hyperlink{group__MAT_ga2bf682f015b22fa796a8885e997661e7}{Mat\-\_\-\-Size\-Of\-Class} (int class\-\_\-type)
\begin{DoxyCompactList}\small\item\em Returns the size of a Matlab Class. \end{DoxyCompactList}\item 
E\-X\-T\-E\-R\-N \hyperlink{group__MAT_gab0fc888f5a5d79943b16284b1f91c2e8}{mat\-\_\-t} $\ast$ \hyperlink{group__MAT_ga22d404f203af7869c841400e7ad247cf}{Mat\-\_\-\-Create\-Ver} (const char $\ast$matname, const char $\ast$hdr\-\_\-str, enum \hyperlink{group__MAT_gad03442b8378999189d510e3745c702b7}{mat\-\_\-ft} mat\-\_\-file\-\_\-ver)
\begin{DoxyCompactList}\small\item\em Creates a new Matlab M\-A\-T file. \end{DoxyCompactList}\item 
E\-X\-T\-E\-R\-N int \hyperlink{group__MAT_ga101c92ff7bde4a2d4615661beba09262}{Mat\-\_\-\-Close} (\hyperlink{group__MAT_gab0fc888f5a5d79943b16284b1f91c2e8}{mat\-\_\-t} $\ast$mat)
\begin{DoxyCompactList}\small\item\em Closes an open Matlab M\-A\-T file. \end{DoxyCompactList}\item 
E\-X\-T\-E\-R\-N \hyperlink{group__MAT_gab0fc888f5a5d79943b16284b1f91c2e8}{mat\-\_\-t} $\ast$ \hyperlink{group__MAT_gafbfedb5636a99f0ef867520c47f77d18}{Mat\-\_\-\-Open} (const char $\ast$matname, int mode)
\begin{DoxyCompactList}\small\item\em Opens an existing Matlab M\-A\-T file. \end{DoxyCompactList}\item 
E\-X\-T\-E\-R\-N int \hyperlink{group__MAT_ga4d6e3892d2e216c507a744ba0e070d0b}{Mat\-\_\-\-Rewind} (\hyperlink{group__MAT_gab0fc888f5a5d79943b16284b1f91c2e8}{mat\-\_\-t} $\ast$mat)
\begin{DoxyCompactList}\small\item\em Rewinds a Matlab M\-A\-T file to the first variable. \end{DoxyCompactList}\item 
E\-X\-T\-E\-R\-N \hyperlink{structmatvar__t}{matvar\-\_\-t} $\ast$ \hyperlink{group__MAT_gae7c9c3699f6e9c31a9c490300013098c}{Mat\-\_\-\-Var\-Calloc} (void)
\begin{DoxyCompactList}\small\item\em Allocates memory for a new \hyperlink{structmatvar__t}{matvar\-\_\-t} and initializes all the fields. \end{DoxyCompactList}\item 
E\-X\-T\-E\-R\-N \hyperlink{structmatvar__t}{matvar\-\_\-t} $\ast$ \hyperlink{group__MAT_ga1c54a84bb4d810c6fccdb8869489eac4}{Mat\-\_\-\-Var\-Create} (const char $\ast$name, enum \hyperlink{group__MAT_gad4d60ae7b709fc81bfd744fb4c857c40}{matio\-\_\-classes} class\-\_\-type, enum \hyperlink{group__MAT_gacf7b3b879282b7ab3a51190e49bf3453}{matio\-\_\-types} data\-\_\-type, int rank, size\-\_\-t $\ast$dims, void $\ast$data, int opt)
\begin{DoxyCompactList}\small\item\em Creates a M\-A\-T Variable with the given name and (optionally) data. \end{DoxyCompactList}\item 
E\-X\-T\-E\-R\-N int \hyperlink{group__MAT_gabf139e48d48177e5069338fa2919c60a}{Mat\-\_\-\-Var\-Delete} (\hyperlink{group__MAT_gab0fc888f5a5d79943b16284b1f91c2e8}{mat\-\_\-t} $\ast$mat, const char $\ast$name)
\begin{DoxyCompactList}\small\item\em Deletes a variable from a file. \end{DoxyCompactList}\item 
E\-X\-T\-E\-R\-N \hyperlink{structmatvar__t}{matvar\-\_\-t} $\ast$ \hyperlink{group__MAT_ga7ef80c5d99d7918b2b09db3bea106ecc}{Mat\-\_\-\-Var\-Duplicate} (const \hyperlink{structmatvar__t}{matvar\-\_\-t} $\ast$in, int opt)
\begin{DoxyCompactList}\small\item\em Duplicates a \hyperlink{structmatvar__t}{matvar\-\_\-t} structure. \end{DoxyCompactList}\item 
E\-X\-T\-E\-R\-N void \hyperlink{group__MAT_ga1d14716f7450530fd1c9d02413787f0e}{Mat\-\_\-\-Var\-Free} (\hyperlink{structmatvar__t}{matvar\-\_\-t} $\ast$matvar)
\begin{DoxyCompactList}\small\item\em Frees all the allocated memory associated with the structure. \end{DoxyCompactList}\item 
E\-X\-T\-E\-R\-N \hyperlink{structmatvar__t}{matvar\-\_\-t} $\ast$ \hyperlink{group__MAT_gac1e15063439c0bd3eb0c986514c742dc}{Mat\-\_\-\-Var\-Get\-Cell} (\hyperlink{structmatvar__t}{matvar\-\_\-t} $\ast$matvar, int index)
\begin{DoxyCompactList}\small\item\em Returns a pointer to the Cell array at a specific index. \end{DoxyCompactList}\item 
E\-X\-T\-E\-R\-N \hyperlink{structmatvar__t}{matvar\-\_\-t} $\ast$$\ast$ \hyperlink{group__MAT_ga0732b0a6c40975b036068b9a14422d45}{Mat\-\_\-\-Var\-Get\-Cells} (\hyperlink{structmatvar__t}{matvar\-\_\-t} $\ast$matvar, int $\ast$start, int $\ast$stride, int $\ast$edge)
\begin{DoxyCompactList}\small\item\em Indexes a cell array. \end{DoxyCompactList}\item 
E\-X\-T\-E\-R\-N \hyperlink{structmatvar__t}{matvar\-\_\-t} $\ast$$\ast$ \hyperlink{group__MAT_ga004987d665654409f74eaf8e82bb1380}{Mat\-\_\-\-Var\-Get\-Cells\-Linear} (\hyperlink{structmatvar__t}{matvar\-\_\-t} $\ast$matvar, int start, int stride, int edge)
\begin{DoxyCompactList}\small\item\em Indexes a cell array. \end{DoxyCompactList}\item 
E\-X\-T\-E\-R\-N size\-\_\-t \hyperlink{group__MAT_gaeeb798fead2f765bddfb19016c7fdbcc}{Mat\-\_\-\-Var\-Get\-Size} (\hyperlink{structmatvar__t}{matvar\-\_\-t} $\ast$matvar)
\begin{DoxyCompactList}\small\item\em Calculates the size of a matlab variable in bytes. \end{DoxyCompactList}\item 
E\-X\-T\-E\-R\-N int \hyperlink{group__MAT_ga56b9a545990a0f253164018e37111741}{Mat\-\_\-\-Var\-Get\-Number\-Of\-Fields} (\hyperlink{structmatvar__t}{matvar\-\_\-t} $\ast$matvar)
\begin{DoxyCompactList}\small\item\em Returns the number of fields in a structure variable. \end{DoxyCompactList}\item 
E\-X\-T\-E\-R\-N int \hyperlink{group__MAT_ga9f8ab8a7e4206c16cb20437acc6960d2}{Mat\-\_\-\-Var\-Add\-Struct\-Field} (\hyperlink{structmatvar__t}{matvar\-\_\-t} $\ast$matvar, \hyperlink{structmatvar__t}{matvar\-\_\-t} $\ast$$\ast$fields)
\begin{DoxyCompactList}\small\item\em Adds a field to a structure. \end{DoxyCompactList}\item 
E\-X\-T\-E\-R\-N \hyperlink{structmatvar__t}{matvar\-\_\-t} $\ast$ \hyperlink{group__MAT_ga7018bfe6934b96ae32e8f2a1712eefab}{Mat\-\_\-\-Var\-Get\-Struct\-Field} (\hyperlink{structmatvar__t}{matvar\-\_\-t} $\ast$matvar, void $\ast$name\-\_\-or\-\_\-index, int opt, int index)
\begin{DoxyCompactList}\small\item\em Finds a field of a structure. \end{DoxyCompactList}\item 
E\-X\-T\-E\-R\-N \hyperlink{structmatvar__t}{matvar\-\_\-t} $\ast$ \hyperlink{group__MAT_ga509178d7dc15faf9f7cd0440df6009b9}{Mat\-\_\-\-Var\-Get\-Structs} (\hyperlink{structmatvar__t}{matvar\-\_\-t} $\ast$matvar, int $\ast$start, int $\ast$stride, int $\ast$edge, int copy\-\_\-fields)
\begin{DoxyCompactList}\small\item\em Indexes a structure. \end{DoxyCompactList}\item 
E\-X\-T\-E\-R\-N \hyperlink{structmatvar__t}{matvar\-\_\-t} $\ast$ \hyperlink{group__MAT_gaa56680fb7b2cd3d410f659e945da8141}{Mat\-\_\-\-Var\-Get\-Structs\-Linear} (\hyperlink{structmatvar__t}{matvar\-\_\-t} $\ast$matvar, int start, int stride, int edge, int copy\-\_\-fields)
\begin{DoxyCompactList}\small\item\em Indexes a structure. \end{DoxyCompactList}\item 
E\-X\-T\-E\-R\-N void \hyperlink{group__MAT_ga9100c145e338b84b55d5d0795d5d390a}{Mat\-\_\-\-Var\-Print} (\hyperlink{structmatvar__t}{matvar\-\_\-t} $\ast$matvar, int printdata)
\begin{DoxyCompactList}\small\item\em Prints the variable information. \end{DoxyCompactList}\item 
E\-X\-T\-E\-R\-N \hyperlink{structmatvar__t}{matvar\-\_\-t} $\ast$ \hyperlink{group__MAT_ga3505f63029763eaa73d5a19f1115eb42}{Mat\-\_\-\-Var\-Read} (\hyperlink{group__MAT_gab0fc888f5a5d79943b16284b1f91c2e8}{mat\-\_\-t} $\ast$mat, const char $\ast$name)
\begin{DoxyCompactList}\small\item\em Reads the variable with the given name from a M\-A\-T file. \end{DoxyCompactList}\item 
E\-X\-T\-E\-R\-N int \hyperlink{group__MAT_ga1845000f4fc6252ec5ff11c4b9f0759f}{Mat\-\_\-\-Var\-Read\-Data} (\hyperlink{group__MAT_gab0fc888f5a5d79943b16284b1f91c2e8}{mat\-\_\-t} $\ast$mat, \hyperlink{structmatvar__t}{matvar\-\_\-t} $\ast$matvar, void $\ast$data, int $\ast$start, int $\ast$stride, int $\ast$edge)
\begin{DoxyCompactList}\small\item\em Reads M\-A\-T variable data from a file. \end{DoxyCompactList}\item 
E\-X\-T\-E\-R\-N int \hyperlink{group__MAT_gaa8060d7c8e5da0aa9ee5f96e5f1eb30a}{Mat\-\_\-\-Var\-Read\-Data\-All} (\hyperlink{group__MAT_gab0fc888f5a5d79943b16284b1f91c2e8}{mat\-\_\-t} $\ast$mat, \hyperlink{structmatvar__t}{matvar\-\_\-t} $\ast$matvar)
\begin{DoxyCompactList}\small\item\em Reads all the data for a matlab variable. \end{DoxyCompactList}\item 
E\-X\-T\-E\-R\-N int \hyperlink{group__MAT_gaad61c8449a2106afa697280ff0ee9dd8}{Mat\-\_\-\-Var\-Read\-Data\-Linear} (\hyperlink{group__MAT_gab0fc888f5a5d79943b16284b1f91c2e8}{mat\-\_\-t} $\ast$mat, \hyperlink{structmatvar__t}{matvar\-\_\-t} $\ast$matvar, void $\ast$data, int start, int stride, int edge)
\begin{DoxyCompactList}\small\item\em Reads M\-A\-T variable data from a file. \end{DoxyCompactList}\item 
E\-X\-T\-E\-R\-N \hyperlink{structmatvar__t}{matvar\-\_\-t} $\ast$ \hyperlink{group__MAT_ga46da2e45ed96d3f1a6ec643757f2b086}{Mat\-\_\-\-Var\-Read\-Info} (\hyperlink{group__MAT_gab0fc888f5a5d79943b16284b1f91c2e8}{mat\-\_\-t} $\ast$mat, const char $\ast$name)
\begin{DoxyCompactList}\small\item\em Reads the information of a variable with the given name from a M\-A\-T file. \end{DoxyCompactList}\item 
E\-X\-T\-E\-R\-N \hyperlink{structmatvar__t}{matvar\-\_\-t} $\ast$ \hyperlink{group__MAT_ga7c321d6aafd93916ba6c5655ad78e9ca}{Mat\-\_\-\-Var\-Read\-Next} (\hyperlink{group__MAT_gab0fc888f5a5d79943b16284b1f91c2e8}{mat\-\_\-t} $\ast$mat)
\begin{DoxyCompactList}\small\item\em Reads the next variable in a M\-A\-T file. \end{DoxyCompactList}\item 
E\-X\-T\-E\-R\-N \hyperlink{structmatvar__t}{matvar\-\_\-t} $\ast$ \hyperlink{group__MAT_ga72dd99330507b17177e22f9ed3bea5e6}{Mat\-\_\-\-Var\-Read\-Next\-Info} (\hyperlink{group__MAT_gab0fc888f5a5d79943b16284b1f91c2e8}{mat\-\_\-t} $\ast$mat)
\begin{DoxyCompactList}\small\item\em Reads the information of the next variable in a M\-A\-T file. \end{DoxyCompactList}\item 
E\-X\-T\-E\-R\-N int \hyperlink{group__MAT_ga77c5ad24d45047830046fe3ed25da8ad}{Mat\-\_\-\-Var\-Write} (\hyperlink{group__MAT_gab0fc888f5a5d79943b16284b1f91c2e8}{mat\-\_\-t} $\ast$mat, \hyperlink{structmatvar__t}{matvar\-\_\-t} $\ast$matvar, int compress)
\begin{DoxyCompactList}\small\item\em Writes the given M\-A\-T variable to a M\-A\-T file. \end{DoxyCompactList}\item 
E\-X\-T\-E\-R\-N int \hyperlink{group__MAT_ga1ae164415dfd98cdf48ad07033b6e0bb}{Mat\-\_\-\-Var\-Write\-Info} (\hyperlink{group__MAT_gab0fc888f5a5d79943b16284b1f91c2e8}{mat\-\_\-t} $\ast$mat, \hyperlink{structmatvar__t}{matvar\-\_\-t} $\ast$matvar)
\begin{DoxyCompactList}\small\item\em Writes the given M\-A\-T variable to a M\-A\-T file. \end{DoxyCompactList}\item 
E\-X\-T\-E\-R\-N int \hyperlink{group__MAT_ga43179b930fb30c025a153a55a083a98a}{Mat\-\_\-\-Var\-Write\-Data} (\hyperlink{group__MAT_gab0fc888f5a5d79943b16284b1f91c2e8}{mat\-\_\-t} $\ast$mat, \hyperlink{structmatvar__t}{matvar\-\_\-t} $\ast$matvar, void $\ast$data, int $\ast$start, int $\ast$stride, int $\ast$edge)
\begin{DoxyCompactList}\small\item\em Writes the given data to the M\-A\-T variable. \end{DoxyCompactList}\item 
E\-X\-T\-E\-R\-N int \hyperlink{group__MAT_ga9b8d09f631538b14ca29792e0334e349}{Mat\-\_\-\-Calc\-Single\-Subscript} (int rank, int $\ast$dims, int $\ast$subs)
\begin{DoxyCompactList}\small\item\em Calculate a single subscript from a set of subscript values. \end{DoxyCompactList}\item 
E\-X\-T\-E\-R\-N int $\ast$ \hyperlink{group__MAT_gabe2571a4b9b6cff3b31aa6f152deba61}{Mat\-\_\-\-Calc\-Subscripts} (int rank, int $\ast$dims, int index)
\begin{DoxyCompactList}\small\item\em Calculate a set of subscript values from a single(linear) subscript. \end{DoxyCompactList}\end{DoxyCompactItemize}


\subsection{Detailed Description}
L\-I\-B\-M\-A\-T\-I\-O Header 
\hypertarget{read__data_8c}{\section{/home/brawner/catkin\-\_\-ws/src/baxter\-\_\-h2r\-\_\-packages/meldon\-\_\-detection/src/kdes/dependencies/matio/src/read\-\_\-data.c File Reference}
\label{read__data_8c}\index{/home/brawner/catkin\-\_\-ws/src/baxter\-\_\-h2r\-\_\-packages/meldon\-\_\-detection/src/kdes/dependencies/matio/src/read\-\_\-data.\-c@{/home/brawner/catkin\-\_\-ws/src/baxter\-\_\-h2r\-\_\-packages/meldon\-\_\-detection/src/kdes/dependencies/matio/src/read\-\_\-data.\-c}}
}
{\ttfamily \#include $<$stdlib.\-h$>$}\\*
{\ttfamily \#include $<$string.\-h$>$}\\*
{\ttfamily \#include $<$stdio.\-h$>$}\\*
{\ttfamily \#include $<$math.\-h$>$}\\*
{\ttfamily \#include $<$time.\-h$>$}\\*
{\ttfamily \#include \char`\"{}matio\-\_\-private.\-h\char`\"{}}\\*
{\ttfamily \#include $<$zlib.\-h$>$}\\*


\subsection{Detailed Description}
Matlab M\-A\-T version 5 file functions 
\addcontentsline{toc}{part}{Index}
\printindex
\end{document}
